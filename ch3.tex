%%%%%%%%%%%%%%%%%%%%%%%%%%%%%%%%%%%%%%%%%%%%%%%%%%%%%%%%%%%%%%%%%%%
%  File name: ch3.tex
%  Title:
%  Version: 12.09.2015 (hve)
%%%%%%%%%%%%%%%%%%%%%%%%%%%%%%%%%%%%%%%%%%%%%%%%%%%%%%%%%%%%%%%%%%%
%%%%%%%%%%%%%%%%%%%%%%%%%%%%%%%%%%%%%%%%%%%%%%%%%%%%%%%%%%%%%%%%%%%
\chapter[Lineare Gleichungssysteme]{Lineare Gleichungssysteme}
\lb{ch3}
%%%%%%%%%%%%%%%%%%%%%%%%%%%%%%%%%%%%%%%%%%%%%%%%%%%%%%%%%%%%%%%%%%%
\hfill\hbox{\fbox{\vbox{\hsize=10cm
Der Inhalt dieses Kapitels ist relevant f\"ur das
Verst\"andnis der in Kapitel 5.2
des Lehrbuchs von Schrey\"ogg und Koch (2010)~\ct{schkoc2010}
behandelten quantitativen Themen.
%Gleichfalls f\"ur Kapitel 10.5, 11.4 des Lehrbuchs von
%Schmalen und Pechtl 2006~\ct{schpec2006}.
}}}

\vspace{10mm}
\noindent
In diesem Kapitel wenden wir uns nun einem anwendungsbezogenen
Einsatzgebiet von Matrizen und Vektoren, also von linearen
Abbildungen, zu.

%%%%%%%%%%%%%%%%%%%%%%%%%%%%%%%%%%%%%%%%%%%%%%%%%%%%%%%%%%%%%%%%%%%
\section[Grundbegriffe]%
{Grundbegriffe}
\lb{sec:lgsgrund}
%%%%%%%%%%%%%%%%%%%%%%%%%%%%%%%%%%%%%%%%%%%%%%%%%%%%%%%%%%%%%%%%%%%
Ausgangspunkt ist ein System von $m$ Gleichungen, welche, jede
f\"ur sich, eine {\em lineare Bedingung\/} an die Werte von $n$
Variablen $x_{1}, %\ldots, x_{j},
\ldots, x_{n} \in \mathbb{R}$ stellt. Diese Bedingungen
sind simultan zu erf\"ullen. Wir finden Probleme dieser Art,
so genannte {\bf lineare Gleichungssysteme (LGS)} (engl.: 
systems of linear equations), h\"aufig in der Form geschrieben:

\medskip
\noindent
$\bullet$ Darstellung 1:\\[-7mm]
%
\bea
a_{11}x_{1}+\ldots+a_{1j}x_{j}+\ldots+a_{1n}x_{n} & = & b_{1} 
\nonumber \\
%
 & \vdots & \nonumber \\
%
a_{i1}x_{1}+\ldots+a_{ij}x_{j}+\ldots+a_{in}x_{n} & = & b_{i} \\
%
 & \vdots & \nonumber \\
%
a_{m1}x_{1}+\ldots+a_{mj}x_{j}+\ldots+a_{mn}x_{n} & = & b_{m}
\nonumber \ .
\eea
%
Je nach Verh\"altnis der das Format des LGS festlegenden
nat\"urlichen Zahlen $m$ und $n$ zueinander, werden LGS wie
folgt klassifiziert:\\[-7mm]
%
\begin{itemize}
\item $m < n$: weniger Gleichungen als Variablen;
das LGS ist {\bf unterbestimmt} \\
(engl.: under-determined),\\[-7mm]

\item $m = n$: gleich viel Gleichungen wie Variablen;
das LGS ist {\bf wohlbestimmt} \\
(engl.: well-determined),\\[-7mm]

\item $m > n$: mehr Gleichungen als Variablen;
das LGS ist {\bf \"uberbestimmt} \\
(engl.: over-determined).\\[-7mm]
\end{itemize}
%
Eine kompaktere Darstellungsweise eines LGS vom Format
$(m \times n)$ liefert die folgende Variante:

\medskip
\noindent
$\bullet$ Darstellung 2:
%
\be
\fbox{$\displaystyle
\mathbf{A}\vec{x} =
\left(\begin{array}{ccccc}
a_{11} & \ldots & a_{1j} & \ldots & a_{1n} \\
\vdots & \ddots & \vdots & \ddots & \vdots \\
a_{i1} & \ldots & a_{ij} & \ldots & a_{in} \\
\vdots & \ddots & \vdots & \ddots & \vdots \\
a_{m1} & \ldots & a_{mj} & \ldots & a_{mn}
\end{array}\right)
\left(\begin{array}{c}
x_{1} \\
\vdots \\
x_{j} \\
\vdots \\
x_{n}
\end{array}\right)
= \left(
\begin{array}{c}
b_{1} \\
\vdots \\
b_{i} \\
\vdots \\
b_{m}
\end{array}\right)
= \vec{b} \ .
$}
\ee
%
Hierbei haben wir uns folgender mathematischer Objekte
zur B\"undelung relevanter Informationen bedient: $\mathbf{A}$
ist die so genannte {\bf Koeffizientenmatrix} (engl.: coefficient 
matrix) des LGS, vom Format $(m \times n)$, $\vec{x}$ der {\bf 
Variablenvektor} (engl.: variable vector), vom Format $(n \times 
1)$, und $\vec{b}$ der {\bf Absolutgliedvektor} (engl.: image 
vector), vom Format $(m \times 1)$.

\medskip
\noindent
Die zentrale Frage bei der Besch\"aftigung mit LGS,
in dieser zweiten Darstellung, lautet:

\medskip
\noindent
\underline{\bf Frage:}
%Liegt der Vektor $\vec{b}$ im Bildraum der durch die
%Matrix $\mathbf{A}$ definierten linearen Abbildung?
Gibt es einen {\bf Objektvektor} $\vec{x}$, der sich durch die
vorgegebene {\bf Matrix} $\mathbf{A}$ auf den vorgegebenen
{\bf Bildvektor} $\vec{b}$ abbilden l\"asst?

\medskip
\noindent
Erfreulicherweise steht uns eine einfache systematische Methode
zur Verf\"ugung, diese Frage zu beantworten, also LGS zu l\"osen.
Es handelt sich hierbei um einen Algorithmus, der von dem
deutschen Mathematiker und Astronomen
\href{http://www-groups.dcs.st-and.ac.uk/~history/Biographies/Gauss.html}{Carl
Friedrich Gau\ss\ (1777--1855)} entwickelt wurde.

%%%%%%%%%%%%%%%%%%%%%%%%%%%%%%%%%%%%%%%%%%%%%%%%%%%%%%%%%%%%%%%%%%%
\section[Gau\ss scher Algorithmus]%
{Gau\ss scher Algorithmus}
\lb{sec:lgsgauss}
%%%%%%%%%%%%%%%%%%%%%%%%%%%%%%%%%%%%%%%%%%%%%%%%%%%%%%%%%%%%%%%%%%%
Der von Gau\ss\ entwickelte L\"osungsalgorithmus (engl.: Gau\ss 
ian elimination) basiert auf der Einsicht: die L\"osungsmenge 
eines {\bf linearen Gleichungssystems} von $m$ Gleichungen f\"{u}r $n$ Variablen,
%
\be
\fbox{$\displaystyle
\mathbf{A}\vec{x} = \vec{b} \ ,
$}
\ee
%
bleibt durch folgende {\bf \"aquivalente Umformungen} (engl.: 
algebraic equivalence transformations) unver\"andert:
%
\begin{enumerate}
\item Vertauschen zweier Gleichungen.
\item Multiplizieren einer Gleichung mit einer reellen Zahl
$c \neq 0$.
\item Addieren eines Vielfachen einer Gleichung zu einer zweiten.
\item \"Andern der Reihenfolge der Variablen.
\end{enumerate}
%
Letztendlich bedeutet dies, wir k\"onnen ein gegebenes LGS
mithilfe dieser vier \"Aquivalenzschritte beliebig manipulieren,
ohne seinen Gehalt, sprich: seine Identit\"at, zu ver\"andern.
Nat\"urlich sollen im konkreten Fall die vier \"Aquivalenzschritte
nicht wahllos zum Einsatz kommen; der Gau\ss sche Algorithmus
zeichnet eine zielgerichtete L\"osungsstrategie vor.

\medskip
\noindent
\underline{\bf Ziel:} Die {\bf erweiterte Koeffizientenmatrix}
$(\mathbf{A}|\vec{b})$ (engl.: augmented coefficient matrix), also 
das Zahlenschema
%
\be
\begin{array}{ccccc|c}
a_{11} & \ldots & a_{1j} & \ldots & a_{1n} & b_{1} \\
\vdots & \ddots & \vdots & \ddots & \vdots & \vdots \\
a_{i1} & \ldots & a_{ij} & \ldots & a_{in} & b_{i} \\
\vdots & \ddots & \vdots & \ddots & \vdots & \vdots \\
a_{m1} & \ldots & a_{mj} & \ldots & a_{mn} & b_{m}
\end{array} \ ,
\ee
%
soll, wenn m\"oglich, durch Anwendung von \"aquivalenten
Umformungen auf so genannte {\bf obere Dreiecksgestalt} (engl.: 
upper triangular form)
%
\be
\begin{array}{ccccc|c}
1 & \ldots & \tilde{a}_{1j} & \ldots & \tilde{a}_{1n} & \tilde{b}_{1} \\
\vdots & \ddots & \vdots & \ddots & \vdots & \vdots \\
0 & \ldots & \tilde{a}_{ij} & \ldots & \tilde{a}_{in} & \tilde{b}_{i} \\
\vdots & \ddots & \vdots & \ddots & \vdots & \vdots \\
0 & \ldots & 0 & \ldots & \tilde{a}_{mn} & \tilde{b}_{m}
\end{array}
\ee
%
transformiert werden, um sodann das diesem Endschema unterliegende
einfachere lineare Gleichungssystem durch {\bf r\"uckwertige
Substitution} (engl.: backward substitution) explizit zu l\"osen.

\medskip
\noindent
Drei F\"alle {\bf resultierender L\"osungsm\"oglichkeiten}
k\"onnen nun auftreten. Das LGS hat entweder
%
\begin{enumerate}
\item {\em keine\/} L\"osung (engl.: no solution), oder eine

\item {\em eindeutige\/} L\"osung (engl.: unique solution), oder 
eine

\item {\em mehrdeutige\/} L\"osung (engl.: multiple solutions).
\end{enumerate}
%
\underline {\bf Bem:} Unterbestimmte LGS, also wenn $m < n$,
sind niemals eindeutig l\"osbar. Denn es gibt dann zu wenig
Bedingungen, um den Wert einer jeden der $n$ Variablen
festzulegen.

\medskip
\noindent
\underline {\bf GTR:} F\"ur eine eingespeicherte erweiterte
Koeffizientenmatrix ${\tt [A]}$ vom Format $(m \times n+1)$
eines gegebenen LGS vom Format $(m \times n)$ kann im Modus
{\tt MATRIX} $\rightarrow$ {\tt MATH} die Funktion
${\tt rref([A])}$ angewendet werden. Je nach L\"osungsfall
ist die r\"uckwertige Substitution m\"oglicherweise von
Hand durchzuf\"uhren.

\medskip
\noindent
Der Vollst\"andigkeit halber wollen wir uns kurz der Theorie
der L\"osbarkeit eines LGS vom Format $(m \times n)$ zuwenden.
Hierzu ben\"otigen wir den im n\"achsten Abschnitt definierten
Begriff.

%%%%%%%%%%%%%%%%%%%%%%%%%%%%%%%%%%%%%%%%%%%%%%%%%%%%%%%%%%%%%%%%%%%
\section[Rang einer Matrix]%
{Rang einer Matrix}
\lb{sec:lgsrang}
%%%%%%%%%%%%%%%%%%%%%%%%%%%%%%%%%%%%%%%%%%%%%%%%%%%%%%%%%%%%%%%%%%%

\noindent
\underline{\bf Def.:}
Eine Matrix $\mathbf{A} \in \mathbb{R}^{m \times n}$ besitzt den
{\bf Rang} (engl.: rank)
%
\be
\fbox{$\displaystyle
\text{Rg}(\mathbf{A}) = r \ , \qquad
r \leq \text{min}\{m,n\}
$}
\ee
%
genau dann, wenn $r$ die {\bf maximale Anzahl} voneinander
linear unabh\"angiger Zeilen- bzw.\ Spaltenvektoren von
$\mathbf{A}$ ist. $r$ kann nur so gro\ss\ sein, wie die kleinere
der das Format von $\mathbf{A}$ bestimmenden Zahlen $m$ und $n$.

\medskip
\noindent
F\"ur {\bf quadratische Matrizen} $\mathbf{A} \in
\mathbb{R}^{n \times n}$ kann ein alternatives, eleganteres
Kriterium zur Bestimmung des Ranges der Matrix $\mathbf{A}$
herangezogen werden. Man arbeitet dabei mit einer
(z.B. mit dem GTR zu berechnenden) Kennzahl, die
die {\bf Determinante} (engl.: determinant) der quadratischen 
Matrix $\mathbf{A}$, $\det(\mathbf{A})$, genannt wird.

\medskip
\noindent
\underline{\bf Def.:}
%
\begin{itemize}
\item[(i)]~Wenn $\mathbf{A} \in \mathbb{R}^{2 \times 2}$,
ist die {\bf Determinante} von $\mathbf{A}$ durch
%
\be
\det(\mathbf{A})
:= \left|\begin{array}{cc}
   	a_{11} & a_{12} \\
   	a_{21} & a_{22}
	\end{array}\right|
:= a_{11}a_{22} - a_{12}a_{21}
\ee
%
definiert, also durch die Differenz der Produkte der Hauptdiagonal-
und der Nebendiagonalelemente von $\mathbf{A}$.

\item[(ii)]~Falls $\mathbf{A} \in \mathbb{R}^{3 \times 3}$, gilt
%
\bea
\det(\mathbf{A})
& := & \left|\begin{array}{ccc}
   	a_{11} & a_{12} & a_{13} \\
   	a_{21} & a_{22} & a_{23} \\
   	a_{31} & a_{32} & a_{33}
	\end{array}\right| \nonumber \\
	%
& := & a_{11}(a_{22}a_{33}-a_{32}a_{23})
+ a_{21}(a_{32}a_{13}-a_{12}a_{33})
+ a_{31}(a_{12}a_{23}-a_{22}a_{13}) \ .
\eea
%
Beachte hierbei das zyklische Vertauschen von Term zu Term
jeweils des ersten Index eines Elements $a_{ij}$ von $\mathbf{A}$
nach der Regel $1 \rightarrow 2 \rightarrow 3 \rightarrow 1$.

\item[(iii)]~Die Definition der {\bf 
Determinante}~$\det(\mathbf{A})$ f\"ur den allgemeinen Fall 
$\mathbf{A} \in \mathbb{R}^{n \times n}$ ist z.B. in Bronstein 
\emph{et al} (2005)~\ct[S.~267]{broetal2005} zu finden.
\end{itemize}
%

\medskip
\noindent
Zur Bestimmung des Rangs einer quadratischen Matrix $\mathbf{A}
\in \mathbb{R}^{n \times n}$ wird nun festgelegt:
$\text{Rg}(\mathbf{A}) = r = n$, falls
$\det(\mathbf{A}) \neq 0$,
und $\text{Rg}(\mathbf{A}) = r < n$, falls
$\det(\mathbf{A}) = 0$. Im ersten Fall hei\ss t $\mathbf{A}$
{\bf regul\"ar} (engl.: regular), im zweiten Fall {\bf singul\"ar} 
(engl.: singular). F\"ur singul\"are quadratische Matrizen 
$\mathbf{A}$ ist der Rang $\text{Rg}(\mathbf{A}) = r$ (mit $r < 
n$) bestimmt durch die Anzahl $r$ der Zeilen (Spalten) der 
gr\"o\ss tm\"oglichen von Null verschiedenen Unterdeterminante von 
$\mathbf{A}$.

\medskip
\noindent
\underline {\bf GTR:} F\"ur eine eingespeicherte quadratische
Matrix ${\tt [A]}$ kann im Modus {\tt MATRIX} $\rightarrow$
{\tt MATH} ihre Determinante leicht \"uber die Funktion
${\tt det([A])}$ berechnet werden.

%%%%%%%%%%%%%%%%%%%%%%%%%%%%%%%%%%%%%%%%%%%%%%%%%%%%%%%%%%%%%%%%%%%
\section[L\"{o}sbarkeitskriterien]%
{L\"{o}sbarkeitskriterien f\"{u}r lineare Gleichungssysteme}
\lb{sec:lgsloes}
%%%%%%%%%%%%%%%%%%%%%%%%%%%%%%%%%%%%%%%%%%%%%%%%%%%%%%%%%%%%%%%%%%%
Auf die Theorie der L\"osbarkeit eines LGS vom Format
$(m \times n)$ kann nun kurz in tabellarischer Form eingegangen
werden. F\"ur lineare Gleichungssysteme
%
$$
\mathbf{A}\vec{x} = \vec{b}
$$
%
mit $\mathbf{A} \in \mathbb{R}^{m \times n}$, $\vec{x} \in
\mathbb{R}^{n \times 1}$ und $\vec{b} \in \mathbb{R}^{m \times 1}$,
gilt:

%
\begin{center}
	\begin{tabular}[h]{c|c|c}
		\hline\hline
		 & & \\
		 & $\vec{b} \neq \vec{0}$ &
		 $\vec{b} = \vec{0}$ \\
		 & & \\
		\hline
		 & & \\
		1. $\text{Rg}(\mathbf{A}) \neq \text{Rg}(\mathbf{A}|\vec{b})$ &
		unl\"osbar & -------- \\
		 & & \\
		\hline
		 & & \\
		2. $\text{Rg}(\mathbf{A}) = \text{Rg}(\mathbf{A}|\vec{b}) = r$ &
		& \\
		 & & \\
		(a) \quad $r = n$ & eindeutig & $\vec{x} = \vec{0}$ \\
		 & l\"osbar & \\
		 & & \\
		(b) \quad $r < n$ & mehrdeutig & mehrdeutig \\
		 & l\"osbar, & l\"osbar, \\
		 & $n-r$ freie &  $n-r$ freie \\
		 & Parameter & Parameter \\
		 & & \\
		\hline\hline
	 \end{tabular}
\end{center}
%

\noindent
$(\mathbf{A}|\vec{b})$ bezeichnet die {\bf erweiterte
Koeffizientenmatrix} (engl.: augmented coefficient matrix).

\medskip
\noindent
Dieses Kapitel abschlie\ss end wenden wir uns nun einer besonderen,
f\"ur Anwendungen \"au\ss erst n\"utzlichen Eigenschaft der
Klasse von {\em regul\"aren\/} quadratischen Matrizen zu.

%%%%%%%%%%%%%%%%%%%%%%%%%%%%%%%%%%%%%%%%%%%%%%%%%%%%%%%%%%%%%%%%%%%
\section[Inverse einer regul\"{a}ren $(n\times n)$-Matrix]%
{Inverse einer regul\"{a}ren $\boldsymbol{(n\times n)}$-Matrix}
\lb{sec:lgsinv}
%%%%%%%%%%%%%%%%%%%%%%%%%%%%%%%%%%%%%%%%%%%%%%%%%%%%%%%%%%%%%%%%%%%

\noindent
\underline{\bf Def.:}
Sei $\mathbf{A} \in
\mathbb{R}^{n \times n}$ eine {\bf regul\"are}
quadratische Matrix, d.h. $\det(\mathbf{A}) \neq 0$. Dann ist
die {\bf Inverse} $\mathbf{A}^{-1}$ (engl.: inverse matrix) von 
$\mathbf{A}$ definiert durch die Relationen
%
\be
\fbox{$\displaystyle
\mathbf{A}^{-1}\mathbf{A} = \mathbf{A}\mathbf{A}^{-1}
= \mathbf{1} \ ,
$}
\ee
%
wobei $\mathbf{1}$ die $\boldsymbol{(n \times 
n)}${\bf-Einheitsmatrix} bezeichnet [vgl.\ Gl.~(\ref{einmatr})].

\medskip
\noindent
Von Hand berechnet man die zu einer regul\"aren quadratischen
Matrix $\mathbf{A}$ Inverse $\mathbf{A}^{-1}$ \"uber
Anwendung des {\bf simultanen Gau\ss schen Algorithmus} (engl.: 
simultaneous Gau\ss ian elimination).

\medskip
\noindent
\underline{\bf GTR:} F\"ur eine eingespeicherte regul\"are
quadratische Matrix ${\tt [A]}$ kann die Inverse \"uber
${\tt [A]}^{-1}$ berechnet werden. Beachten Sie, dass
hierbei die $x^{-1}$-Funktionstaste zu verwenden ist.

\pagebreak
\medskip
\noindent
{\bf Rechenregeln f\"ur Inversbildungen}

\nopagebreak
\noindent
F\"ur $\mathbf{A},\mathbf{B} \in \mathbb{R}^{n \times n}$
mit $\det(\mathbf{A}) \neq 0 \neq \det(\mathbf{B})$
gilt:

\begin{enumerate}
\item $(\mathbf{A}^{-1})^{-1} = \mathbf{A}$
\item $(\mathbf{A}\mathbf{B})^{-1}
= \mathbf{B}^{-1}\mathbf{A}^{-1}$
\item $(\mathbf{A}^{T})^{-1} = (\mathbf{A}^{-1})^{T}$
\item $\displaystyle (\lambda\mathbf{A})^{-1}
= \frac{1}{\lambda}\,\mathbf{A}^{-1}$.
\end{enumerate}

\medskip
\noindent
F\"ur Anwendungen ist das Konzept einer inversen Matrix
aus folgendem Grunde von Interesse. Gegeben sei ein LGS
\[
\mathbf{A}\vec{x}=\vec{b}
\]
vom Format $(n \times n)$, mit {\em regul\"arer\/} quadratischer
Koeffizientenmatrix $\mathbf{A}$, also $\det(\mathbf{A})
\neq 0$. Dann existiert eine zu $\mathbf{A}$ inverse Matrix
$\mathbf{A}^{-1}$. Matrizenmultiplizieren wir beide Seiten des
LGS von {\em links\/}~(!) mit der Inversen $\mathbf{A}^{-1}$,
so ergibt sich dadurch
%
\be
\underbrace{\mathbf{A}^{-1}(\mathbf{A}\vec{x})
= (\mathbf{A}^{-1}\mathbf{A})\vec{x}
= \mathbf{1}\vec{x}
= \vec{x}}_{\text{linke Seite}}
= \underbrace{\mathbf{A}^{-1}\vec{b}}_{\text{rechte Seite}} \ .
\ee
%
Die eindeutige (!) L\"osung $\vec{x}=\mathbf{A}^{-1}\vec{b}$ des
LGS l\"asst sich in diesem Fall elegant durch eine einfache
Matrizenmultiplikation mit der zu $\mathbf{A}$ Inversen erhalten.

%%%%%%%%%%%%%%%%%%%%%%%%%%%%%%%%%%%%%%%%%%%%%%%%%%%%%%%%%%%%%%%%%%%
\section[Ausblick]{Ausblick}
\lb{sec:lgsausblick}
%%%%%%%%%%%%%%%%%%%%%%%%%%%%%%%%%%%%%%%%%%%%%%%%%%%%%%%%%%%%%%%%%%%
Die {\bf Lineare Algebra} beinhaltet viele spannende
weiterf\"uhrende Themen. Von besonderer Relevanz f\"ur praktische
Anwendungen ist das Thema der {\bf Eigenwerte} (engl.: 
eigenvalues) und den zugeordneten {\bf Eigenvektoren} (engl.: 
eigenvectors) von {\bf quadratischen Matrizen}, welche
letztere in besonderer Art und Weise charakterisieren. Die
Fragestellung ist hierbei die folgende: F\"ur eine gegebene
quadratische reellwertige Matrix $\mathbf{A}
\in \mathbb{R}^{n \times n}$,
existieren reelle Zahlenwerte $\lambda_{n} \in \mathbb{R}$ und
reellwertige Vektoren $\vec{v}_{n} \in \mathbb{R}^{n}$, die die
Bedingung
%
\be
\lb{eigenveq1}
\mathbf{A}\vec{v}_{n} \stackrel{!}{=} \lambda_{n}\vec{v}_{n}
\ee
%
erf\"ullen? Anders gewendet lautet das Problem: F\"ur welche
Vektoren $\vec{v}_{n} \in \mathbb{R}^{n}$ l\"asst sich deren
Abbildung durch eine vorgegebene Matrix $\mathbf{A}
\in \mathbb{R}^{n \times n}$ auf ein einfaches Skalieren
dieser Vektoren mit Zahlenwerten $\lambda_{n} \in \mathbb{R}$
zur\"uckf\"uhren?

\medskip
\noindent
Durch Umstellen und Ausklammern erh\"alt man aus
Gl.~(\ref{eigenveq1}) die alternative Bedingung
%
\be
\lb{eigenveq2}
\boldsymbol{0} \stackrel{!}{=}
\left(\mathbf{A}-\lambda_{n}\boldsymbol{1}\right)\vec{v}_{n} \ ,
\ee
%
mit $\boldsymbol{1}$ einer $(n \times n)$-Einheitsmatrix
[vgl.\ Gl.~(\ref{einmatr})] und $\boldsymbol{0}$
einem $n$-komponentigen Nullvektor.
Diese Bedingung entspricht einem zu l\"osenden homogenen LGS
vom Format $(n \times n)$. Nichttriviale L\"osungen dieses
homogenen LGS, also L\"osungen f\"ur den Fall $\vec{v}_{n}
\neq \boldsymbol{0}$, existieren unter der Voraussetzung,
dass die durch ein Polynom vom Grad $n$ (vgl.
Abschnitt~\ref{subsec:polynomials}) gegebene
{\bf charakteristische Gleichung} (engl.: characteristic equation)
%
\be
0 \stackrel{!}{=} 
\det\left(\mathbf{A}-\lambda_{n}\boldsymbol{1}\right)
\ee
%
reellwertige Nullstellen $\lambda_{n} \in \mathbb{R}$ besitzt.
Symmetrische quadratische Matrizen (vgl. Abschnitt 
\ref{sec:matrech}) besitzen ausschlie\ss lich reelle Eigenwerte 
$\lambda_{n}$.

\medskip
\noindent
Kenntnis des Spektrums der {\bf Eigenwerte}
$\lambda_{n} \in \mathbb{R}$ und der {\bf Eigenvektoren}
$\vec{v}_{n} \in \mathbb{R}^{n}$ einer reellwertigen Matrix
$\mathbf{A} \in \mathbb{R}^{n \times n}$ bildet die Grundlage
einer Transformation von $\mathbf{A}$ auf {\bf Diagonalgestalt}
$\mathbf{A}_{\lambda_{n}}$ (engl.: diagonal form), wobei die 
Eigenwerte $\lambda_{n}$ die Diagonalelemente von 
$\mathbf{A}_{\lambda_{n}}$ bilden. Details dieser Fragestellung 
sind z.B. in Arens {\em et al\/} (2008) \ct[S.~585ff]{areetal2008} 
zu finden.

\medskip
\noindent
Eine besondere Rolle spielt das Konzept der Eigenwerte und
Eigenvektoren von quadratischen reellwertigen Matrizen z.B.
in der {\bf Statistik}, im Rahmen explorativer
{\bf Hauptkomponentenanalysen} (engl.: exploratory principal 
component analysis) multivariater Datens\"atze
zur Identifikation von dominanten datenintrinsischen Strukturen;
vgl. z.B. Rinne (2008) \ct[S.~699ff]{rin2008}.

%%%%%%%%%%%%%%%%%%%%%%%%%%%%%%%%%%%%%%%%%%%%%%%%%%%%%%%%%%%%%%%%%%%
%%%%%%%%%%%%%%%%%%%%%%%%%%%%%%%%%%%%%%%%%%%%%%%%%%%%%%%%%%%%%%%%%%%
