%%%%%%%%%%%%%%%%%%%%%%%%%%%%%%%%%%%%%%%%%%%%%%%%%%%%%%%%%%%%%%%%%%%
%  File name: ch1.tex
%  Title:
%  Version: 11.10.2013 (hve)
%%%%%%%%%%%%%%%%%%%%%%%%%%%%%%%%%%%%%%%%%%%%%%%%%%%%%%%%%%%%%%%%%%%
%%%%%%%%%%%%%%%%%%%%%%%%%%%%%%%%%%%%%%%%%%%%%%%%%%%%%%%%%%%%%%%%%%%
\chapter[Vektorrechnung im Euklidischen Raum ${\mathbb R}^{n}$
]{Vektorrechnung im Euklidischen Raum ${\mathbb R}^{n}$}
\lb{ch1}
%%%%%%%%%%%%%%%%%%%%%%%%%%%%%%%%%%%%%%%%%%%%%%%%%%%%%%%%%%%%%%%%%%%
Wir beginnen unsere Betrachtungen mit der Einf\"uhrung einer
speziellen Klasse von mathematischen Objekten, die uns sp\"ater
behilflich sein werden, geeignete Probleme quantitativer Natur
mathematisch m\"oglichst elegant und kompakt zu formulieren.
Neben diesen mathematischen Objekten m\"ussen wir zus\"atzlich
definieren, welche Rechenoperationen wir auf diese Objekte
anwenden k\"onnen, und welche Regeln hierbei zu beachten sind.

%%%%%%%%%%%%%%%%%%%%%%%%%%%%%%%%%%%%%%%%%%%%%%%%%%%%%%%%%%%%%%%%%%%
\section[Grundbegriffe und Grundrechenarten]%
{Grundbegriffe und Grundrechenarten}
\lb{sec:vekeinf}
%%%%%%%%%%%%%%%%%%%%%%%%%%%%%%%%%%%%%%%%%%%%%%%%%%%%%%%%%%%%%%%%%%%
Gegeben sei eine Menge $V$ von mathematischen Objekten $\vec{a}$,
die wir %alle zun\"achst
als eine Ansammlung von jeweils $n$ beliebigen reellen Zahlen
$a_{1}$, \ldots, $a_{i}$, \ldots, $a_{n}$ betrachten wollen, also
%
\be
\lb{vecsp}
V =  \left\{
\vec{a} = \left(
\begin{array}{c}
a_{1} \\ \vdots \\ a_{i} \\ \vdots \\ a_{n}
\end{array}\right)
\left|\, a_{i} \in {\mathbb R},\hspace{1mm}
i=1, \dots, n\right.\right\} \ .
\ee
%
Diese $n$ reellen Zahlen k\"onnen wir formal entweder in einer
Spalte oder aber in einer Zeile anordnen. Wir definieren:

\medskip
\noindent
\underline{\bf Def.:}
Reellwertiger {\bf Spaltenvektor} (engl.: column vector) mit $n$ 
Komponenten
%
\be
\fbox{$\displaystyle
\vec{a} := \left(
\begin{array}{c}
a_{1} \\ \vdots \\ a_{i} \\ \vdots \\ a_{n}
\end{array}
\right) \ ,
\hspace{10mm}
a_{i} \in {\mathbb R},\hspace{3mm}
i=1, \dots, n \ ,
$}
\ee
%
Schreibweise: $\vec{a} \in \mathbb{R}^{n \times 1}$,

\medskip
\noindent
und

\medskip
\noindent
\underline{\bf Def.:}
Reellwertiger {\bf Zeilenvektor} (engl.: row vector) mit $n$ 
Komponenten
%
\be
\fbox{$\displaystyle
\vec{a}^{T} := \left(
a_{1}, \dots, a_{i}, \dots, a_{n}\right) \ ,
\hspace{10mm}
a_{i} \in {\mathbb R},\hspace{3mm}
i=1, \dots, n \ ,
$}
\ee
%
Schreibweise: $\vec{a}^{T} \in \mathbb{R}^{1 \times n}$.

\medskip
\noindent
Entsprechend werden die durch
%
\be
\vec{0} := \left(
\begin{array}{c}
0 \\ \vdots \\ 0 \\ \vdots \\ 0
\end{array}
\right)
\qquad\text{und}\qquad
\vec{0}^{T} := \left(
0, \dots, 0, \dots, 0\right)
\ee
%
definierten Objekte $n$-komponentige {\bf Nullvektoren} (engl.: 
zero vector) genannt.

\medskip
\noindent
Als n\"achstes definieren wir f\"ur gleichartige Elemente
aus der Menge $V$, also entweder f\"ur $n$-komponentige
Spaltenvektoren oder f\"ur $n$-komponentige Zeilenvektoren,
zwei einfache Rechenoperationen:

\medskip
\noindent
\underline{\bf Def.:}
{\bf Addition} von Vektoren (engl.: addition of vectors)
%
\be
\lb{vekadd}
\fbox{$\displaystyle
\vec{a} + \vec{b}
:= \left(
\begin{array}{c}
a_{1} \\ \vdots \\ a_{i} \\ \vdots \\ a_{n}
\end{array}
\right)
+ \left(
\begin{array}{c}
b_{1} \\ \vdots \\ b_{i} \\ \vdots \\ b_{n}
\end{array}
\right)
= \left(
\begin{array}{c}
a_{1}+b_{1} \\ \vdots \\ a_{i}+b_{i} \\ \vdots \\
a_{n}+b_{n}
\end{array}
\right) \ ,
\hspace{10mm}
a_{i}, b_{i} \in {\mathbb R} \ ,
$}
\ee
%

\medskip
\noindent
und

\medskip
\noindent
\underline{\bf Def.:}
{\bf Skalieren} von Vektoren (engl.: scaling of vectors)
%
\be
\lb{vekskal}
\fbox{$\displaystyle
\lambda\vec{a}
:= \left(
\begin{array}{c}
\lambda a_{1} \\ \vdots \\ \lambda a_{i} \\ \vdots \\
\lambda a_{n}
\end{array}
\right) \ ,
\hspace{10mm}
\lambda, a_{i} \in {\mathbb R} \ .
$}
\ee
%

\medskip
\noindent
Das Skalieren eines Vektors $\vec{a}$ mit einer von Null
verschiedenen reellen Zahl $\lambda$ hat folgende Auswirkungen
auf diesen:
\begin{itemize}
	\item $|\lambda| > 1$ --- Streckung der L\"ange von $\vec{a}$
	\item $0 < |\lambda| < 1$ --- Stauchung der L\"ange von
	$\vec{a}$
	\item $\lambda < 0$ --- Richtungsumkehr von $\vec{a}$;
\end{itemize}
den Begriff der L\"ange eines Vektors $\vec{a}$ werden wir
in K\"urze pr\"azisieren. Bei der Addition bzw.\ dem
Skalieren von $n$-komponentigen Vektoren sind folgende
Rechenregeln zu beachten:

\medskip
\noindent
{\bf Rechenregeln f\"ur Addition und Skalierung
von Vektoren}

\noindent
F\"ur Vektoren
$\vec{a}, \vec{b}, \vec{c} \in {\mathbb R}^{n}$:

\begin{enumerate}
	\item $\vec{a}+\vec{b} = \vec{b}+\vec{a}$
	\hfill ({\bf Kommutativit\"at})
	\item $\vec{a}+(\vec{b}+\vec{c}) = (\vec{a}+\vec{b})+\vec{c}$
	\hfill ({\bf Assoziativit\"at})
	\item Zu $\vec{a}$ und $\vec{b}$ gibt es genau ein $\vec{x}$,
	sodass $\vec{a}+\vec{x}=\vec{b}$ \hfill ({\bf Umkehrbarkeit})
	\item $(\lambda\mu)\vec{a}=\lambda(\mu\vec{a})$ mit $\lambda
	\in {\mathbb R}$ \hfill ({\bf Assoziativit\"at})
	\item $1\vec{a}=\vec{a}$ \hfill ({\bf Skalenfestlegung})
	\item $\lambda(\vec{a}+\vec{b}) = \lambda\vec{a}+\lambda\vec{b}$;
	
	$(\lambda+\mu)\vec{a} = \lambda\vec{a}+\mu\vec{a}$ mit $\lambda,
	\mu	\in {\mathbb R}$ \hfill ({\bf Distributivit\"at}).
\end{enumerate}

\medskip
\noindent
Wir bemerken zum Schluss:
jede entsprechend Gl.~(\ref{vecsp}) konstruierte Menge $V$ von
Objekten, f\"ur deren Elemente eine Addition sowie eine
Skalierung wie in Gln.~(\ref{vekadd}) und~(\ref{vekskal})
definiert sind, und die den angef\"uhrten
Rechenregeln gen\"ugen, wird {\bf linearer Vektorraum \"uber
dem Euklidischen Raum} ${\mathbb R}^{n}$
genannt (engl.: linear vector space over Euclidian space ${\mathbb 
R}^{n}$).\footnote{Benannt nach dem griechischen Mathematiker
\href{http://www-groups.dcs.st-and.ac.uk/~history/Biographies/Euclid.html}{Euklid von Alexandria (ca. 325~v.~Chr.--ca. 265~v.~Chr.)}.}

%%%%%%%%%%%%%%%%%%%%%%%%%%%%%%%%%%%%%%%%%%%%%%%%%%%%%%%%%%%%%%%%%%%
\section[Dimension und Basis des ${\mathbb R}^{n}$]%
{Dimension und Basis des ${\mathbb R}^{n}$}
\lb{sec:vekdim}
%%%%%%%%%%%%%%%%%%%%%%%%%%%%%%%%%%%%%%%%%%%%%%%%%%%%%%%%%%%%%%%%%%%
Gegeben seien $m$ $n$-komponentige Vektoren\footnote{Zur
Kennzeichnung $n$-komponentiger
Spaltenvektoren $\vec{a} \in \mathbb{R}^{n\times 1}$ schreibt man
oft auch verk\"urzend $\vec{a} \in \mathbb{R}^{n}$; gelegentlich
ebenfalls $\vec{a}^{T} \in \mathbb{R}^{n}$ f\"{u}r
$n$-komponentige Zeilenvektoren $\vec{a}^{T} \in
\mathbb{R}^{1\times n}$.} $\vec{a}_{1}, \ldots,
\vec{a}_{i}, \ldots, \vec{a}_{m} \in
{\mathbb R}^{n}$ sowie $m$ reelle Zahlen
$\lambda_{1}, \ldots, \lambda_{i}, \ldots, \lambda_{m}
\in {\mathbb R}$. Wird aus diesen unter Anwendung der 
oben eingef\"uhrten Rechenoperationen Addition und Skalierung von
Vektoren ein neuer $n$-komponentiger Vektor $\vec{b}$ nach der
Vorschrift
%
\be
\lb{linkomb}
\fbox{$\displaystyle
\vec{b} = \lambda_{1}\vec{a}_{1}+\ldots+\lambda_{i}\vec{a}_{i}
+\ldots+\lambda_{m}\vec{a}_{m}
=: \sum_{i=1}^{m}\lambda_{i}\vec{a}_{i} \in {\mathbb R}^{n}
$}
\ee
%
konstruiert, so nennt man $\vec{b}$ eine {\bf Linearkombination} 
(engl.: linear combination) der $m$ Vektoren $\vec{a}_{i},\,i=1, 
\ldots, m$.

\medskip
\noindent
\underline{\bf Def.:} $m$ Vektoren $\vec{a}_{1}, \ldots,
\vec{a}_{i}, \ldots, \vec{a}_{m} \in {\mathbb R}^{n}$ hei\ss en
voneinander {\bf linear unabh\"angig} (engl.: linearly 
independent), wenn aus der Bedingung
%
\be
\lb{linunabh}
\boldsymbol{0} \stackrel{!}{=} \lambda_{1}\vec{a}_{1}+\ldots
+\lambda_{i}\vec{a}_{i}+\ldots+\lambda_{m}\vec{a}_{m}
= \sum_{i=1}^{m}\lambda_{i}\vec{a}_{i} \ ,
\ee
%
also der Aufgabe, aus den $m$ Vektoren $\vec{a}_{1}, \ldots,
\vec{a}_{i}, \ldots, \vec{a}_{m} \in {\mathbb R}^{n}$ durch
Linearkombination den {\bf Nullvektor} $\boldsymbol{0}$ zu
konstruieren, zwingend als einzige L\"osung
$0=\lambda_{1}=\ldots=\lambda_{i}=\ldots=\lambda_{m}$ folgt.
Kann diese Bedingung hingegen mit $\lambda_{i} \neq 0$
erf\"ullt werden, so hei\ss en die $m$ Vektoren $\vec{a}_{1},
\ldots, \vec{a}_{i}, \ldots,\vec{a}_{m} \in {\mathbb R}^{n}$
voneinander {\bf linear abh\"angig} (engl.: linearly 
dependent).

\medskip
\noindent
Im Euklidischen Raum ${\mathbb R}^{n}$ k\"onnen maximal
$n$ (!) Vektoren
voneinander linear unabh\"angig sein. Diese Maximalzahl linear
unabh\"angiger Vektoren hei\ss t {\bf Dimension des Euklidischen
Raumes} ${\mathbb R}^{n}$ (engl.: dimension of Euclidian space 
${\mathbb R}^{n}$). Jede Menge von $n$ linear unabh\"angigen
Vektoren des Euklidischen Raumes ${\mathbb R}^{n}$ bildet
eine m\"ogliche {\bf Basis des Euklidischen Raumes} ${\mathbb 
R}^{n}$ (engl.: basis of Euclidian space ${\mathbb R}^{n}$).
Ist $\{\vec{a}_{1}, \ldots, \vec{a}_{i}, \ldots,\vec{a}_{n}\}$
eine Basis des ${\mathbb R}^{n}$, so gilt f\"ur alle weiteren
Vektoren $\vec{b} \in {\mathbb R}^{n}$
%
\be
\vec{b} = \beta_{1}\vec{a}_{1}+\ldots+\beta_{i}\vec{a}_{i}
+\ldots+\beta_{n}\vec{a}_{n}
= \sum_{i=1}^{n}\beta_{i}\vec{a}_{i} \ .
\ee
%
Die Zahlen $\beta_{i}\in{\mathbb R}$ hei\ss en die
{\bf Komponenten von} $\vec{b}$ {\bf bez\"uglich der Basis}
$\{\vec{a}_{1}, \ldots, \vec{a}_{i}, \ldots, \vec{a}_{n}\}$ 
(engl.: components of $\vec{b}$ with respect to the given basis).

\vspace{5mm}
\noindent
\underline{\bf Bem.:} Die $n$ Einheitsvektoren (engl.: unit 
vectors)
%
\be
\lb{kanbasis}
\vec{e}_{1} := \left(
\begin{array}{c}
1 \\ 0 \\ \vdots \\ 0
\end{array}
\right) \ , \hspace{5mm}
\vec{e}_{2} := \left(
\begin{array}{c}
0 \\ 1 \\ \vdots \\ 0
\end{array}
\right) \ , \hspace{5mm}
\dots \ , \hspace{5mm}
\vec{e}_{n} := \left(
\begin{array}{c}
0 \\ 0 \\ \vdots \\ 1
\end{array}
\right) \ , \hspace{5mm}
\ee
%
bilden die sogenannte {\bf kanonische Basis des Euklidischen
Raumes} ${\mathbb R}^{n}$ (engl.: canonical basis of Euclidian 
space ${\mathbb R}^{n}$). Bez\"uglich dieser sind alle Vektoren 
$\vec{b} \in \mathbb{R}^{n}$ als Linearkombinationen
%
\be
\vec{b} = \left(
\begin{array}{c}
b_{1} \\ b_{2} \\ \vdots \\ b_{n}
\end{array}
\right)
= b_{1}\vec{e}_{1}+b_{2}\vec{e}_{2}+\dots
+b_{n}\vec{e}_{n}
= \sum_{i=1}^{n}b_{i}\vec{e}_{i}
\ee
%
darstellbar.

%%%%%%%%%%%%%%%%%%%%%%%%%%%%%%%%%%%%%%%%%%%%%%%%%%%%%%%%%%%%%%%%%%%
\section[Euklidisches Skalarprodukt]%
{Euklidisches Skalarprodukt}
\lb{sec:vekskal}
%%%%%%%%%%%%%%%%%%%%%%%%%%%%%%%%%%%%%%%%%%%%%%%%%%%%%%%%%%%%%%%%%%%
Schlie\ss lich f\"uhren wir noch eine dritte Rechenoperation f\"ur
Vektoren des $\mathbb{R}^{n}$ ein.

\medskip
\noindent
\underline{\bf Def.:} F\"ur einen $n$-komponentigen Zeilenvektor
$\vec{a}^{T} \in \mathbb{R}^{1 \times n}$ und einen
$n$-komponentigen Spaltenvektor $\vec{b} \in
\mathbb{R}^{n \times 1}$ definiert das {\bf Euklidische
Skalarprodukt} (engl.: Euclidian scalar product)
%
\be
\lb{skalprod}
\fbox{$\displaystyle
\vec{a}^{T}\cdot\vec{b}
:= \left(
a_{1}, \ldots, a_{i}, \ldots a_{n}\right)
\left(
\begin{array}{c}
b_{1} \\ \vdots \\ b_{i} \\ \vdots \\ b_{n}
\end{array}
\right)
= a_{1}b_{1}+\ldots+a_{i}b_{i}\ldots+a_{n}b_{n}
=: \sum_{i=1}^{n}a_{i}b_{i}
$}
\ee
%
eine Abbildung $f: \mathbb{R}^{1 \times n}\times
\mathbb{R}^{n \times 1} \rightarrow \mathbb{R}$, d.h. eine
Abbildung dieser beiden Vektoren in die Menge der reellen
Zahlen.

\medskip
\noindent
Beachten Sie, dass im Unterschied zur Addition oder dem Skalieren
von Vektoren das Resultat einer Skalarproduktbildung {\em eine
einzige reelle Zahl\/} darstellt.

\medskip
\noindent
Zwei Vektoren $\vec{a}, \vec{b} \in {\mathbb R}^{n}$ (mit $\vec{a}
\neq \vec{0} \neq \vec{b}$) werden zueinander {\bf orthogonal}
genannt, wenn sie die Eigenschaft
$0 = \vec{a}^{T}\cdot\vec{b} = \vec{b}^{T}\cdot\vec{a}$
aufweisen.

%\pagebreak
\medskip
\noindent
{\bf Rechenregeln f\"ur das Skalarprodukt von Vektoren}

\noindent
F\"ur Vektoren
$\vec{a}, \vec{b}, \vec{c} \in {\mathbb R}^{n}$:

\begin{enumerate}
	\item $(\vec{a}+\vec{b})^{T}\cdot\vec{c}
	= \vec{a}^{T}\cdot\vec{c}+\vec{b}^{T}\cdot\vec{c}$
	\hfill ({\bf Distributivit\"at})
	\item $\vec{a}^{T}\cdot\vec{b} = \vec{b}^{T}\cdot\vec{a}$
	\hfill ({\bf Kommutativit\"at})
	\item $(\lambda\vec{a}^{T})\cdot\vec{b} =
	\lambda(\vec{a}^{T}\cdot\vec{b})$ mit $\lambda \in {\mathbb R}$
	\hfill ({\bf Homogenit\"at})
	\item $\vec{a}^{T}\cdot\vec{a} > 0$ f\"ur alle $\vec{a}
	\neq \vec{0}$ \hfill ({\bf positive Definitheit}).
\end{enumerate}

\medskip
\noindent
Nun wenden wir uns dem Begriff der L\"ange eines $n$-komponentigen
Vektors zu.

\medskip
\noindent
\underline{\bf Def.:} Die {\bf L\"ange} (engl.: length) eines 
Vektors $\vec{a} \in \mathbb{R}^{n}$ ist definiert als
%
\be
\fbox{$\displaystyle
|\vec{a}| :=
\sqrt{\vec{a}^{T}\cdot\vec{a}}
= \sqrt{a_{1}^{2}+\ldots+a_{i}^{2}+\ldots+a_{n}^{2}}
=: \sqrt{\sum_{i=1}^{n}a_{i}^{2}} \ .
$}
\ee
%

\medskip
\noindent
Die nichtnegative reelle Zahl $|\vec{a}|$ wird technisch auch
Betrag von $\vec{a}$ bzw.\ Euklidische Norm
von $\vec{a}$ genannt. Die L\"ange von $\vec{a} \in
\mathbb{R}^{n}$ gen\"ugt folgenden Eigenschaften: 
%
\begin{itemize}
	\item $|\vec{a}| \geq 0$; $|\vec{a}|=0 \Leftrightarrow
	\vec{a}=\vec{0}$;
	\item $|\lambda\vec{a}| = |\lambda||\vec{a}|$ f\"ur $\lambda \in
	{\mathbb R}$;
	\item $|\vec{a}+\vec{b}| \leq |\vec{a}|+|\vec{b}|$
	\hfill ({\bf Dreiecksungleichung}).
\end{itemize}

\medskip
\noindent
Jeder Vektor $\vec{a} \in \mathbb{R}^{n}$ mit $|\vec{a}|>0$
kann durch seine L\"ange dividiert (mit dem Kehrwert seiner
L\"ange skaliert) werden.

\medskip
\noindent
\underline{\bf Def.:}
{\bf Normieren} (engl.: normalisation) eines Vektors $\vec{a} \in
\mathbb{R}^{n}$;
%
\be
\hat{\vec{a}} := \frac{\vec{a}}{|\vec{a}|}
\qquad\Rightarrow\qquad
|\hat{\vec{a}}| = 1 \ .
\ee
%

\medskip
\noindent
Dadurch erh\"alt man einen Vektor der L\"ange $1$, einen
{\bf Einheitsvektor} $\hat{\vec{a}}$. Zur Kennzeichnung dessen
verwenden wir das Symbol "`Dach"'.

\medskip
\noindent
Abschlie\ss end f\"uhren wir \"uber das Skalarprodukt
den zwischen zwei Vektoren von L\"ange ungleich Null
eingeschlossenen Winkel ein.

\medskip
\noindent
\underline{\bf Def.:}
{\bf Winkel} (engl.: angle) zwischen $\vec{a},\vec{b} \neq \vec{0}
\in {\mathbb R}^{n}$ 
%
\be
\fbox{$\displaystyle
\cos[\varphi(\vec{a},\vec{b})]
= \frac{\vec{a}^{T}}{|\vec{a}|}\cdot
\frac{\vec{b}}{|\vec{b}|}
= \hat{\vec{a}}{}^{T}\cdot\hat{\vec{b}}
\hspace{5mm} \Rightarrow \hspace{5mm}
\varphi(\vec{a},\vec{b})
= \cos^{-1}(\hat{\vec{a}}{}^{T}\cdot\hat{\vec{b}}) \ .
$}
\ee
%
\underline{\bf Bem.:} Die inverse Kosinusfunktion\footnote{Der
Begriff einer inversen Funktion wird in Kapitel~\ref{ch7}
gekl\"art werden.} $\cos^{-1}(\ldots)$ wird durch jeden GTR zur
Verf\"ugung gestellt.

%%%%%%%%%%%%%%%%%%%%%%%%%%%%%%%%%%%%%%%%%%%%%%%%%%%%%%%%%%%%%%%%%%%
%%%%%%%%%%%%%%%%%%%%%%%%%%%%%%%%%%%%%%%%%%%%%%%%%%%%%%%%%%%%%%%%%%%
