%%%%%%%%%%%%%%%%%%%%%%%%%%%%%%%%%%%%%%%%%%%%%%%%%%%%%%%%%%%%%%%%%%%
%  File name: ch6.tex
%  Title:
%  Version: 06.12.2016 (hve)
%%%%%%%%%%%%%%%%%%%%%%%%%%%%%%%%%%%%%%%%%%%%%%%%%%%%%%%%%%%%%%%%%%%
%%%%%%%%%%%%%%%%%%%%%%%%%%%%%%%%%%%%%%%%%%%%%%%%%%%%%%%%%%%%%%%%%%%
\chapter[Elementare Finanzmathematik]%
{Elementare Finanzmathematik}
\lb{ch6}
%%%%%%%%%%%%%%%%%%%%%%%%%%%%%%%%%%%%%%%%%%%%%%%%%%%%%%%%%%%%%%%%%%%
\hfill\hbox{\fbox{\vbox{\hsize=10cm
Der Inhalt dieses Kapitels diskutiert grundlegende
finanzmathematische Themen,
welche im Rahmen des Drittsemestermoduls 0.3.2 RESO: Resources:
Financial Resources, Human Resources, Organisation wiederkehren
werden.
}}}

\vspace{10mm}
\noindent
In diesem Kapitel wenden wir uns kurz den quantitativen
Grundlagen des auf jegliches unternehmerisches %wirtschaftliches
Handeln Einfluss nehmenden Finanzwesens zu.
Wie wir im Folgenden aufzeigen wollen, beruhen viele grundlegende,
f\"ur die praktische Anwendung hochrelevante \"Uberlegungen
der {\bf Finanzmathematik} auf zwei einfach zug\"anglichen
mathematischen Strukturen, den sogenannten arithmetischen und
geometrischen reellen Zahlenfolgen und ihren zugeordneten
endlichen Reihen.

%%%%%%%%%%%%%%%%%%%%%%%%%%%%%%%%%%%%%%%%%%%%%%%%%%%%%%%%%%%%%%%%%%%
\section[Arithmetische und geometrische Folgen und Reihen]%
{Arithmetische und geometrische Zahlenfolgen und Reihen}
\lb{sec:folgreih}
%%%%%%%%%%%%%%%%%%%%%%%%%%%%%%%%%%%%%%%%%%%%%%%%%%%%%%%%%%%%%%%%%%%
%------------------------------------------------------------------
\subsection{Arithmetische Zahlenfolge und Reihe}
\lb{subsec:arithseq}
%------------------------------------------------------------------
Eine {\bf arithmetische reelle Zahlenfolge} (engl.: arithmetical 
real-valued sequence) mit $n \in \mathbb{N}$ Gliedern,
%
\[
(a_{n})_{n \in \mathbb{N}} \ ,
\]
%
ist definiert durch die Eigenschaft einer {\em konstant\/}
bleibenden {\bf Differenz} (engl.: difference) $d$ zwischen 
benachbarten Gliedern der Folge, d.h. f\"ur $n>1$ gilt
%
\be
\lb{arifolrek}
\fbox{$\displaystyle
a_{n}-a_{n-1}=:d=\text{konstant}\neq 0 \ ,
$}
\ee
%
mit $a_{n}, a_{n-1}, d \in \mathbb{R}$. Daraus l\"asst sich das
einfache explizite {\bf Bildungsgesetz} (engl.: explicit 
representation)
%
\be
\lb{arifolex}
a_{n} = a_{1} + (n-1)d
\quad\text{mit}\quad
n \in \mathbb{N}
\ee
%
ableiten. Man sieht also, dass bei Kenntnis der Werte der zwei
freien Parameter: Startglied $a_{1}$ und konstante Differenz $d$,
die Glieder {\em jeder\/} arithmetischen reellen Zahlenfolge
{\em eindeutig\/} bestimmt sind. Wie Gl.~(\ref{arifolex}) zeigt,
wachsen/fallen die Glieder $a_{n}$ {\bf linear} (engl.: linear 
growth) mit $n$.

\medskip
\noindent
Betrachtet man f\"ur eine arithmetische reelle Zahlenfolge
mit $n+1$ Gliedern f\"ur die Nachbarglieder von $a_{n}$
(mit $n \geq 2$) den {\bf arithmetischen Mittelwert} (engl.: 
arithmetical mean), so findet man das Resultat
%
\be
\frac{1}{2}\,(a_{n-1}+a_{n+1})
= \frac{1}{2}\left(a_{1}+(n-2)d+a_{1}+nd\right)
= a_{1}+(n-1)d
= a_{n} \ .
\ee
%

\medskip
\noindent
Das Aufsummieren der ersten $n$ Glieder einer beliebigen
arithmetischen reellen Zahlenfolge f\"uhrt zur {\bf endlichen
arithmetischen Reihe} (engl.: finite arithmetical series)
%
\be
S_{n} := a_{1}+a_{2}+\ldots+a_{n} = \sum_{k=1}^{n}a_{k}
= \sum_{k=1}^{n}\left[a_{1} + (k-1)d\right]
= na_{1}+ \frac{d}{2}\,(n-1)n \ .
\ee
%
Im letzten Schritt wurde Gebrauch gemacht von der Gau\ss schen
{\bf Identit\"at}\footnote{Analog gilt die Gau\ss sche Identit\"at
$\displaystyle \sum_{k=1}^{n}(2k-1) \equiv n^{2}$.} (engl.: 
identity) (siehe z.B. Bosch (2003)~\ct[S.21]{bos2003})
%
\be
\lb{id1}
\fbox{$\displaystyle
\sum_{k=1}^{n-1}k \equiv \frac{1}{2}\,(n-1)n \ .
$}
\ee
%

%------------------------------------------------------------------
\subsection{Geometrische Zahlenfolge und Reihe}
\lb{subsec:geomseq}
%------------------------------------------------------------------
Eine {\bf geometrische reelle Zahlenfolge} (engl.: geometrical 
real-valued sequence) mit $n \in \mathbb{N}$ Gliedern,
%
\[
(a_{n})_{n \in \mathbb{N}} \ ,
\]
%
ist definiert durch die Eigenschaft eines {\em konstant\/}
bleibenden {\bf Quotienten} (engl.: quotient) $q$ zwischen 
benachbarten Gliedern der Folge, d.h. f\"ur $n>1$ gilt
%
\be
\lb{geofolrek}
\fbox{$\displaystyle
\frac{a_{n}}{a_{n-1}}=:q=\text{konstant}\neq 0 \ ,
$}
\ee
%
mit $a_{n}, a_{n-1} \in \mathbb{R}$
und $q \in \mathbb{R}\backslash\{0,1\}$.
Daraus l\"asst sich das einfache explizite {\bf Bildungsgesetz} 
(engl.: explicit representation)
%
\be
\lb{geofolex}
a_{n} = a_{1}q^{n-1}
\quad\text{mit}\quad
n \in \mathbb{N}
\ee
%
ableiten. Man sieht also, dass bei Kenntnis der Werte der zwei
freien Parameter: Startglied $a_{1}$ und konstanter Quotient $q$,
die Glieder {\em jeder\/} geometrischen reellen Zahlenfolge
{\em eindeutig\/} bestimmt sind. Wie Gl.~(\ref{geofolex}) zeigt,
wachsen/fallen die Glieder $a_{n}$ {\bf exponentiell} (engl.: 
exponential growth) mit $n$ (vgl. 
Abschnitt~\ref{subsec:exponentials}).

\medskip
\noindent
Betrachtet man f\"ur eine geometrische reelle Zahlenfolge
mit $n+1$ Gliedern f\"ur die Nachbarglieder von $a_{n}$
(mit $n \geq 2$) den {\bf geometrischen Mittelwert} (engl.: 
geometrical mean), so findet man das Resultat
%
\be
\sqrt{a_{n-1}\cdot a_{n+1}}
= \sqrt{a_{1}q^{n-2}\cdot a_{1}q^{n}}
= a_{1}q^{n-1}
= a_{n} \ .
\ee
%

\medskip
\noindent
Das Aufsummieren der ersten $n$ Glieder einer beliebigen
geometrischen reellen Zahlenfolge f\"uhrt zur {\bf endlichen
geometrischen Reihe} (engl.: finite geometrical series)
%
\be
S_{n} := a_{1}+a_{2}+\ldots+a_{n} = \sum_{k=1}^{n}a_{k}
= \sum_{k=1}^{n}\left[a_{1}q^{k-1}\right]
= a_{1}\sum_{k=0}^{n-1}q^{k}
= a_{1}\,\frac{q^{n}-1}{q-1} \ .
\ee
%
Im letzten Schritt wurde Gebrauch gemacht von der
{\bf Identit\"at} (engl.: identity) (siehe z.B. Bosch 
(2003)~\ct[S.27]{bos2003})
%
\be
\lb{id2}
\fbox{$\displaystyle
\sum_{k=0}^{n-1}q^{k} \equiv \frac{q^{n}-1}{q-1}
\quad\text{f\"ur}\quad
q \in \mathbb{R}\backslash\{0,1\} \ .
$}
\ee
%

%%%%%%%%%%%%%%%%%%%%%%%%%%%%%%%%%%%%%%%%%%%%%%%%%%%%%%%%%%%%%%%%%%%
\section[Zins und Zinseszins]%
{Zins und Zinseszins}
\lb{sec:zins}
%%%%%%%%%%%%%%%%%%%%%%%%%%%%%%%%%%%%%%%%%%%%%%%%%%%%%%%%%%%%%%%%%%%
Gegeben seien ein {\bf Anfangskapital} (engl.: initial capital) 
des positiven Werts $K_{0}$ (in GE), und ein Betrachtungszeitraum 
von $n \in \mathbb{N}$ gleich langen {\bf Perioden} (engl.: 
periods). Das Anfangskapital soll jeweils zum Ende jeder Periode 
(d.h. "`nachsch\"ussig"') zu einem in Prozent
anzugebenden {\bf Zinsfu\ss} (engl.: interest rate) $p>0$ mit 
Zinseszins verzinst werden. \"Uber den durch\footnote{Durch 
Umkehrung ergibt sich hieraus die Relation $p=100\cdot(q-1)$.}
%
\be
q:=1+\frac{p}{100}>1
\ee
%
definierten dimensionslosen {\bf Aufzinsfaktor} (engl.: interest 
factor) folgt, dass sich nach Ablauf von einer Zinsperiode ein 
Kapital vom Wert
%
\[
K_{1} = K_{0} + K_{0}\cdot\frac{p}{100}
= K_{0}\left(1+\frac{p}{100}\right) = K_{0}q
\]
%
angesammelt haben wird. Nach Ablauf von insgesamt
$n$ Zinsperioden wird also ein {\bf Endkapital} (engl.: final 
capital) vom Wert (in GE)
%
\be
\lb{kap1}
\fbox{$\displaystyle
\text{rekursiv:}\ K_{n}=K_{n-1}q \ , \quad
n\in\mathbb{N} \ ,
$}
\ee
%
vorliegen, wobei $K_{n-1}$ den Wert des
Kapitals nach $n-1$ Verzinsungen bezeichnet. Aus dieser
rekursiven Darstellung des durch $n$-malige Verzinsung
mit Zinseszins bedingten Wachstums eines Anfangskapitals
$K_{0}$ geht sofort der Bezug der hier aufgezeigten
zeitlichen Entwicklung zur geometrischen Zahlenfolge
(\ref{geofolrek}) hervor.

\medskip
\noindent
Wie sich leicht zeigen l\"asst, h\"angen Endkapital
$K_{n}$ und Anfangskapital $K_{0}$ \"uber
%
\be
\lb{kap2}
\fbox{$\displaystyle
\text{explizit:}\ K_{n}=K_{0}q^{n} \ , \quad
n\in\mathbb{N} \ ,
$}
\ee
%
zusammen. Durch diese Gleichung werden die vier Gr\"o\ss en
$K_{n}$, $K_{0}$, $q$ und $n$ zueinander in Relation gesetzt.
Bei Kenntnis von jeweils drei dieser Gr\"o\ss en kann somit
der Wert der vierten ausgerechnet werden. Beispielsweise
findet man durch Aufl\"osen von Gl.~(\ref{kap2}) nach $K_{0}$
das Ergebnis
%
\be
K_{0} = \frac{K_{n}}{q^{n}} =: B_{0} \ .
\ee
%
In dieser Variante wird von $K_{0}$ als dem {\bf Barwert} (engl.: 
present value) $B_{0}$ des Kapitals $K_{n}$ gesprochen; dieser 
ergibt sich durch $n$-maliges Abzinsen von $K_{n}$ mit $q$. Weitere
M\"oglichkeiten sind:
%
\begin{itemize}
\item[(i)] Aufl\"osen von Gl.~(\ref{kap2}) nach dem
{\bf Aufzinsfaktor} (engl.: interest factor) $q$:
%
\be
q = \sqrt[n]{\frac{K_{n}}{K_{0}}} \ ,
\ee
%
\item[(ii)] Aufl\"osen von Gl.~(\ref{kap2}) nach
der {\bf Laufzeit} (engl.: contract period) $n$:
%
\be
n = \frac{\ln\left(K_{n}/K_{0}\right)}{\ln(q)} \ .
\ee
%
\end{itemize}
%
Im Folgenden sei die Dauer einer {\bf Zinsperiode} (engl.: 
interest rate period) ein Kalenderjahr.

\medskip
\noindent
Wird ein {\bf Anfangskapital} (engl.: initial capital) $K_{0}>0$ 
\"uber ein Kalenderjahr hinweg $m$-malig anteilsm\"a\ss ig 
nachsch\"ussig zu einem {\bf nominalen Jahreszinsfu\ss} (engl.: 
annual nominal interest rate) $p_{\rm nom}>0$ verzinst, so ist 
$K_{0}$ am Ende des ersten von $m$ Teiljahren auf den Betrag
%
\[
K_{1/m} = K_{0} + K_{0}\cdot\frac{p_{\rm nom}}{m\cdot 100}
= K_{0}\left(1+\frac{p_{\rm nom}}{m\cdot 100}\right) \ ,
\quad
m \in \mathbb{N} \ ,
\]
%
angewachsen. Nach Ablauf von $k$ von $m$ Teiljahren gilt
%
\[
K_{k/m} = K_{0}\left(1+\frac{p_{\rm nom}}{m\cdot 100}\right)^{k} \ ,
\quad
k \leq m \ ;
\]
%
der anteilsm\"a\ss ige Aufzinsfaktor $\displaystyle
\left(1+\frac{p_{\rm nom}}{m\cdot 100}\right)$
ist also $k$-mal auf $K_{0}$ anzuwenden.
Am Ende eines vollen Kalenderjahrs ist $K_{0}$ auf den Wert
%
\[
K_{1} = K_{m/m} = K_{0}\left(1+\frac{p_{\rm nom}}{m\cdot 100}
\right)^{m} \ , \quad
m \in \mathbb{N} \ ,
\]
%
angewachsen. Diese Relation definiert den {\bf effektiven
Aufzinsfaktor} (engl.: effective interest factor)
%
\be
\lb{qeff}
q_{\rm eff} := \left(1+\frac{p_{\rm nom}}{m\cdot 100}
\right)^{m} \ ,
\ee
%
welchem, wegen $\displaystyle
q_{\rm eff}=1+\frac{p_{\rm eff}}{100}$, \"uber Umstellen
der {\bf effektive Jahreszinsfu\ss} (engl.: effective annual 
interest rate)
%
\be
\lb{peff}
\fbox{$\displaystyle
p_{\rm eff} = 100\cdot\left[\left(1+\frac{p_{\rm nom}}{m\cdot 100}
\right)^{m}-1\right] \ , \quad m\in\mathbb{N} \ ,
$
}
\ee
%
zugeordnet ist.

\medskip
\noindent
Nach Ablauf von $n$ Kalenderjahren ist das
Anfangskapital $K_{0}$ bei der hier beschriebenen
anteilsm\"a\ss igen Verzinsungsvariante schlie\ss lich
auf den Endwert
%
\be
\lb{kap3}
\fbox{$\displaystyle
K_{n}=K_{0}\left(1+\frac{p_{\rm nom}}{m\cdot 100}\right)^{n\cdot m}
= K_{0}q_{\rm eff}^{n} \ , \quad n,m\in\mathbb{N} \ ,
$}
\ee
%
angewachsen. Als Barwert von $K_{n}$ haben wir also in diesem Fall
%
\be
B_{0} = \frac{K_{n}}{q_{\rm eff}^{n}} = K_{0} \ .
\ee
%

\medskip
\noindent
Zum Ende dieses Abschnitts wollen wir uns kurz mit dem
{\bf vorsch\"ussigen Ratensparen} (engl.: installment savings) 
befassen. Der Einfachheit halber beschr\"anken wir uns auf den 
Fall von $n$ j\"ahrlichen vorsch\"ussigen Einzahlungen vom
{\em konstanten\/} Betrag $E>0$ (in GE) auf ein Konto mit
Startwert $K_{0}=0$, welches jeweils zum Ende eines
Kalenderjahres mit einem Zinsfu\ss\ $p>0$ (also $q>1$)
verzinst wird. Nach Ablauf des ersten Kalenderjahres lautet
dann der neue Kontostand
%
\[
K_{1} = E + E\cdot\frac{p}{100}
= E\left(1+\frac{p}{100}\right)
= Eq \ .
\]
%
Der Kontostand nach Ablauf von zwei Kalenderjahren ist auf
den Wert
%
\[
K_{2} = (K_{1}+E)q = (Eq+E)q = E(q^{2}+q) = Eq(q+1)
\]
%
angewachsen. Nach Ablauf von insgesamt $n$ Kalenderjahren
gilt
%
\[
K_{n} = (K_{n-1}+E)q = E(q^{n}+\ldots+q^{2}+q)
= Eq(q^{n-1}+\ldots+q+1)
= Eq\sum_{k=0}^{n-1}q^{k} \ .
\]
%
Unter Anwendung der Identit\"at~(\ref{id2}), da $q>1$,
l\"asst sich dieses Ergebnis f\"ur den {\bf Kontostand}
(engl.: account balance) nach $n$ Kalenderjahren $K_{n}$ auf die 
Form
%
\be
\lb{kap4}
\fbox{$\displaystyle
K_{n}=Eq\,\frac{q^{n}-1}{q-1} \ , \quad
q \in \mathbb{R}_{>1} \ , \quad
n\in\mathbb{N} \ ,
$}
\ee
%
reduzieren. Der $K_{n}$ zugeordnete {\bf Barwert} (engl.: present 
value) $B_{0}$ wird definiert \"uber $n$-maliges Abzinsen von 
$K_{n}$ mit $q$:
%
\be
B_{0} := \frac{K_{n}}{q^{n}}
\overbrace{=}^{\text{Gl.}~\ref{kap4}}
 = \frac{E(q^{n}-1)}{q^{n-1}(q-1)} \ .
\ee
%
Er gibt an welches Anfangskapital $B_{0}$ nach $n$-maligem
Aufzinsen mit $q$ den Wert $K_{n}$ erreicht.

\medskip
\noindent
Aufl\"osen von Gl.~(\ref{kap4}) nach der {\bf Laufzeit} (engl.: 
contract period) $n$ f\"uhrt zum Ergebnis:
%
\be
n = \frac{\ln\left[1+(q-1)(K_{n}/Eq)\right]}{\ln(q)} \ .
\ee
%

%%%%%%%%%%%%%%%%%%%%%%%%%%%%%%%%%%%%%%%%%%%%%%%%%%%%%%%%%%%%%%%%%%%
\section[Tilgung in konstanten Annuit\"aten]%
{Tilgung in konstanten Annuit\"aten}
\lb{sec:tilg}
%%%%%%%%%%%%%%%%%%%%%%%%%%%%%%%%%%%%%%%%%%%%%%%%%%%%%%%%%%%%%%%%%%%
Ausgangspunkt der folgenden Betrachtung ist ein {\bf Darlehen} 
(engl.: mortgage, loan of money) vom Betrag $R_{0}>0$ (in GE), 
welches ein Kunde zu Honorarzahlungen mit einem {\bf 
Jahreszinsfu\ss} $p>0$ (d.h. $q>1$)
von einer Bank ausleiht. Die vertraglichen Vereinbarungen zwischen
beiden Parteien sehen vor, dass der Kunde dieses Darlehen an die
Bank mit einem {\bf anf\"anglichen Tilgungsprozentsatz} (engl.: 
initial percentage rate of redemption payments) $t>0$ in
{\em konstant\/} bleibenden nachsch\"ussigen {\bf Annuit\"aten}
(engl.: annuities, annual installments) (d.h. j\"ahrlichen 
Zahlungen) vom Betrag $A>0$ (in GE)
zur\"uckzahlt. Die Annuit\"at $A$ setzt sich zusammen aus der
variablen {\bf $n$-ten Zinszahlung} (engl.: interest payment) 
$Z_{n}$ (in GE) und der
variablen {\bf $n$-ten Tilgungszahlung} (engl.: redemption 
payment) $T_{n}$ (in GE), die in
ihrer {\em Summe\/} \"uber vollst\"andige Kalenderjahre hinweg
konstant bleiben m\"ussen, also
%
\be
\lb{annuity1}
A = Z_{n} + T_{n}
\stackrel{!}{=} \text{konstant} \ .
\ee
%
Mit $n=1$ bspw., l\"asst sich dieser Ausdruck darstellen
als
%
\be
\lb{annuity2}
A = Z_{1} + T_{1}
= R_{0}\cdot\frac{p}{100} + R_{0}\cdot\frac{t}{100}
= R_{0}\left(\frac{p+t}{100}\right)
= R_{0}\left[(q-1)+\frac{t}{100}\right]
\stackrel{!}{=} \text{konstant} \ .
\ee
%
F\"ur das erste Kalenderjahr des laufenden Darlehensvertrags
haben also die Zinsschuld, der Tilgungsbetrag und die nach
Zahlung der ersten Annuit\"at bestehende Restschuld die Werte
%
\begin{eqnarray*}
Z_{1} & = & R_{0}\cdot\frac{p}{100} \ = \ R_{0}(q-1) \\
%
T_{1} & = & A - Z_{1} \\
%
R_{1} & = & R_{0} + Z_{1} - A
\ \overbrace{=}^{\text{Einsetzten f\"ur}\ Z_{1}}
\ R_{0} + R_{0}\cdot\frac{p}{100} - A
\ = \ R_{0}q - A \ .
\end{eqnarray*}
%
Am Ende eines zweiten Kalenderjahres gilt
%
\begin{eqnarray*}
Z_{2} & = & R_{1}(q-1) \\
%
T_{2} & = & A - Z_{2} \\
%
R_{2} & = & R_{1} + Z_{2} - A
\ \overbrace{=}^{\text{Einsetzten f\"ur}\ Z_{2}}
\ R_{1}q - A
\ \overbrace{=}^{\text{Einsetzten f\"ur}\ R_{1}}
\ R_{0}q^{2} - A(q+1) \ .
\end{eqnarray*}
%
Die sich hier andeutenden Schemata der Bildung
mathematischer Terme setzten sich mit zunehmender
Tilgungsdauer fort. Die {\bf Zinsschuld} (engl.: remaining 
interest debt) f\"ur Kalenderjahr $n$ des Darlehensvertrags hat den
Wert (rekursiv)
%
\be
\lb{zinsn}
Z_{n}=R_{n-1}(q-1) \ , \quad n\in\mathbb{N} \ ,
\ee
%
mit $R_{n-1}$ der Restschuld am Ende des vorangegangenen
Kalenderjahrs. Der {\bf Tilgungsbetrag} (engl.: redemption 
payment) im Kalenderjahr $n$ ist gegeben durch (rekursiv)
%
\be
\lb{tilgn}
T_{n} = A - Z_{n} \ , \quad n\in\mathbb{N} \ .
\ee
%
Die {\bf Restschuld} (engl.: remaining debt) am Ende von 
Kalenderjahr $n$ (in GE) lautet
%
\be
\lb{remdebtrek}
\fbox{$\displaystyle
\text{rekursiv:} \quad
R_{n} = R_{n-1}+Z_{n}-A
\ = \ R_{n-1}q-A \ , \quad n\in\mathbb{N} \ .
$}
\ee
%
Durch sukzessive r\"uckwertige Substitution f\"ur $R_{n-1}$, 
l\"asst sich dieser Ausdruck in die Gestalt
%
\[
R_{n} = R_{0}q^{n}-A(q^{n-1}+\ldots+q+1)
= R_{0}q^{n}-A\sum_{k=0}^{n-1}q^{k} 
\]
%
\"uberf\"uhren. Unter Anwendung der Identit\"at~(\ref{id2})
findet man schlie\ss lich (da $q>1$)
%
\be
\lb{remdebtex}
\fbox{$\displaystyle
\text{explizit:} \quad
R_{n} = R_{0}q^{n}-A\,\frac{q^{n}-1}{q-1} \ , \quad
n\in\mathbb{N} \ .
$}
\ee
%
Die hier vorgestellten Ergebnisse bilden die Basis
eines {\bf Tilgungsplans} (engl.: redemption payment plan) 
mit tabellarisch geordneten Angaben zu den Gr\"o\ss en $\{n, Z_{n}, 
T_{n}, R_{n}\}$, d.h. einem Schema
%
\begin{center}
		\begin{tabular}{c||c|c|c}
		$n$ & $Z_{n}$ [GE] & $T_{n}$ [GE] & $R_{n}$ [GE] \\
		\hline\hline
		$0$ & -- & -- & $R_{0}$ \\
		$1$ & $Z_{1}$ & $T_{1}$ & $R_{1}$ \\
		$2$ & $Z_{2}$ & $T_{2}$ & $R_{2}$ \\
		\vdots & \vdots & \vdots & \vdots
		\end{tabular} \ ,
\end{center}
%
wie es Banken ihren Kreditkunden zur finanziellen
Orientierung zur Verf\"ugung stellen (m\"ussen).

\medskip
\noindent
\underline{\bf Bem.:} Zur Erstellung eines Tilgungsplans lassen
sich die rekursiven Formeln (\ref{zinsn}), (\ref{tilgn}) und
(\ref{remdebtrek}), bei vorgegebenem Darlehensbetrag $R_{0}$,
Aufzinsfaktor $q$ und Annuit\"at $A$, sehr einfach in einem
Tabellenkalkulationsprogramm wie {\tt EXCEL} umsetzen.

\medskip
\noindent
Bez\"uglich Gl.~(\ref{remdebtex}) l\"asst sich folgende
Beobachtung machen: Da die konstant bleibende Annuit\"at $A$
implizit einen Faktor $(q-1)$ enth\"alt [vgl. Gl.~(\ref{annuity2})],
wachsen die zwei miteinander konkurierenden Terme  jeweils
exponentiell mit $n$. F\"ur eine erfolgreiche Tilgung der
Restschuld $R_{n}$ ist es also entscheidend, den freien
Tilgungsparameter $t$ (bei gegebenem $p>0 \leftrightarrow
q>1$) derart festzulegen, dass der zweite Term die M\"oglichkeit
erh\"alt, den ersten (der mit einem Vorsprung $R_{0}>0$
bei $n=0$ startet) mit $n$ einholen zu k\"onnen. Die notwendige
Voraussetzung $t>0$, welche aus der Forderung $R_{n}
\stackrel{!}{\leq} R_{n-1}$ folgt, gew\"ahrleistet gerade
diese M\"oglichkeit.

\medskip
\noindent
Gleichung~(\ref{remdebtex}) stellt eine Beziehung zwischen
f\"unf Gr\"o\ss en dar. Hat man Werte zu vieren vorgegeben,
l\"asst sich daraus der resultiernde Wert der f\"unften
Gr\"o\ss e ausrechnen.
%
\begin{itemize}
\item[(i)] Berechnen der {\bf Laufzeit} (engl.: contract period) 
$n$ eines Darlehensvertrags, bei vorgegebenem Darlehensbetrag 
$R_{0}$, Aufzinsfaktor $q$ und Annuit\"at $A$. Aufl\"osen der
Bedingungsgleichung $R_{n}\stackrel{!}{=}0$ nach $n$
liefert (nach wenigen algebraischen Schritten)
%
\be
n = \frac{\ln\left(1+\frac{p}{t}\right)}{\ln(q)} \ ;
\ee
%
die Laufzeit ist also unabh\"angig vom Betrag $R_{0}$ des Darlehens.

\item[(ii)] Berechnen der {\bf Annuit\"at} (engl.: annuity) $A$,
bei vorgegebener Laufzeit $n$, Darlehensbetrag $R_{0}$
und Aufzinsfaktor $q$. Aufl\"osen der Bedingungsgleichung
$R_{n}\stackrel{!}{=}0$ nach $A$ liefert unmittelbar
%
\be
\lb{annuity3}
A = \frac{q^{n}(q-1)}{q^{n}-1}\,R_{0} \ .
\ee
%
Durch Gleichsetzen der Ausdr\"ucke (\ref{annuity3}) und
(\ref{annuity2}) f\"ur $A$ findet man zus\"atzlich
%
\be
\frac{t}{100} = \frac{q-1}{q^{n}-1} \ .
\ee
%
\end{itemize}
%

%%%%%%%%%%%%%%%%%%%%%%%%%%%%%%%%%%%%%%%%%%%%%%%%%%%%%%%%%%%%%%%%%%%
\section[Rentenrechnung]{Rentenrechnung}
\lb{sec:rente}
%%%%%%%%%%%%%%%%%%%%%%%%%%%%%%%%%%%%%%%%%%%%%%%%%%%%%%%%%%%%%%%%%%%
Bei der {\bf Rentenrechnung} (engl.: pension calculations) 
betrachtet man, ausgehend von einem {\bf Anfangskapital} $K_{0}>0$ 
(in GE), die (diskrete) zeitliche
Entwicklung eines {\bf Kontostands} $K_{n}$ (in GE) unter dem
Einfluss von zwei gegenl\"aufigen Wirkungen: einerseits verdient
das Rentenkonto Zinsen zu einem {\bf Jahreszinsfu\ss} $p>0$
(d.h. $q>1$), andererseits werden von diesem Konto in gleichen
Abst\"anden $m \in \mathbb{N}$ vorsch\"ussige Rentenauszahlungen
vom {\em konstanten\/} {\bf Betrag} (engl.: amount) $a$ pro 
Kalenderjahr get\"atigt.

\medskip
\noindent
Wir beginnen mit der Berechnung der pro Kalenderjahr anfallenden
Zinsen. Hierbei ber\"ucksichtigt die Bank die Tatsache, dass
sich das auf dem Konto befindende Kapital w\"ahrend eines
Kalenderjahres stetig vermindert. Zinsen werden deshalb
pro verstrichenem von $m$ Teiljahren anteilsm\"a\ss ig, aber ohne
Zinseszinseffekt, berechnet. Folglich hat das Konto zum Ende
des ersten von $m$ Teiljahren anteilsm\"a\ss ig Zinsen vom Betrag
(in GE)
%
\[
Z_{1/m} = (K_{0}-a)\cdot\frac{p}{m\cdot 100}
= (K_{0}-a)\,\frac{(q-1)}{m}
\]
%
verdient. Der Zinsbetrag f\"ur das $k$-te von $m$ Teiljahren
($k \leq m$) ist durch
%
\[
Z_{k/m} = (K_{0}-ka)\,\frac{(q-1)}{m}
\]
%
gegeben. \"Uber Aufsummieren der einzelnen Teilbeitr\"age
hat also die Zinsgutschrift f\"ur ein erstes verstrichenes
Kalenderjahr den Wert
%
\[
Z_{1} = \sum_{k=1}^{m}Z_{k/m}
= \sum_{k=1}^{m}(K_{0}-ka)\,\frac{(q-1)}{m}
= \frac{(q-1)}{m}\left[mK_{0}-a\sum_{k=1}^{m}k\right] \ .
\]
%
Unter Anwendung der Identit\"at~(\ref{id1}) l\"asst sich dieses
Ergebnis in die Form
%
\be
\lb{rentezins1}
Z_{1} = \left[K_{0}-\frac{1}{2}\,(m+1)a\right](q-1)
\ee
%
bringen. Beachten Sie die Tatsache, dass dieser Zinsbetrag linear
mit dem Auszahlungsbetrag $a$ bzw. mit der Anzahl der
Auszahlungen $m$ sinkt.

\medskip
\noindent
Der Kontostand nach Ablauf eines
ersten Kalenderjahres ist nun durch
%
\[
K_{1} = K_{0} - ma + Z_{1}
\overbrace{=}^{\text{Gl.}~(\ref{rentezins1})} K_{0}q
- \left[m+\frac{1}{2}\,(m+1)(q-1)\right]a
\]
%
gegeben. Nach Ablauf eines zweiten Kalenderjahres gilt
%
\[
Z_{2} = \left[K_{1}-\frac{1}{2}\,(m+1)a\right](q-1) \ ,
\]
%
sowie
%
\[
K_{2} = K_{1} - ma + Z_{2}
\overbrace{=}^{\text{Einsetzen f\"ur}\ K_{1}
\ \text{und}\ Z_{2}} K_{0}q^{2}
- \left[m+\frac{1}{2}\,(m+1)(q-1)\right]a(q+1) \ .
\]
%
In Analogie zu den sich hier abzeichnenden Strukturen,
haben die im Kalenderjahr $n$ verdienten {\bf Zinsen} (engl.: 
interest gained) den Wert
%
\be
Z_{n} = \left[K_{n-1}-\frac{1}{2}\,(m+1)a\right](q-1) \ ,
\ee
%
w\"ahrend der {\bf Kontostand} (engl.: account balance) nach $n$ 
Kalenderjahren

%
\[
K_{n} = K_{n-1} - ma + Z_{n}
\overbrace{=}^{\text{Einsetzen f\"ur}\ K_{n-1}
\ \text{und}\ Z_{n}} K_{0}q^{n}
- \left[m+\frac{1}{2}\,(m+1)(q-1)\right]a\sum_{k=0}^{n-1}q^{k}
\]
%
betr\"agt. Unter Verwendung der Identit\"at~(\ref{id2})
l\"asst sich $K_{n}$ abschlie\ss end schreiben als
%
\be
\lb{pensionex}
\fbox{$\displaystyle
\text{explizit:} \quad
K_{n} = K_{0}q^{n} - \left[m+\frac{1}{2}\,(m+1)(q-1)\right]
a\,\frac{q^{n}-1}{q-1} \ , \quad
n, m \in \mathbb{N} \ .
$
}
\ee
%
Wie schon bei der Tilgungsrechnung in Abschnitt \ref{sec:tilg},
wachsen auch hier die beiden gegeneinander agierenden Terme
auf der rechten Seite %dieser Gleichung
exponentiell mit $n$. Dabei
h\"angt es von den Werten der Gr\"o\ss en $K_{0}$, $q$ und $a$
sowie dem Parameter $m$ ab, ob der zweite Term den ersten (welcher
mit dem Vorsprung $K_{0}$ bei $n=0$ startet) einholen kann.

\medskip
\noindent
Gleichung (\ref{pensionex}) darf wieder beliebig algebraisch
umgeformt werden (solange nicht durch Null dividiert wird).
%
\begin{itemize}

\item[(i)] Die {\bf Laufzeit} (engl.: contract period) $n$ der 
Rente ergibt sich aus der
Forderung $K_{n}\stackrel{!}{=}0$ durch entsprechendes Aufl\"osen.
Unter der Voraussetzung $[\ldots]a-K_{0}(q-1)>0$, findet man
somit\footnote{Der Platzhalter $[\ldots]$ steht hier f\"ur den
konstanten Ausdruck $\left[m+\frac{1}{2}\,(m+1)(q-1)\right]$.}
%
\be
\lb{rentelaufzeit}
n = \frac{\ln\left(\frac{[\ldots]a}{
[\ldots]a-K_{0}(q-1)}\right)}{\ln(q)} \ .
\ee
%

\item[(ii)] Der {\bf Barwert} (engl.: present value) $B_{0}$ einer 
Rente ergibt sich aus der folgenden Fragestellung: Bei 
vorgegebenem j\"ahrlichen Aufzinsfaktor $q>1$ f\"ur ein 
Rentenkonto, welches Anfangskapital $K_{0}$ muss bereit gestellt 
werden, damit man sich \"uber $n$ Kalenderjahre hinweg $m$-mal 
j\"ahrlich den konstanten Betrag $a$ vorsch\"ussig auszahlen 
lassen kann?
Der Wert von $B_{0}=K_{0}$ ergibt sich wiederum aus der
Forderung $K_{n}\stackrel{!}{=}0$, die nun jedoch nach $K_{0}$
aufzul\"osen ist. Es ergibt sich daraus das Resultat
%
\be
\lb{rentebarwert}
B_{0} = K_{0} = \left[m+\frac{1}{2}\,(m+1)(q-1)\right]a\,
\frac{q^{n}-1}{q^{n}(q-1)} \ .
\ee
%

\item[(iii)] Die sogenannte (vorsch\"ussige) {\bf ewige Rente}
(engl.: everlasting pension payments) $a_{\rm ewig}$ basiert auf 
der Strategie, nur die anfallenden
Zinsen zu verkonsumieren, die ein Anfangskapital $K_{0}>0$ auf
einem Rentenkonto mit j\"ahrlichem Aufzinsfaktor $q>1$ abwirft.
Aufzul\"osen ist folglich die Bedingungsgleichung
$K_{n}\stackrel{!}{=}K_{0}$, f\"ur alle $n$, nach $a$. Dies
f\"uhrt zum Ergebnis
%
\be
a_{\rm ewig} = \frac{q-1}{m+\frac{1}{2}\,(m+1)(q-1)}\,K_{0} \ ;
\ee
%
$a_{\rm ewig}$ ist direkt proportional zu $K_{0}$!

\end{itemize}
%

%%%%%%%%%%%%%%%%%%%%%%%%%%%%%%%%%%%%%%%%%%%%%%%%%%%%%%%%%%%%%%%%%%%
\section[Lineare und geometrisch--degressive Abschreibungen]%
{Lineare und geometrisch--degressive Abschreibungen}
\lb{sec:abschr}
%%%%%%%%%%%%%%%%%%%%%%%%%%%%%%%%%%%%%%%%%%%%%%%%%%%%%%%%%%%%%%%%%%%
{\bf Abschreibung} (engl.: depreciation) nennt man den Versuch 
einer quantitatitven Beschreibung des materiellen Wertverlusts mit 
der Zeit eines Wirtschaftsguts vom {\bf Anfangswert} (engl.: 
initial value) $K_{0}>0$ (in GE). Im Steuerrecht gesetzlich 
verankert findet man f\"ur Investoren insbesondere die zwei 
nachfolgend diskutierten mathematischen Varianten der Abschreibung.

%------------------------------------------------------------------
\subsection{Lineare Abschreibung}
%------------------------------------------------------------------
Soll der {\bf Anfangswert} $K_{0}$ \"uber einen vorgegebenen
Zeitraum von $N$ Kalenderjahren zu gleichen Teilen {\em voll\/}
abgeschrieben werden, so wird der {\bf Restwert} (engl.: remaining 
value) nach Ablauf von $n$ Kalenderjahren (in GE) durch die 
Gleichung
%
\be
\fbox{$\displaystyle
R_{n} = K_{0} - n\left(\frac{K_{0}}{N}\right) \ ,
\quad n=1,\ldots,N \ ,
$}
\ee
%
dargestellt. F\"ur die Restwerte aus benachbarten Kalenderjahren
gilt hierbei $\displaystyle R_{n}-R_{n-1}
= -\left(\frac{K_{0}}{N}\right) =: d < 0$. Der {\bf linearen
Abschreibung} (engl.: straight line depreciation method) 
unterliegt also die Struktur einer arithmetischen
Zahlenfolge mit {\em negativer\/} Differenz $d$ zwischen ihren
Nachbargliedern (vgl. Abschnitt \ref{subsec:arithseq}).

%------------------------------------------------------------------
\subsection{Geometrisch--degressive Abschreibung}
%------------------------------------------------------------------
Grundlage der zweiten Abschreibungsnmethode f\"ur ein
Wirtschaftsgut mit {\bf Anfangswert} $K_{0}$ ist die \"Uberlegung,
pro abgelaufenem Kalenderjahr jeweils einen {\bf Prozentsatz} $p>0$
vom Restwert des Guts im Vorjahr abzuschreiben. Mit der Definition
eines dimensionslosen positiven {\bf Abschreibungsfaktors}
%
\be
q := 1-\frac{p}{100} < 1 \ ,
\ee
%
gilt dann f\"ur den {\bf Restwert} nach Ablauf von $n$
Kalenderjahren (in GE)
%
\be
\lb{deprrec}
\fbox{$\displaystyle
\text{rekursiv:} \quad
R_{n} = R_{n-1}q \ , \quad R_{0} \equiv K_{0} \ ,\quad
n\in\mathbb{N} \ .
$}
\ee
%
Der {\bf geometrisch--degressiven Abschreibung} (engl.: declining 
balance depreciation method) unterliegt
somit die Struktur einer geometrischen Zahlenfolge mit
konstantem Quotienten $q$ zwischen ihren Nachbargliedern
(vgl. Abschnitt \ref{subsec:geomseq}). Mit steigendem $n$ wird
der Wert dieser Glieder immer kleiner. Durch r\"uckwertige
Substitution kann der Ausdruck (\ref{deprrec}) in die
alternative Form
%
\be
\lb{deprex}
\fbox{$\displaystyle
\text{explizit:} \quad
R_{n} = K_{0}q^{n} \ , \quad 0 < q < 1 \ , \quad
n\in\mathbb{N} \ ,
$}
\ee
%
\"uberf\"uhrt werden.

\medskip
\noindent
Aus Gl.~(\ref{deprex}) lassen sich Antworten zu folgenden
Fragestellungen quantitativer Natur ableiten:

%
\begin{itemize}

\item[(i)] Vorgegeben seien ein Abschreibungsfaktor $q$ und
ein projizierter Restwert $R_{n}$ f\"ur ein Wirtschaftsgut.
Nach welcher {\bf Abschreibungszeit} (engl.: period of time) $n$ 
wird dieser Restwert erreicht?
%
\be
n = \frac{\ln\left(R_{n}/K_{0}\right)}{\ln(q)} \ .
\ee
%

\item[(ii)] Vorgegeben sind eine projizierte Abschreibungszeit
$n$ mit Restwert $R_{n}$. Zu welchem {\bf Prozentsatz} (engl.: 
percentage rate) $p$ muss das Wirtschaftsgut abgeschrieben werden?
%
\be
q = \sqrt[n]{\frac{R_{n}}{K_{0}}}
\quad\Rightarrow\quad
p = 100 \cdot \left(1-\sqrt[n]{\frac{R_{n}}{K_{0}}}\right) \ .
\ee
%

\end{itemize}
%

%%%%%%%%%%%%%%%%%%%%%%%%%%%%%%%%%%%%%%%%%%%%%%%%%%%%%%%%%%%%%%%%%%%
\section[Zusammenfassende Formel]{Zusammenfassende Formel}
\lb{sec:zus}
%%%%%%%%%%%%%%%%%%%%%%%%%%%%%%%%%%%%%%%%%%%%%%%%%%%%%%%%%%%%%%%%%%%
Abschlie\ss end wollen wir die Ergebnisse dieses Kapitels nochmals
rekapitulieren. Bemerkenswerterweise gelingt dies mit einer
einzigen Formel, aus welcher die oben diskutierten F\"alle durch
Spezialisierung hervorgehen. Diese Formel, die sich auf einen
Betrachtungszeitraum von $n$ Kalenderjahren bezieht, ist gegeben
durch (vgl. Zeh--Marschke (2010) \ct{zeh2010}):
%
\be
\lb{fincmastereq}
\fbox{$\displaystyle
K_{n} = K_{0}q^{n} + R\,\frac{q^{n}-1}{q-1} \ , \quad
q \in \mathbb{R}_{>0}\backslash\{1\} \ , \quad
n\in\mathbb{N} \ .
$}
\ee
%

\medskip
\noindent
Die darin enthaltenen Spezialf\"alle sind:
%
\begin{itemize}
\item[(i)] {\bf Zinseszins} f\"ur ein Anfangskapital $K_{0}>0$:
mit $R=0$ und $q>1$ reduziert sich Gl.~(\ref{fincmastereq}) auf
Gl.~(\ref{kap2}).

\item[(ii)] {\bf Vorsch\"ussiges Ratensparen} mit konstantem
Einzahlungsbetrag $E>0$: mit $K_{0}=0$, $q>1$ und $R=Eq$
reduziert sich Gl.~(\ref{fincmastereq}) auf Gl.~(\ref{kap4}).

\item[(iii)] {\bf Tilgung in konstanten Annuit\"aten}:
mit $K_{0}=-R_{0}<0$, $q>1$ und $R=A>0$ reduziert sich
Gl.~(\ref{fincmastereq}) auf das {\em Negative (!)\/}  von
Gl.~(\ref{remdebtex}). In dieser Formulierung werden
Restschulden $K_{n}=R_{n}$ also (sinnvollerweise) durch
negative Werte zum Ausdruck gebracht.

\item[(iv)] {\bf Rentenrechnung} mit vorsch\"ussigen Auszahlungen
eines konstanten Betrages $a>0$:
mit $q>1$ und $\displaystyle R=-\left[m
+\frac{1}{2}\,(m+1)(q-1)\right]a$
reduziert sich Gl.~(\ref{fincmastereq}) auf
Gl.~(\ref{pensionex}).

\item[(v)] {\bf Geometrisch--degressive Abschreibung}
eines Anfangswerts $K_{0}>0$: mit $R=0$ und $0<q<1$
reduziert sich Gl.~(\ref{fincmastereq}) auf
Gl.~(\ref{deprex}) f\"ur den Restwert $K_{n}=R_{n}$.

\end{itemize}
%

%%%%%%%%%%%%%%%%%%%%%%%%%%%%%%%%%%%%%%%%%%%%%%%%%%%%%%%%%%%%%%%%%%%
%%%%%%%%%%%%%%%%%%%%%%%%%%%%%%%%%%%%%%%%%%%%%%%%%%%%%%%%%%%%%%%%%%%
