%%%%%%%%%%%%%%%%%%%%%%%%%%%%%%%%%%%%%%%%%%%%%%%%%%%%%%%%%%%%%%%%%%%
%  File name: ch8.tex
%  Title:
%  Version: 22.08.2012 (hve)
%%%%%%%%%%%%%%%%%%%%%%%%%%%%%%%%%%%%%%%%%%%%%%%%%%%%%%%%%%%%%%%%%%%
%%%%%%%%%%%%%%%%%%%%%%%%%%%%%%%%%%%%%%%%%%%%%%%%%%%%%%%%%%%%%%%%%%%
\chapter[Integralrechnung bei reellen Funktionen]%
{Integralrechnung bei reellen Funktionen einer Variablen}
\lb{ch8}
%%%%%%%%%%%%%%%%%%%%%%%%%%%%%%%%%%%%%%%%%%%%%%%%%%%%%%%%%%%%%%%%%%%
\hfill\hbox{\fbox{\vbox{\hsize=10cm
Der Inhalt dieses Kapitels dient unter anderem der Vorbereitung
auf elementare Teilaspekte der Wahrscheinlichkeitstheorie, welche
im Rahmen der Module 0.1.3 WISS: Einf\"uhrung in das
wissenschaftliche Arbeiten und die empirische Sozialforschung
und 1.3.2 MARE: Marketing Research in der Schlie\ss enden 
Statistik ben\"otigt wird.
}}}

\vspace{10mm}
\noindent
Im letzten Kapitel dieser Vorlesungsnotizen besprechen wir in
einer kurzen \"Ubersicht die wesentlichen Definitionen und Regeln
der Integralrechnung bei reellen Funktionen einer Variablen. Daran
anschlie\ss end betrachten wir eine einfache \"okonomische
Anwendung dieses mathematischen Werkzeuges.

%%%%%%%%%%%%%%%%%%%%%%%%%%%%%%%%%%%%%%%%%%%%%%%%%%%%%%%%%%%%%%%%%%%
\section[Unbestimmte Integrale]{Unbestimmte Integrale}
\lb{sec:unbint}
%%%%%%%%%%%%%%%%%%%%%%%%%%%%%%%%%%%%%%%%%%%%%%%%%%%%%%%%%%%%%%%%%%%
\underline{\bf Def.:} $f$ sei eine stetige reelle Funktion und
$F$ eine differenzierbare reelle Funktion, mit $D(f)=D(F)$. Gilt
zwischen $f$ und $F$ die Beziehung
%
\be
\fbox{$\displaystyle
F^{\prime}(x) = f(x)
\qquad\text{{\em f\"ur alle\/}}\qquad x\in D(f) \ ,
$}
\ee
%
so hei\ss t $F$ {\bf Stammfunktion} (engl.: primitive) von $f$.

\medskip
\noindent
\underline{\bf Bem:} F\"ur eine gegebene stetige reelle
Funktion $f$ ist eine Stammfunktion niemals eindeutig. Denn nach
den in Abschnitt~\ref{sec:ablt} besprochenen
Differenziationsregeln ist neben $F$ auch immer $F+c$, mit
$c \in \mathbb{R}$ einer reellwertigen Konstante, Stammfunktion
zu~$f$, da $(c)^{\prime} = 0$.

\medskip
\noindent
\underline{\bf Def.:} Ist $F$ Stammfunktion zu einer stetigen
reellen Funktion $f$, so hei\ss t
%
\be
\fbox{$\displaystyle
\int f(x)\,{\rm d}x = F(x) + c \ , \quad
c~=~\text{konst.}~\in~\mathbb{R} \ ,
\quad \text{mit}\ F^{\prime}(x) = f(x)
$}
\ee
%
{\bf Unbestimmtes Integral} (engl.: indefinite integral) der 
Funktion $f$. In diesem Ausdruck werden
%
\begin{itemize}
\item $x$ --- {\bf Integrationsvariable} (engl.: integration 
variable),
\item $f(x)$ --- {\bf Integrand} (engl.: integrand),
\item ${\rm d}x$ --- {\bf Integrationsdifferenzial} (engl.: 
differential), und
\item $c$ --- {\bf Integrationskonstante} (engl.: constant of integration)
\end{itemize}
%
genannt.

\medskip
\noindent
Speziell f\"ur die in Abschnitt~\ref{sec:fkt} besprochenen
Familien elementarer stetiger reeller Funktionen einer
Variablen gelten folgende Integrationsregeln:

\medskip
\noindent
{\bf Integrationsregeln}
%
\begin{enumerate}
\item $\int \alpha\,{\rm d}x = \alpha x + c$ mit $\alpha
= \text{konstant} \in \mathbb{R}$ \hfill ({\bf Konstanten})
\item $\displaystyle \int x\,{\rm d}x = \frac{x^{2}}{2} + c$
 \hfill ({\bf Lineare Funktion})
\item $\displaystyle \int x^{n}\,{\rm d}x = \frac{x^{n+1}}{n+1} + c$
f\"ur $n \in \mathbb{N}$ \hfill ({\bf Nat\"urliche Potenzfunktionen})
\item $\displaystyle \int x^{\alpha}\,{\rm d}x = \frac{x^{\alpha+1}}{\alpha+1}
+ c$ f\"ur $\alpha \in \mathbb{R}\backslash\{-1\}$ und
$x \in \mathbb{R}_{> 0}$
\hfill ({\bf Allgemeine Potenzfunktionen})
\item $\displaystyle \int a^{x}\,{\rm d}x = \frac{a^{x}}{\ln(a)} + c$
f\"ur $a \in \mathbb{R}_{> 0}\backslash\{1\}$
\hfill ({\bf Exponentialfunktionen})
\item $\displaystyle \int e^{ax}\,{\rm d}x = \frac{e^{ax}}{a} + c$
f\"ur $a \in \mathbb{R}\backslash\{0\}$
\hfill ({\bf Nat\"urliche Exponentialfunktionen})
\item $\int x^{-1}\,{\rm d}x = \ln|x| + c$
f\"ur $x \in \mathbb{R}\backslash\{0\}$.
\end{enumerate}
%
Die Integration zusammengesetzer Funktionen erfordert
im allgemeinen die Anwendung besonderer Integrationsmethoden.
Nachfolgend listen wir die f\"ur diese Zwecke g\"angigen
Standardmethoden auf. F\"ur differenzierbare reelle
Funktionen $f$ und $g$ gilt:\footnote{(Engl.: summation rule, 
integration by parts, substitution method, logarithmic 
integration.)}
%
\begin{enumerate}
\item $\int(\alpha f(x) \pm \beta g(x))\,{\rm d}x
= \alpha\int f(x)\,{\rm d}x \pm \beta\int g(x)\,{\rm d}x$ \\
mit $\alpha,\beta = \text{konstant} \in \mathbb{R}$
\hfill ({\bf Summenregel})
\item $\int f(x)g^{\prime}(x)\,{\rm d}x = f(x)g(x)
- \int f^{\prime}(x)g(x)\,{\rm d}x$
\hfill ({\bf Partielle Integration})
\item $\int f(g(x))g^{\prime}(x)\,{\rm d}x
\overbrace{=}^{u=g(x)\ \text{und}\ {\rm d}u=g^{\prime}(x){\rm d}x}
\int f(u)\,{\rm d}u = F(g(x)) + c$
\hfill ({\bf Substitutionsmethode})
\item $\displaystyle \int\frac{f^{\prime}(x)}{f(x)}\,{\rm d}x
= \ln|f(x)| + c$ f\"ur $f(x) \neq 0$
\hfill ({\bf Logarithmische Integration}).
\end{enumerate}
%

%%%%%%%%%%%%%%%%%%%%%%%%%%%%%%%%%%%%%%%%%%%%%%%%%%%%%%%%%%%%%%%%%%%
\section[Bestimmte Integrale]{Bestimmte Integrale}
\lb{sec:bint}
%%%%%%%%%%%%%%%%%%%%%%%%%%%%%%%%%%%%%%%%%%%%%%%%%%%%%%%%%%%%%%%%%%%
\medskip
\noindent
\underline{\bf Def.:} Sei $f$ eine auf dem Intervall $[a,b]
\subset D(f)$ stetige reelle Funktion einer Variablen, und $F$ eine
beliebige Stammfunktion von $f$. Dann definiert der Ausdruck
%
\be
\fbox{$\displaystyle
\int_{a}^{b}f(x)\,{\rm d}x = \left.F(x)\right|_{x=a}^{x=b}
= F(b) - F(a)
$}
\ee
%
das {\bf bestimmte Integral} (engl.: definite integral) von $f$ in 
den {\bf Integrationsgrenzen} (engl.: limits of integration) $a$ 
und $b$.

\medskip
\noindent
F\"ur bestimmte Integrale gelten allgemein die Regeln:
%
\begin{enumerate}
\item $\int_{a}^{a}f(x)\,{\rm d}x = 0$
\hfill ({\bf Zusammenfallende Integrationsgrenzen})
\item $\int_{b}^{a}f(x)\,{\rm d}x = -\int_{a}^{b}f(x)\,{\rm d}x$
\hfill ({\bf Vertauschte Integrationsgrenzen})
\item $\int_{a}^{b}f(x)\,{\rm d}x = \int_{a}^{c}f(x)\,{\rm d}x
+ \int_{c}^{b}f(x)\,{\rm d}x$ f\"ur $c \in [a,b]$
\hfill ({\bf Integrationszwischenstelle}).
\end{enumerate}
%

\medskip
\noindent
\underline{\bf Bem:} Der wesentliche qualitative Unterschied
zwischen (i)~einem unbestimmten und (ii)~einem bestimmten
Integral einer stetigen reellen Funktion einer Variablen ist
durch die Art des Ergebnisses bedingt: (i)~liefert als Resultat
eine reelle (Stamm-){\em Funktion\/}, (ii)~eine einzige reelle
{\em Zahl\/}.

\medskip
\noindent
\underline{\bf GTR:} Die Berechnung von bestimmten Integralen
%der Form $\int_{a}^{b}f(x)\,{\rm d}x$
kann f\"ur eine eingespeicherte Funktion im Modus {\tt CALC} mit
der vorprogrammierten Funktion $\int${\tt f(x)dx} durchgef\"uhrt
werden. Die Eingabe der erforderlichen Integrationsgrenzen erfolgt
hierbei interaktiv.

\medskip
\noindent
Wie schon in Abschnitt \ref{sec:elast} angedeutet, spielen
in praktischen Anwendungen skaleninvariante allgemeine
Potenzfunktionen $f(x)=x^{\alpha}$ f\"ur $\alpha \in \mathbb{R}$
und $x \in \mathbb{R}_{> 0}$ eine bedeutsame Rolle. Mit $x \in
\left[a,b\right] \subset \mathbb{R}_{> 0}$ und $\alpha \neq -1$
gilt also
%
\be
\int_{a}^{b}x^{\alpha}\,{\rm d}x
= \left.\frac{x^{\alpha+1}}{\alpha+1}\right|_{x=a}^{x=b}
= \frac{1}{\alpha+1}\left(b^{\alpha+1}-a^{\alpha+1}\right) \ .
\ee
%
Problematisch k\"onnen in diesem Zusammenhang
Grenzwertbetrachtungen der Art $a \to 0$ bzw. $b \to \infty$
werden. Denn die Forderungen
%
\begin{itemize}
\item[(i)] Fall $\alpha < -1$:
%
\be
\lim_{a \to 0}\int_{a}^{b}x^{\alpha}\,{\rm d}x \to \infty \ ,
\ee
%
\item[(ii)] Fall $\alpha > -1$:
%
\be
\lim_{b \to \infty}\int_{a}^{b}x^{\alpha}\,{\rm d}x \to \infty \ ,
\ee
%
\end{itemize}
%
f\"uhren jeweils zu {\bf divergenten} (engl.: divergent) 
mathematischen Ausdr\"ucken.

%%%%%%%%%%%%%%%%%%%%%%%%%%%%%%%%%%%%%%%%%%%%%%%%%%%%%%%%%%%%%%%%%%%
\section[\"Okonomische Anwendungen]{\"Okonomische Anwendungen}
\lb{sec:intanw}
%%%%%%%%%%%%%%%%%%%%%%%%%%%%%%%%%%%%%%%%%%%%%%%%%%%%%%%%%%%%%%%%%%%
Ausgangspunkt sei eine vereinfachte Marktsituation f\"ur ein
einziges Produkt. Die Marktsituation werde einerseits durch eine
Nachfragefunktion $N(p)$ (in ME) beschrieben, welche auf dem
Preisintervall $[p_{u},p_{o}]$ monoton fallend sei. Die Werte
$p_{u}$ und $p_{o}$ bezeichen Mindest-- und H\"ochstst\"uckpreise
(in GE/ME) f\"ur das Produkt. Andererseits werde die Marktsituation
beschrieben durch eine Angebotsfunktion $A(p)$ (in ME), die auf dem
Preisintervall $[p_{u},p_{o}]$ monoton steigend sei.

\medskip
\noindent
Der {\bf Marktpreis} (engl.: equilibrium price) $p_{M}$ (in GE/ME) 
f\"ur das Produkt definiert sich \"uber den Zustand des {\bf 
\"okonomischen Gleichgewichtes} (engl.: economic equilibrium), 
d.h.\ durch die Bedingung
%
\be
A(p_{M}) = N(p_{M}) \ ,
\ee
%
wodurch ein gemeinsamer Schnittpunkt f\"ur die beiden Funktionen
$A(p)$ und $N(p)$ beschrieben wird.

\medskip
\noindent
\underline{\bf GTR:} F\"ur eingespeicherte Funktionen $f$ und $g$
k\"onnen gemeinsame Schnittpunkte leicht im Modus {\tt CALC}
mit der vorprogrammierten interaktiven Funktion {\tt intersect}
bestimmt werden.

\medskip
\noindent
In stark vereinfachter Weise wollen wir nun den von den
Anbietern unter drei unterschiedlichen
{\bf Markteinf\"uhrungsstrategien} f\"ur das Produkt erzielten
Umsatz berechnen.
%
\begin{enumerate}

\item {\bf Strategie 1:} Werden die angebotenen Mengen des Produktes
direkt zum Marktpreis $p_{M}$ verkauft, so betr\"agt der Umsatz der
Anbieter
%
\be
U_{1} = U(p_{M}) = p_{M}N(p_{M}) \ .
\ee
%

\item {\bf Strategie 2:} Manche Konsumenten des Produktes w\"aren
bereit, dieses bei Markteinf\"uhrung auch zu einem h\"oheren
St\"uckpreis als $p_{M}$ zu erwerben. Entscheiden sich also die
Anbieter f\"ur einen Einf\"uhrungsst\"uckpreis $p_{o}$, um sodann,
mit der Absicht die Nachfrage zu steigern, den St\"uckpreis
stetig\footnote{Dies ist eine drastisch vereinfachendene Annahme,
die die bei der anstehenden Berechnung durchzuf\"uhrenden
mathematischen Schritte erleichtern soll.} (!) bis auf $p_{M}$ zu
reduzieren, bel\"auft sich der erzielte Umsatz auf den h\"oheren
Wert
%
\be
U_{2} = U(p_{M}) + \int_{p_{M}}^{p_{o}}N(p)\,{\rm d}p \ .
\ee
%
Da der durch
%
\be
K := \int_{p_{M}}^{p_{o}}N(p)\,{\rm d}p \qquad
(\text{in\ GE})
\ee
%
definierte Geldbetrag von den Konsumenten gespart wird, wenn
das Produkt nach Strategie~1 in den Markt eingef\"uhrt wird,
wird dieser in der wirtschaftswissenschaftlichen Fachliteratur
{\bf Konsumentenrente} (engl.: consumer surplus) genannt.

\item {\bf Strategie 3:} Manche Produzenten w\"aren bereit, das
Produkt bei Markteinf\"uhrung auch zu einem niedrigeren
St\"uckpreis als $p_{M}$ anzubieten. Entscheiden sich die Anbieter
also f\"ur einen Einf\"uhrungsst\"uckpreis $p_{u}$, um sodann
den St\"uckpreis stetig\footnote{Dies ist eine drastisch
vereinfachendene Annahme, die die bei der anstehenden Berechnung
durchzuf\"uhrenden mathematischen Schritte erleichtern soll.} (!)
bis auf $p_{M}$ zu erh\"ohen, bel\"auft sich der erzielte Umsatz
auf den niedrigeren Wert
%
\be
U_{3} = U(p_{M}) - \int_{p_{u}}^{p_{M}}A(p)\,{\rm d}p \ .
\ee
%
Da die Produzenten den durch
%
\be
P := \int_{p_{u}}^{p_{M}}A(p)\,{\rm d}p \qquad
(\text{in\ GE})
\ee
%
definierten Geldbetrag mehr einnehmen, wenn sie das Produkt nach
Strategie~1 in den Markt einf\"uhren, wird dieser in der
wirtschaftswissenschaftlichen Fachliteratur {\bf Produzentenrente}
(engl.: producer surplus) genannt.

\end{enumerate}
%

%%%%%%%%%%%%%%%%%%%%%%%%%%%%%%%%%%%%%%%%%%%%%%%%%%%%%%%%%%%%%%%%%%%
%%%%%%%%%%%%%%%%%%%%%%%%%%%%%%%%%%%%%%%%%%%%%%%%%%%%%%%%%%%%%%%%%%%