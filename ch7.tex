%%%%%%%%%%%%%%%%%%%%%%%%%%%%%%%%%%%%%%%%%%%%%%%%%%%%%%%%%%%%%%%%%%%
%  File name: ch7.tex
%  Title:
%  Version: 10.09.2015 (hve)
%%%%%%%%%%%%%%%%%%%%%%%%%%%%%%%%%%%%%%%%%%%%%%%%%%%%%%%%%%%%%%%%%%%
%%%%%%%%%%%%%%%%%%%%%%%%%%%%%%%%%%%%%%%%%%%%%%%%%%%%%%%%%%%%%%%%%%%
\chapter[Differenzialrechnung bei reellen Funktionen]%
{Differenzialrechnung bei reellen Funktionen einer Variablen}
\lb{ch7}
%%%%%%%%%%%%%%%%%%%%%%%%%%%%%%%%%%%%%%%%%%%%%%%%%%%%%%%%%%%%%%%%%%%
\hfill\hbox{\fbox{\vbox{\hsize=10cm
Der Inhalt dieses Kapitels ist relevant f\"ur das
Verst\"andnis der in Kapitel 5.5 des Lehrbuchs von Schrey\"ogg und
Koch (2010)~\ct{schkoc2010} behandelten quantitativen Themen.
%Gleichfalls f\"ur Kapitel 5.3, 10.4, 10.5, 11.4, 13.3, 13.4, 14.4
%des Lehrbuchs von Schmalen und Pechtl 2006~\ct{schpec2006}.
}}}

\vspace{10mm}
\noindent
Die Kapitel \ref{ch1} bis \ref{ch5} dieser Vorlesungsnotizen
beschr\"ankten
sich auf die Diskussion {\em linearer\/} funktionaler Beziehungen
zwischen gegebenen INPUT--Gr\"o\ss en und zugeordneten
OUTPUT--Gr\"o\ss en. In diesem Kapitel nun betrachten wir
vornehmlich quantitative Relationen zwischen (einer)
INPUT--Gr\"o\ss en und (einer) zugeordneten OUTPUT--Gr\"o\ss en,
die von {\bf nichtlinearer Natur} (engl.: non-linear nature) sind.

%%%%%%%%%%%%%%%%%%%%%%%%%%%%%%%%%%%%%%%%%%%%%%%%%%%%%%%%%%%%%%%%%%%
\section[Reelle Funktionen einer Variablen]%
{Reelle Funktionen einer Variablen}
\lb{sec:fkt}
%%%%%%%%%%%%%%%%%%%%%%%%%%%%%%%%%%%%%%%%%%%%%%%%%%%%%%%%%%%%%%%%%%%
Wir beginnen mit der Einf\"uhrung des Begriffs einer reellen
Funktion einer Variablen. Das Wesen einer reellen Funktion
einer Variablen ist das einer {\bf Abbildung} (engl.: 
mapping),\footnote{Vergleiche
hierzu unsere Einf\"uhrung von Matrizen als mathematische Objekte
in Kapitel~\ref{ch2}.} welche eine einfache aber strenge Regel zu
befolgen hat:

\medskip
eine Funktion $f$ muss {\em jedem\/} Wert $x$ aus
einer Menge von reellen Zahlen $D \subseteq \mathbb{R}$
{\em genau einen\/} Wert $y$ aus einer Menge von reellen Zahlen
$W\subseteq \mathbb{R}$ zuordnen.

\medskip
\noindent
\underline{\bf Def.:}
Eine {\bf eindeutige} (engl.: unique) Abbildung $f$ einer Teilmenge
$D\subseteq\mathbb{R}$ der reellen Zahlen auf eine
Teilmenge $W\subseteq\mathbb{R}$ der reellen Zahlen,
%
\be
\fbox{$\displaystyle
f\!: D \rightarrow W \ ,
\hspace{10mm} x \mapsto y=f(x)
$}
\ee
%
hei\ss t {\bf reelle Funktion einer Variablen} (engl.: real-valued 
function of one variable).

\medskip
\noindent
Zentrale Aspekte des Konzepts einer reellen Funktionen
einer Variablen werden beschrieben durch die Begriffe:
\pagebreak
%
\begin{itemize}
\item $D$: {\bf Definitionsbereich} (engl.: domain) von $f$,
\item $W$: {\bf Wertebereich} (engl.: range) von $f$,
\item $x \in D$: {\bf unabh\"angige Variable} ((engl.: independent 
variable) von $f$, auch Argument (engl.: argument) von $f$ genannt,
\item $y \in W$: {\bf abh\"angige Variable} (engl.: dependent 
variable) von $f$,
\item $f(x)$: {\bf Abbildungsvorschrift} (engl.: mapping 
instruction),
\item {\bf Graph} (engl.: graph) von $f$: Zahlenpaare 
$G=\{(x,f(x))|x \in D\} \subset \mathbb{R}^{2}$.
\end{itemize}
%
F\"ur die sp\"atere Analyse der Eigenschaften bestimmter
reeller Funktionen einer Variablen ben\"otigen wir die
Definitionen weiterer wichtiger Begriffe.

\medskip
\noindent
\underline{\bf Def.:}
Die einer {\bf eineindeutigen} (engl.: one-to-one and onto) 
Abbildung $f$, mit $D(f)\subseteq\mathbb{R}$ und 
$W(f)\subseteq\mathbb{R}$, zugeordnete
Abbildung $f^{-1}$, mit $D(f^{-1})=W(f)$ und $W(f^{-1})=D(f)$,
wird {\bf inverse Funktion} (oder Umkehrfunktion) (engl.: inverse 
function) zu $f$ genannt.

\medskip
\noindent
Bei einer eineindeutigen Abbildung $f$ wird nicht nur jedem
$x \in D(f)$ genau ein $y \in W(f)$ zugeordnet, sondern
gleichfalls auch jedem $y \in W(f)$ genau ein $x \in D(f)$.

\medskip
\noindent
\underline{\bf Def.:}
Eine reelle Funktion $f$ einer Variablen hei\ss t {\bf stetig
an der Stelle} $x \in D(f)$ (engl.: continuous at), wenn sie f\"ur
$\Delta x \in \mathbb{R}_{>0}$ die Bedingungenen
%
\be
\lim_{\Delta x \to 0}f(x-\Delta x)
= \lim_{\Delta x \to 0}f(x+\Delta x)
= f(x)
\ee
%
erf\"ullt, wenn also bei $x$ die linken und rechten
Grenzwerte der Funktion $f$ mit dem Funktionswert $f(x)$
zusammenfallen. Eine reelle Funktion $f$ hei\ss t {\bf stetig}
(engl.: continuous), wenn sie {\em f\"ur alle\/} $x \in D(f)$ 
stetig ist.

\medskip
\noindent
\underline{\bf Def.:}
Gilt f\"ur eine reelle Funktion $f$ einer Variablen die
Beziehung
%
\be
f(a) < f(b)
\qquad\text{{\em f\"ur alle\/}}\qquad
a,b \in D(f)\ \text{mit}\ a < b \ ,
\ee
%
so wird $f$
{\bf streng monoton steigend} (engl.: strictly monotonously 
increasing) genannt. Gilt hingegen die Beziehung
%
\be
f(a) > f(b)
\qquad\text{{\em f\"ur alle\/}}\qquad
a,b \in D(f)\ \text{mit}\ a < b \ ,
\ee
%
so wird $f$ {\bf streng monoton fallend} (engl.: strictly 
monotonously decreasing) genannt.

\medskip
\noindent
Beachten Sie, dass insbesondere {\em streng monotone stetige
Funktionen\/} immer eineindeutige Abbildungen darstellen und
deshalb invertierbar (umkehrbar) sind.

\medskip
\noindent
Nun wenden wir uns kurz f\"unf, in wirtschaftswissenschaftlichen
Anwendungen h\"aufig auftretenden Familien reeller Funktionen
einer Variablen und deren wichtigsten qualitativen Eigenschaften
zu.

%------------------------------------------------------------------
\subsection{Polynome vom Grad $n$}
\lb{subsec:polynomials}
%------------------------------------------------------------------
Polynome vom Grad $n$ (engl.: polynomials of degree $n$) sind 
reelle Funktionen einer Variablen der Form
%
\be
\lb{npol}
\fbox{$\displaystyle\begin{array}{c}
y = f(x) = a_{n}x^{n}+a_{n-1}x^{n-1}+\ldots
+a_{i}x^{i}+\ldots+a_{2}x^{2}+a_{1}x+a_{0}
\\[5mm]
\text{mit}\ a_{i}\in\mathbb{R},\ i=1,\ldots,n,
\ \ n\in\mathbb{N},\ a_{n}\neq 0 \ .
\end{array}
$}
\ee
%
Ihr Definitionsbereich ist $D(f)=\mathbb{R}$. Ihr Wertebereich
h\"angt im konkreten Fall von den Werten der konstanten
Koeffizienten $a_{i}\in\mathbb{R}$ ab. Diese Funktionen besitzen
maximal $n$ reelle {\bf Nullstellen}.

%------------------------------------------------------------------
\subsection{Gebrochen rationale Funktionen}
%------------------------------------------------------------------
Gebrochen rationale Funktionen (engl.: rational functions) werden 
durch den {\bf Quotienten zweier Polynome} vom Grad $m$ bzw.\ vom 
Grad $n$ gebildet, d.h.
%
\be
\fbox{$\displaystyle\begin{array}{c}
y = f(x) = \displaystyle{\frac{p_{m}(x)}{q_{n}(x)}
= \frac{a_{m}x^{m}+\ldots+a_{1}x+a_{0}}{
b_{n}x^{n}+\ldots+b_{1}x+b_{0}}}
\\[5mm]
\text{mit}\ a_{i},b_{j}\in\mathbb{R},\ i=1,\ldots,m,
\ j=1,\ldots,n,
\ \ m,n\in\mathbb{N},\ a_{m},b_{n}\neq 0 \ .
\end{array}
$}
\ee
%
Ihr Definitionsbereich ist
$D(f)=\mathbb{R}\backslash\{x|q_{n}(x)=0\}$. Wenn f\"ur die
Grade $m$ und $n$ der Polynome $p_{m}(x)$ und $q_{n}(x)$ gilt
\begin{itemize}
\item[(i)] $m<n$, so wird $f$ {\bf echt gebrochen rational} 
(engl.: proper rational functions), bzw.
\item[(ii)] $m\geq n$, so wird $f$ {\bf unecht gebrochen rational} 
(engl.: improper rational functions)
\end{itemize}
genannt. Im zweiten Fall l\"asst sich durch {\em Polynomdivision\/}
(engl.: polynomial division) immer ein rein polynomialer Anteil 
von $f$ abspalten. Die {\bf Nullstellen} (engl.: roots) von $f$ 
sind die Nullstellen des 
Z\"ahlerpolynoms $p_{m}(x)$, f\"ur welche gleichzeitig $q_{n}(x) 
\neq 0$ gilt. Nullstellen des Nennerpolynoms $q_{n}(x)$ stellen so 
genannte {\bf Polstellen} (engl.: poles) von $f$ dar. Echt 
gebrochen rationale Funktionen tendieren f\"ur sehr kleine ($x \to 
-\infty$) bzw.\ sehr gro\ss e($x \to +\infty$) Argumentwerte immer 
gegen Null.
%{\bf Asymptoten}

%------------------------------------------------------------------
\subsection{Potenzfunktionen}
%------------------------------------------------------------------
Potenzfunktionen (engl.: power-law functions) sind Funktionen der 
Bauart
%
\be
\fbox{$\displaystyle
y = f(x) = x^{\alpha} \quad
\text{mit}\ \alpha\in\mathbb{R} \ .
$}
\ee
%
Hier beschr\"anken wir uns auf F\"alle mit Definitionsbereich
$D(f)=\mathbb{R}_{>0}$, sodass als Wertebereich
$W(f)=\mathbb{R}_{>0}$ folgt. Unter den genannten Bedingungen sind
Potenzfunktionen streng monoton steigend f\"ur $\alpha > 0$, und
streng monoton fallend f\"ur $\alpha < 0$; also k\"onnen sie durch
$y = \sqrt[\alpha]{x} = x^{1/\alpha}$ invertiert werden.
Nullstellen liegen unter der hier eingef\"uhrten Beschr\"ankung
keine vor.

%------------------------------------------------------------------
\subsection{Exponentialfunktionen}
\lb{subsec:exponentials}
%------------------------------------------------------------------
Exponentialfunktionen (engl.: exponential functions) sind durch 
Konstruktionen der Form
%
\be
\fbox{$\displaystyle
y = f(x) = a^{x} \quad
\text{mit}\ a\in\mathbb{R}_{>0}\backslash\{1\}
$}
\ee
%
gegeben. Ihr Definitionsbereich ist $D(f)=\mathbb{R}$, ihr
Wertebereich $W(f)=\mathbb{R}_{>0}$. Sie zeigen streng monoton
steigendes Verhalten f\"ur $a>1$, und streng monoton fallendes
Verhalten f\"ur $0<a<1$. Folglich k\"onnen sie invertiert werden.
Ihr $y$--Achsenabschnitt liegt generell bei $y=1$.
%Nullstellen besitzen sie keine.
Exponentialfunktionen sind (f\"ur $a>1$) auch
unter dem Namen {\bf Wachstumsfunktionen} bekannt.

\medskip
\noindent
\underline{Spezialfall:} Wird die konstante (!) Basiszahl zu
$a=e$ gew\"ahlt, wobei $e$ die irrationale {\bf Eulersche Zahl}
(benannt nach dem schweizer Mathematiker
\href{http://turnbull.mcs.st-and.ac.uk/history/Biographies/Euler.html}{Leonhard
Euler, 1707--1783}) darstellt, welche durch die unendliche
Zahlenreihe
%
\[
e := \sum_{k=0}^{\infty}\frac{1}{k!}
= \frac{1}{0!} + \frac{1}{1!} + \frac{1}{2!}
+ \frac{1}{3!} + \ldots
\]
%
definiert ist, so spricht man von der
{\bf nat\"urlichen Exponentialfunktion} (engl.: natural 
exponential function)
%
\be
y=f(x)=e^{x} =: \exp(x) \ .
\ee
%
Analog zur Definition von $e$ gilt:
%
\[
e^{x} = \exp(x)
= \sum_{k=0}^{\infty}\frac{x^{k}}{k!}
= \frac{x^{0}}{0!} + \frac{x^{1}}{1!} + \frac{x^{2}}{2!}
+ \frac{x^{3}}{3!} + \ldots \ .
\]
%

%------------------------------------------------------------------
\subsection{Logarithmusfunktionen}
%------------------------------------------------------------------
Logarithmusfunktionen (engl.: logarithmic functions), gegeben durch
%
\be
\fbox{$\displaystyle
y = f(x) = \log_{a}(x) \quad
\text{mit}\ a\in\mathbb{R}_{>0}\backslash\{1\} \ ,
$}
\ee
%
sind als inverse Funktionen (Umkehrfunktionen) der streng monotonen
Exponentialfunktionen $y=f(x)=a^{x}$ definiert --- und umgekehrt.
Folglich gilt f\"ur sie $D(f)=\mathbb{R}_{>0}$ und
$W(f)=\mathbb{R}$. Streng monoton steigendes Verhalten liegt vor
f\"ur $a>1$, streng monoton fallendes Verhalten f\"ur $0<a<1$.
Ihr $x$--Achsenabschnitt ist generell $x=1$.

\medskip
\noindent
\underline{Spezialfall:} Die {\bf nat\"urliche 
Logarithmusfunktion} (engl.: natural logarithmic function)
erhalten wir, wenn die Basiszahl zu $a=e$ gew\"ahlt wird; also

%
\be
y=f(x)=\log_{e}(x):=\ln(x) \ .
\ee
%

%------------------------------------------------------------------
\subsection{Zusammengesetzte Funktionen}
\lb{subsec:kombfkt}
%------------------------------------------------------------------
Funktionen aus allen f\"unf hier betrachteten Familien k\"onnen
sowohl \"uber die vier Grundrechenarten, als auch durch Verkettung
bzw.\ Verschachtelung (im Rahmen g\"ultiger Rechenregeln) beliebig
miteinander kombiniert werden (engl.: concatenations of 
real-valued functions).

\medskip
\noindent
\underline{\bf Satz:} Seien $f$ und $g$ auf $D(f)$ bzw.\ $D(g)$
stetige reelle Funktionen. Dann sind auch die kombinierten
Funktionen
%
\begin{enumerate}
	\item {\bf Summe/Differenz} $f \pm g$, wobei $(f \pm g)(x)
	:=f(x) \pm g(x)$ mit $D(f) \cap D(g)$,
	\item {\bf Produkt} $f \cdot g$, wobei  $(f\cdot g)(x):=f(x)g(x)$
	mit $D(f) \cap D(g)$,
	\item {\bf Quotient} ${\displaystyle\frac{f}{g}}$, wobei 
	${\displaystyle\left(\frac{f}{g}\right)(x):=\frac{f(x)}{g(x)}}$ 
	mit $g(x) \neq 0$ und $D(f) \cap D(g)
	\backslash\{x|g(x)=0\}$,
	\item {\bf Verkettung} $f \circ g$, wobei
	$(f\circ g)(x):=f(g(x))$ mit $\{x \in D(g)|g(x) \in D(f)\}$,
\end{enumerate}
%
stetig auf den entsprechenden Definitionsbereichen.

%%%%%%%%%%%%%%%%%%%%%%%%%%%%%%%%%%%%%%%%%%%%%%%%%%%%%%%%%%%%%%%%%%%
\section[Ableitungen einer differenzierbaren reellen Funktion]%
{Ableitungen einer differenzierbaren reellen Funktion}
\lb{sec:ablt}
%%%%%%%%%%%%%%%%%%%%%%%%%%%%%%%%%%%%%%%%%%%%%%%%%%%%%%%%%%%%%%%%%%%
Das zentrale Thema dieses Kapitels ist die mathematische
Beschreibung des {\bf lokalen \"Anderungsverhaltens} (engl.: local 
variability) von
{\em stetigen\/} reellen Funktion einer Variablen,
$f:D\subseteq\mathbb{R} \rightarrow W\subseteq\mathbb{R}$, deren
Grundlagen wir uns nun zuwenden. Dazu wollen wir eine kleine
\"Anderung des Wertes des Argumentes $x$ von $f$ betrachten,
d.h.\ wir ver\"andern $x \rightarrow x+\Delta x$ mit $\Delta x
\in \mathbb{R}$ und untersuchen, welche Auswirkung auf $f$ diese
\"Anderung von $x$ nach sich zieht. Wir finden unmittelbar, dass
sich aus der vorgegebenen \"Anderung die Konsequenz
$y \rightarrow y+\Delta y = f(x+\Delta x)$ mit $\Delta y
\in \mathbb{R}$ einstellt. Im Fazit: eine vorgegenene \"Anderung
des Argumentes $x$ der Gr\"o\ss e $\Delta x$ bewirkt f\"ur $f$
eine \"Anderung der Gr\"o\ss e $\Delta y = f(x+\Delta x)-f(x)$.
Interessant ist nun f\"ur uns ein {\bf Gr\"o\ss envergleich}
dieser beiden \"Anderungen. Dazu bilden wir das Verh\"altnis
%
\[
\frac{\Delta y}{\Delta x} = \frac{f(x+\Delta x)-f(x)}{\Delta x}
\ ,
\]
%
auch {\em Differenzenquotient\/} f\"ur $f$ genannt. Eine
naheliegende Frage lautet nun: wie verh\"alt sich der Wert dieses
Differenzenquotienten im Grenzfall, dass wir die vorgegebene
\"Anderung $\Delta x$ des Argumentes von $f$ kontinuierlich
kleiner werden lassen, bis sie auf Null gesunken ist?

\medskip
\noindent
\underline{\bf Def.:}
Eine stetige reelle Funktion $f$ einer Variablen hei\ss t
{\bf differenzierbar an der Stelle} (engl.: differentiable at) $x 
\in D(f)$, wenn f\"ur beliebiges $\Delta x \in \mathbb{R}$ der 
Grenzwert
%
\be
\fbox{$\displaystyle
f^{\prime}(x) := \lim_{\Delta x \to 0}
\frac{\Delta y}{\Delta x}
= \lim_{\Delta x \to 0}
\frac{f(x+\Delta x)-f(x)}{\Delta x}
$}
\ee
%
existiert und eindeutig ist. Ist $f$ differenzierbar
{\em f\"ur alle\/} $x \in D(f)$, so wird $f$ {\bf differenzierbar} 
(engl.: differentiable) genannt.

\medskip
\noindent
Die Existenz dieses Grenzwertes in einem Punkt $(x,f(x))$ setzt
f\"ur die Funktion $f$ in diesem Punkt faktisch die Abwesenheit
von "`Spr\"ungen und Knicken"' voraus, dass also $f$ in
$(x,f(x))$ hinreichend "`glatt"' ist. Die Gr\"o\ss e
$f^{\prime}(x)$ wird {\bf erste Ableitung} (engl.: first 
derivative) der (differenzierbaren) Funktion $f$ an der Stelle $x$ 
genannt. Sie ist ein Ma\ss\ f\"ur die {\bf lokale \"Anderungsrate}
(engl.: local rate of change) der Funktion $f$ im Punkt $(x,f(x))$.
Allgemein gilt f\"ur die erste Ableitung $f^{\prime}(x)$ die
Interpretation: wird das Argument $x$ einer differenzierbaren
reellen Funktion $f$ um $1$ (eine Einheit) erh\"oht, so \"andert
sich die Funktion $f$ in Konsequenz dessen n\"aherungsweise um
$f^{\prime}(x)\cdot 1$ (Einheiten).

\medskip
\noindent
Alternative Schreibweise:
%
$$
f^{\prime}(x) \equiv \frac{{\rm d}f(x)}{{\rm d}x} \ .
$$
%

\medskip
\noindent
Die Differenzialrechnung wurde zusammen mit der Integralrechnung
(siehe Kapitel~\ref{ch7}) w\"ahrend der zweiten H\"alfte des
17.\ Jahrhunderts unabh\"angig voneinander von dem englischen
Physiker, Mathematiker, Astronomen und Philosophen
\href{http://turnbull.mcs.st-and.ac.uk/history/Biographies/Newton.html}{Sir
Isaac Newton (1643--1727)} sowie dem deutschen Philosophen,
Mathematiker und Physiker
\href{http://turnbull.mcs.st-and.ac.uk/history/Biographies/Leibniz.html}{Gottfried
Wilhelm Leibniz (1646--1716)} entwickelt.

\medskip
\noindent
\"Uber die erste Ableitung einer differenzierbaren Funktion $f$
an einer Stelle $x_{0} \in D(f)$, also $f^{\prime}(x_{0})$,
definieren wir die so genannte {\bf Linearisierung} (engl.: 
linearisation) von $f$ in einer Umgebung der Stelle $x_{0}$. Die 
damit verbundene Gleichung der {\bf Tangente} (engl.: tangent) an 
$f$ im Punkt $(x_{0},f(x_{0}))$ ist durch
%
\be
\lb{ftangente}
\fbox{$\displaystyle
y = f(x_{0}) + f^{\prime}(x_{0})(x-x_{0})
$}
\ee
%
gegeben.

\medskip
\noindent
\underline{\bf GTR:} Die Berechnung lokaler Ableitungswerte
$f^{\prime}(x_{0})$ kann, f\"ur vorgegebenes, eingespeichertes $f$,
im Modus {\tt CALC} mit der vorprogrammierten interaktiven
Funktion {\tt dy/dx} durchgef\"uhrt werden.

\medskip
\noindent
F\"ur die in Abschnitt~\ref{sec:fkt} besprochenen f\"unf Familien
elementarer reeller Funktionen, sowie f\"ur (u.a.\ aus diesen
nach diversen M\"oglichkeiten) zusammengesetze Funktionen, gelten
folgende Differenziationsregeln:

\medskip
\noindent
{\bf Differenziationsregeln}
%f\"ur reelle Funktionen einer Variablen}
%
\begin{enumerate}
\item $(c)^{\prime} = 0$ f\"ur $c = \text{konstant} \in \mathbb{R}$
\hfill ({\bf Konstanten})
\item $(x)^{\prime} = 1$ \hfill ({\bf Lineare Funktion})
\item $(x^{n})^{\prime} = nx^{n-1}$ f\"ur $n \in \mathbb{N}$
\hfill ({\bf Nat\"urliche Potenzfunktionen})
\item $(x^{\alpha})^{\prime} = \alpha x^{\alpha-1}$ f\"ur
$\alpha \in \mathbb{R}$ und $x \in \mathbb{R}_{> 0}$
\hfill ({\bf Allgemeine Potenzfunktionen})
\item $(a^{x})^{\prime} = \ln(a)a^{x}$ f\"ur
$a \in \mathbb{R}_{> 0}\backslash\{1\}$
\hfill ({\bf Exponentialfunktionen})
\item $(e^{ax})^{\prime} = ae^{ax}$ f\"ur $a \in \mathbb{R}$
\hfill ({\bf Nat\"urliche Exponentialfunktionen})
\item $\displaystyle (\log_{a}(x))^{\prime} = \frac{1}{x\ln(a)}$
f\"ur $a \in \mathbb{R}_{> 0}\backslash\{1\}$ und $x \in
\mathbb{R}_{> 0}$ \hfill ({\bf Logarithmusfunktionen})
\item $(\displaystyle \ln(x))^{\prime} = \frac{1}{x}$
f\"ur $x \in \mathbb{R}_{> 0}$
\hfill ({\bf Nat\"urliche Logarithmusfunktion}).
\end{enumerate}
%
F\"ur differenzierbare reelle Funktionen $f$ und $g$ 
gilt:\footnote{(Engl.: summation rule, product rule, quotient 
rule, chain rule, logarithmic differentiation, differentiation of 
an inverse function.)}
%
\begin{enumerate}
\item $(cf(x))^{\prime} = cf^{\prime}(x)$ f\"ur
$c = \text{konstant} \in \mathbb{R}$
\item $(f(x) \pm g(x))^{\prime} = f^{\prime}(x) \pm g^{\prime}(x)$
\hfill ({\bf Summenregel})
\item $(f(x)g(x))^{\prime} = f^{\prime}(x)g(x) + f(x)g^{\prime}(x)$
\hfill ({\bf Produktregel})
\item $\displaystyle \left(\frac{f(x)}{g(x)}\right)^{\prime}
= \frac{f^{\prime}(x)g(x) - f(x)g^{\prime}(x)}{(g(x))^{2}}$
\hfill ({\bf Quotientenregel})
\item $((f \circ g)(x))^{\prime}
%= (f(g(x)))^{\prime}
= \left.f^{\prime}(g)\right|_{g=g(x)}\cdot
g^{\prime}(x)$ \hfill ({\bf Kettenregel})
\item $\displaystyle (\ln(f(x)))^{\prime}
= \frac{f^{\prime}(x)}{f(x)}$ f\"ur $f(x) > 0$
\hfill ({\bf Logarithmisches Differenzieren})
\item $\displaystyle (f^{-1}(x))^{\prime}
= \left.\frac{1}{f^{\prime}(y)}
\right|_{y=f^{-1}(x)}$, falls $f$ eineindeutig ist.

\hfill ({\bf Differenzieren einer inversen Funktion}).
\end{enumerate}
%
Die Methoden der Differenzialrechnung wollen wir nun exemplarisch
zur Beschreibung des \"Anderungsverhaltens einiger in
\"okonomischen Anwendungen h\"aufig wiederkehrender einfacher
Funktionen heranziehen, sowie zur Bestimmung von m\"oglichen
Extremwerten dieser. Die im folgenden Abschnitt bereit gestellte
Liste bietet einen kurzen \"Uberblick \"uber verschiedene
g\"angige Beispiele.

%%%%%%%%%%%%%%%%%%%%%%%%%%%%%%%%%%%%%%%%%%%%%%%%%%%%%%%%%%%%%%%%%%%
\section[\"Okonomische Funktionen der Wirtschaftstheorie]%
{\"Okonomische Funktionen der Wirtschaftstheorie}
\lb{sec:oekfkt}
%%%%%%%%%%%%%%%%%%%%%%%%%%%%%%%%%%%%%%%%%%%%%%%%%%%%%%%%%%%%%%%%%%%
%
\begin{enumerate}

\item {\bf Kostenfunktion} (engl.: total cost function) $K(x) \geq 
0$ \hfill (dim: GE) \\
Argument: Ausbringungsmenge (engl.: level of physical output) $x 
\geq 0$ (dim: ME)

\item {\bf Grenzkostenfunktion} (engl.: marginal cost function) 
$K^{\prime}(x) > 0$ \hfill (dim: GE/ME) \\
Argument: Ausbringungsmenge $x \geq 0$ (dim: ME)

\item {\bf Durchschnittskostenfunktion} (engl.: average cost 
function) $K(x)/x > 0$ \hfill (dim: GE/ME) \\
Argument: Ausbringungsmenge $x > 0$ (dim: ME)

\item {\bf St\"{u}ckpreisfunktion} (engl.: unit price function) 
$p(x) \geq 0$ \hfill (dim: GE/ME) \\
Argument: Ausbringungsmenge $x > 0$ (dim: ME)

\item {\bf Ertragsfunktion} (engl.: total revenue function) $E(x) 
:= xp(x) \geq 0$ \hfill (dim: GE) \\
Argument: Ausbringungsmenge $x > 0$ (dim: ME)

\item {\bf Grenzertragsfunktion} (engl.: marginal revenue function)
$E^{\prime}(x) = xp^{\prime}(x)+p(x)$
\hfill (dim: GE/ME) \\
Argument: Ausbringungsmenge $x > 0$ (dim: ME)

\item {\bf Gewinnfunktion} (engl.: profit function)
$G(x) := E(x)-K(x)$
\hfill (dim: GE) \\
Argument: Ausbringungsmenge $x > 0$ (dim: ME)

\item {\bf Grenzgewinnfunktion} (engl.: marginal profit function)
$G^{\prime}(x) := E^{\prime}(x)-K^{\prime}(x)
= xp^{\prime}(x)+p(x)-K^{\prime}(x)$
\hfill (dim: GE/ME) \\
Argument: Ausbringungsmenge $x > 0$ (dim: ME)

\item {\bf Nutzenfunktion}\footnote{Nutzenfunktionen werden in der 
Wirtschaftstheorie nicht notwendigerweise als differenzierbar 
vorausgesetzt.} (engl.: utility function) $U(x)$ \hfill (dim: 
fallabh\"angig) \\
Argument: Wohlstand, Handlungsm\"oglichkeit (engl.: wealth, 
opportunity, action) $x$ (dim: fallabh\"angig)

Das Konzept einer Nutzenfunktion zur Quantifizierung des 
psychologischen Werts, den ein Wirtschaftsakteur einer bestimmten 
Geldsumme oder dem Besitz eines bestimmten Guts zuschreibt, wurde 
von dem Schweizerischen Mathematiker und Physiker 
\href{http://www-history.mcs.st-and.ac.uk/Biographies/Bernoulli_Daniel.html}{Daniel Bernoulli (1700--1782)} 
im Jahre 1738 in die {\bf Wirtschaftstheorie} (engl: economic 
theory) eingef\"uhrt; vgl. Bernoulli (1738)~\ct{ber1738}. Die 
Existenz einer Nutzenfunktion geh\"ort zu den Grundanahmen der 
Theorie und wird allgemein als rechtsgekr\"ummte Funktion 
dargestellt, $U^{\prime\prime}(x) < 0$, vor dem Hintergrund der 
\emph{Annahme} abnehmenden Grenznutzens mit steigendem Wohlstand.

\item {\bf Wirtschaftlichkeit} (engl.: economic efficiency)
$W(x) := E(x)/K(x) \geq 0$
\hfill (dim: 1) \\
Argument: Ausbringungsmenge $x > 0$ (dim: ME)

\item {\bf Nachfragefunktion} (engl.: demand function) $N(p) \geq 
0$, monoton fallend \hfill (dim: ME) \\
Argument: Preis pro ME $p$, ($0 \leq p \leq p_{\rm max}$)
(dim: GE/ME)

\item {\bf Angebotsfunktion} (engl.: supply function) $A(p) \geq 
0$, monoton steigend \hfill (dim: ME) \\
Argument: Preis pro ME $p$, ($p_{\rm min} \leq p$)
(dim: GE/ME).

\end{enumerate}
%

\medskip
\noindent
Ein besonders prominentes Beispiel einer reellen 
\"{o}konomischen Funktion einer Variablen stellt die 
{\bf psychologische Wertfunktion} (engl.: psychological value 
function) dar, welche die Israelisch--US-Amerikanischen 
experimentellen Psychologen Daniel Kahneman und Amos Tversky 
(1937--1996) im Rahmen ihrer {\bf Neuen Erwartungstheorie} (engl.: 
Prospect Theory), einem Fundament der {\bf Verhaltens\"{o}konomik} 
(engl.: Behavioural Economics), entwickelt haben (vgl. Kahneman 
und Tversky (1979)~\ct[S.~279]{kahtve1979}, und Kahneman 
(2011)~\ct[S.~282f]{kah2011}). Geehrt wurden die Untersuchungen 
dieser Autoren im Jahre 2002 mit dem
\href{http://www.nobelprize.org/nobel_prizes/economics/laureates/2002/}{Preis f\"ur Wirtschaftswissenschaften im Gedenken an Alfred 
Nobel}. Eine m\"{o}gliche Representation dieser psychologischen 
Wertfunktion wird durch abschnittsweise Beschreibung
%
\be
\lb{psychvaluefct}
v(x)=
\begin{cases}
a\log_{10}\left(1+x\right) & \text{f\"{u}r}\quad
x \in \mathbb{R}_{\geq 0} \\
\\
-2a\log_{10}\left(1-x\right) & \text{f\"{u}r}\quad
x \in \mathbb{R}_{< 0}
\end{cases} \ ,
\ee
%
mit Parameter $a \in \mathbb{R}_{>0}$, gegeben. Im Kontrast zu 
Bernoullis Nutzenfunktion, stellt hierbei das Argument~$x$ die 
\emph{\"{A}nderung einer quantitativen Wohlstandsgr\"{o}\ss e} 
bez\"{u}glich eines vorgegebenen Referenzpunktes dar (anstatt die 
quantitative Wohlstandsgr\"{o}\ss e selbst).

%%%%%%%%%%%%%%%%%%%%%%%%%%%%%%%%%%%%%%%%%%%%%%%%%%%%%%%%%%%%%%%%%%%
\section{Kurvendiskussion}
\lb{sec:kurvdisk}
%%%%%%%%%%%%%%%%%%%%%%%%%%%%%%%%%%%%%%%%%%%%%%%%%%%%%%%%%%%%%%%%%%%
Bevor wir zur Anwendung der Differenzialrechnung auf einige
einfache quantitative Probleme aus den Wirtschaftswissenschaften
schreiten, fassen wir in diesem Abschnitt kurz die
grundlegenden Schritte einer umfassenden Kurvendiskussion (engl.: 
curve sketching) f\"ur
reelle Funktionen einer Variablen zusammen. Wesentlicher Aspekte
einer Kurvendiskussion bedienen wir uns sodann in drei
nachfolgenden Betrachtungen \"okonomischer Art.
%
\begin{enumerate}

\item {\bf Definitionsbereich} (engl.: domain): $D(f)=\{x \in 
\mathbb{R}|f(x) \ \text{ist regul\"ar}\}$

\item {\bf Symmetrien} (engl.: symmetries): f\"ur alle $x \in D(f)$, ist
\begin{itemize}
\item[(i)] $f(-x) = f(x)$, d.h. $f$ {\bf gerade} (engl.: even), 
oder
\item[(ii)] $f(-x) = -f(x)$, d.h. $f$ {\bf ungerade} (engl.: odd), 
oder
\item[(iii)] $f(-x) \neq f(x) \neq -f(x)$, d.h. $f$ hat {\bf keine
Symmetrie}?
\end{itemize}

\item {\bf Nullstellen} (engl.: roots): alle $x_{N} \in D(f)$, die 
die Bedingung $f(x) \stackrel{!}{=} 0$ erf\"ullen.

\item {\bf Lokale Extremwerte} (engl.: local extremal values):
\begin{itemize}
\item[(i)] lokale {\bf Minima} (engl.: local minima) von $f$ liegen vor bei allen $x_{E} \in D(f)$, die die

notwendige Bedingung $f^{\prime}(x) \stackrel{!}{=} 0$, und die

hinreichende Bedingung $f^{\prime\prime}(x) \stackrel{!}{>} 0$
simultan erf\"ullen.

\item[(ii)] lokale {\bf Maxima} (engl.: local maxima) von $f$ 
liegen vor bei allen $x_{E} \in D(f)$, die die

notwendige Bedingung $f^{\prime}(x) \stackrel{!}{=} 0$, und die

hinreichende Bedingung $f^{\prime\prime}(x) \stackrel{!}{<} 0$
simultan erf\"ullen.
\end{itemize}

\item {\bf Wendepunkte} (engl.: points of inflection): alle $x_{W} 
\in D(f)$, die die

notwendige Bedingung $f^{\prime\prime}(x) \stackrel{!}{=} 0$, und 
die

hinreichende Bedingung $f^{\prime\prime\prime}(x) 
\stackrel{!}{\neq} 0$
simultan erf\"ullen.

\item {\bf Monotonie} (engl.: monotonous behaviour):
\begin{itemize}
\item[(i)] $f$ monoton {\bf steigend} (engl.: monotonously 
increasing) f\"ur $x \in D(f)$ mit $f^{\prime}(x) > 0$
\item[(ii)] $f$ monoton {\bf fallend} (engl.: monotonously 
decreasing) f\"ur $x \in D(f)$ mit $f^{\prime}(x) < 0$
\end{itemize}

\item {\bf Lokale Kr\"ummung} (engl.: local curvature):
\begin{itemize}
\item[(i)] $f$ {\bf linksgekr\"ummt} (engl.: left-handedly curved) 
f\"ur $x \in D(f)$ mit $f^{\prime\prime}(x) > 0$
\item[(ii)] $f$ {\bf rechtsgekr\"ummt} (engl.: right-handedly 
curved) f\"ur $x \in D(f)$ mit $f^{\prime\prime}(x) < 0$
\end{itemize}

\item {\bf Asymptoten} (engl.: asymptotic behaviour):
\begin{itemize}
\item[(i)] Geraden $y=ax+b$ mit den Eigenschaften
$\lim_{x \to +\infty}[f(x)-ax-b]=0$
oder $\lim_{x \to -\infty}[f(x)-ax-b]=0$
\item[(ii)] Geraden $x=x_{0}$ an Polstellen
$x_{0} \notin D(f)$
\end{itemize}

\item {\bf Wertebereich} (engl.: range): $W(f)=\{y \in 
\mathbb{R}|y=f(x)\}$.

\end{enumerate}
%

%%%%%%%%%%%%%%%%%%%%%%%%%%%%%%%%%%%%%%%%%%%%%%%%%%%%%%%%%%%%%%%%%%%
\section[Analytische Untersuchungen \"okonomischer Funktionen]%
{Analytische Untersuchungen \"okonomischer Funktionen}
\lb{sec:extroekfkt}
%%%%%%%%%%%%%%%%%%%%%%%%%%%%%%%%%%%%%%%%%%%%%%%%%%%%%%%%%%%%%%%%%%%
%------------------------------------------------------------------
\subsection{Kostenfunktionen nach Turgot und von Th\"unen}
%------------------------------------------------------------------
Nach dem wirtschaftstheoretischen {\bf Ertragsgesetz} (engl.: law 
of diminishing returns), welches auf den franz\"osischen 
Staatsmann und \"Okonomen 
\href{http://en.wikipedia.org/wiki/Anne-Robert-Jacques_Turgot,_Baron_de_Laune}{Anne
Robert Jacques Turgot (1727--1781)} sowie den deutschen Agrar- und
Wirtschaftswissenschaftler
\href{http://en.wikipedia.org/wiki/Johann_Heinrich_von_Th�nen}{Johann
Heinrich von Th\"unen (1783--1850)} zur\"uckgeht, wird f\"ur eine
nichtnegative {\bf Kostenfunktion} (engl.: total cost function) 
$K(x)$ (in GE) mit
{\bf Ausbringungsmenge} (engl.: level of physical output) $x \geq 
0~\text{ME}$ als quantitative
Abbildungsvorschrift die mathematische Form eines speziellen
{\em Polynoms dritten Grades\/} [vgl.\ Gl.~(\ref{npol})]
angenommen, welches durch
%
\be
\lb{totalcostfct}
\fbox{$\displaystyle\begin{array}{c}
K(x) = \underbrace{a_{3}x^{3}
+ a_{2}x^{2} + a_{1}x}_{=K_{v}(x)}
+ \underbrace{a_{0}}_{=K_{f}} \\[10mm]
\text{mit}\ a_{3},a_{1} > 0,\, a_{2} < 0,
\, a_{0} \geq 0,
\, a_{2}^{2}-3a_{3}a_{1} < 0
\end{array}
$}
\ee
%
beschrieben wird und somit vier freie Parameter besitzt.
Es handelt sich hierbei um ein konkretes Beispiel einer
empirisch motivierten, aus einer {\bf Regressionsanalyse}
hervorgegangenen {\bf Modellierung} eines
{\bf nichtlinearen} (engl.: non-linear) funktionalen Zusammenhangs 
zwischen zwei gegebenen \"okonomischen Gr\"o\ss en: hier konkret
der Ausbringungsmenge $x$ eines Produktes einerseits,
und den daran gekoppelten Produktionskosten $K(x)$
andererseits. Die f\"ur $K(x)$ angenommene funktionale Form
ist aus der Praxis abgeleitet; sie soll folgende empirisch
beobachtete Eigenschaften der Produktionskosten reflektieren:
f\"ur Ausbringungsmengen $x \geq 0~\text{ME}$ liegt streng monoton
steigendes Verhalten vor, insbesondere gibt es {\em keine\/}
Nullstellen, {\em keine\/} lokalen Extremwerte,\footnote{Die
letzte Bedingung in Gl.~(\ref{totalcostfct}) gew\"ahrleistet
eine erste Ableitung von $K(x)$, welche {\em keine\/} reellen
Nullstellen zul\"asst; vgl. $0\stackrel{!}{=}ax^{2}+bx+c$ mit
$b^{2}-4ac<0$.} jedoch {\em genau eine\/} Wendestelle. Als
Kurve betrachtet stellt $K(x)$ in dieser speziellen Form
einen seitenverkehrten, langestreckten Gro\ss buchstaben
"`S"' dar: ausgehend von einem Fixkostenwert, steigen die
Produktionskosten zun\"achst degressiv bis zu einem
Wendepunkt, um danach progressiv weiterzuwachsen.

\medskip
\noindent
Allgemein setzt sich eine
Kostenfunktion in Abh\"angigkeit einer Ausbringungsmenge
aus {\bf variablen Kosten} (engl.: variable costs) $K_{v}(x)$ und 
{\bf fixen Kosten} (engl.: fixed costs)
$K_{f}=a_{0}$ zusammen. Diese Tatsache bringen wir durch
den einfachen Ansatz
%
\be
K(x)=K_{v}(x)+K_{f}
\ee
%
zum Ausdruck.

\medskip
\noindent
Ertragsgesetzliche Kostenfuntionen unterteilt man in der
Betriebswirtschaftslehre in vier Phasen, deren Grenzen durch
besonders ausgezeichnete Werte der Ausbringungsmenge
festgelegt werden: 
%
\begin{itemize}

\item {\bf Phase I} ($0~\text{ME} \leq x \leq x_{W}$):

Die Kostenfunktion $K(x)$ besitzt f\"ur
die Ausbringungsmenge $x_{W} = -a_{2}/(3a_{3}) > 0~\text{ME}$
einen {\bf Wendepunkt}. F\"ur kleinere Werte von $x$
ist $K^{\prime\prime}(x) < 0~\text{GE}/\text{ME}^{2}$, d.h.
$K(x)$ w\"achst hier degressiv, f\"ur gr\"o\ss ere
$x$ ist $K^{\prime\prime}(x) > 0~\text{GE}/\text{ME}^{2}$,
d.h. $K(x)$ w\"achst hier progressiv. Die {\bf Grenzkosten} 
(engl.: marginal costs)
%
\be
K^{\prime}(x) = 3a_{3}x^{2}+2a_{2}x+a_{1}
> 0~\text{GE}/\text{ME} \ \text{f\"ur\ alle} \ x \geq 0~\text{ME}
\ee
%
erreichen bei der Ausbringungsmenge
$x_{W} = -a_{2}/(3a_{3})$ ein %(lokales)
{\bf Minimum}.

\item {\bf Phase II} ($x_{W} < x \leq x_{g_{1}}$):

Die {\bf variablen Durchschnittskosten} (bzw.\ variablen
St\"uckkosten) (engl.: variable average costs)
%
\be
\frac{K_{v}(x)}{x} = a_{3}x^{2} + a_{2}x + a_{1}
\ ,\quad x > 0~\text{ME}
\ee
%
werden {\bf minimal} f\"ur eine Ausbringungsmenge
$x_{g_{1}}=-a_{2}/(2a_{3}) > 0~\text{ME}$. Bei diesem Wert
von $x$ liegt \emph{Gleichheit der variablen Durchschnittskosten 
und der Grenzkosten} vor, also
%
\be
\lb{betrmin}
\frac{K_{v}(x)}{x} = K^{\prime}(x) \ ,
\ee
%
was \"uber die Quotientenregel des Ableitens aus der
notwendigen Extremalbedingung
%
\[
0 \stackrel{!}{=} \left(\frac{K_{v}(x)}{x}\right)^{\prime}
= \frac{(K(x)-K_{f})^{\prime}\cdot x - K_{v}(x)\cdot 1}{x^{2}}
\ ,
\]
%
mit $K_{f}^{\prime}=0~\text{GE/ME}$, folgt. Man spricht
vom Wertepaar $(x_{g_{1}},K(x_{g_{1}}))$ als
dem {\bf Betriebsminimum}. F\"{u}r die Tangente an $K(x)$
in diesem Punkt [vgl. Gl.~(\ref{ftangente})] gilt unter
Verwendung der Eigenschaft (\ref{betrmin})
%
\[
T(x) = K(x_{g_{1}}) + K^{\prime}(x_{g_{1}})(x-x_{g_{1}})
= K_{v}(x_{g_{1}}) + K_{f} + \frac{K_{v}(x_{g_{1}})}{x_{g_{1}}}\,
(x-x_{g_{1}})
= K_{f} + \frac{K_{v}(x_{g_{1}})}{x_{g_{1}}}\,x \ ,
\]
%
d.h. sie schneidet die $K$-Achse an der Stelle $K_{f}$.

\item {\bf Phase III} ($x_{g_{1}} < x \leq x_{g_{2}}$):

Die {\bf Durchschnittskosten} (bzw.\ St\"uckkosten) (engl.: 
average costs)
%
\be
\frac{K(x)}{x} = a_{3}x^{2} + a_{2}x + a_{1}
+ \frac{a_{0}}{x} \ ,\quad x > 0~\text{ME}
\ee
%
werden {\bf minimal} f\"ur eine Ausbringungsmenge
$x_{g_{2}} > 0~\text{ME}$, welche die Bedingungsgleichung
$0~\text{GE}\stackrel{!}{=}2a_{3}x_{g_{2}}^{3}+a_{2}x_{g_{2}}^{2}-a_{0}$
erf\"ullt. Bei diesem Wert von $x$ liegt {\em Gleichheit der
Durchschnittskosten und der Grenzkosten\/} vor, also
%
\be
\lb{betropt1}
\frac{K(x)}{x} = K^{\prime}(x) \ ,
\ee
%
was \"uber die Quotientenregel des Ableitens aus der
notwendigen Extremalbedingung
%
\[
0 \stackrel{!}{=} \left(\frac{K(x)}{x}\right)^{\prime}
= \frac{K^{\prime}(x)\cdot x - K(x)\cdot 1}{x^{2}}
\]
%
folgt. Alternativ findet man aus dieser Bedingung durch Umstellen
die Beziehung
%
\be
\lb{betropt2}
\frac{K^{\prime}(x)}{K(x)/x}
=x\,\frac{K^{\prime}(x)}{K(x)}=1
\quad\text{f\"ur}\quad x=x_{g_{2}} \ .
\ee
%
Man spricht vom Wertepaar $(x_{g_{2}},K(x_{g_{2}}))$ als dem
{\bf Betriebsoptimum} (engl.: minimum efficient scale (MES)). 
Betriebswirtschaftlich betrachtet ist f\"ur eine Ausbringungsmenge 
$x=x_{g_{2}}$ das (bez\"uglich der in der Einleitung genannten 
Gr\"o\ss e inverse) Verh\"altnis "`INPUT zu OUTPUT"', also 
$K(x)/x$, am g\"unstigsten. F\"{u}r die Tangente an $K(x)$
in diesem Punkt [vgl. Gl~(\ref{ftangente})] gilt unter
Verwendung der Eigenschaft (\ref{betropt1})
%
\[
T(x) = K(x_{g_{2}}) + K^{\prime}(x_{g_{2}})(x-x_{g_{2}})
= K(x_{g_{2}}) + \frac{K(x_{g_{2}})}{x_{g_{2}}}\,(x-x_{g_{2}})
= \frac{K(x_{g_{2}})}{x_{g_{2}}}\,x \ ,
\]
%
d.h. sie schneidet die $K$-Achse an der Stelle $0~\text{GE}$.

\item {\bf Phase IV} ($x > x_{g_{2}}$):

Es gilt $K^{\prime}(x)/K(x)/x>1$; die mit der Fertigung einer
zus\"atzlichen ME des betrachteten Produkts verbundenen Kosten,
$K^{\prime}(x)$, fallen jetzt h\"oher aus als die
Durchschnittskosten des Produkts, $K(x)/x$. Diese Situation
ist betriebswirtschaftlich betrachtet ung\"unstig.
\end{itemize}
%

%------------------------------------------------------------------
\subsection{Ertragsgesetzliche Gewinnfunktionen}
%------------------------------------------------------------------
In diesem Abschnitt seien unsere \"Uberlegungen auf die
Preispolitik im {\bf Monopol} (engl.: monopoly) beschr\"ankt. 
Dar\"uber hinaus
wollen wir die Annahme der G\"ultigkeit eines {\bf \"okonomischen
Gleichgewichtes} (engl.: economic equilibrium) machen, dass sich 
also das Angebot/der Absatz
bzgl.\ eines Produktes und die Nachfrage bzgl.\ desselben
Produktes mengenm\"a\ss ig entsprechen; als Formel
%
\be
x(p) = N(p)
\ee
%
geschrieben. Deshalb identifizieren wir im Folgenden $x(p)$
mit einer nichtnegativen {\bf Nachfragefunktion} (engl.: demand 
function) (in ME), die
von einem {\bf St\"uckpreis} (engl.: price per unit) $p$ (in 
GE/ME) abh\"angt, den
ein Monopolist beliebig vorgeben kann. Je nach herangezogenem
Wirtschaftsmodell k\"onnte $x(p)$ beispielsweise linear oder
quadratisch von $p$ abh\"angen. Wichtig ist jedoch, dass wir
$x(p)$ realistischerweise als streng monoton fallende Funktion
annehmen wollen und sie somit auch {\em umkehrbar\/} sei. Die
Nachfragefunktion $x(p)$ wird durch zwei spezielle Werte
gekennzeichnet. Der so genannte {\bf Prohibitivpreis} (engl.: 
prohibitive price) $p_{proh}$
der Nachfrage wird durch die Bedingung $x(p_{proh}) = 0~\text{ME}$
definiert; er gibt also eine Nullstelle von $x(p)$ an. Die so
genannte {\bf S\"attigungsmenge} (engl.: saturation quantity) 
$x_{satt}$ der Nachfrage 
definiert sich durch die Bedingung
$x_{satt}:=x(0~\text{GE/ME})$.\footnote{Prohibitivpreis und
S\"attigungsmenge geben also die Werte der "`$x$-
und $y$-Achsenabschnitte"' einer nichtnegativen
Nachfragefunktion an.}

\medskip
\noindent
Die zur streng monoton fallenden, nichtnegativen Nachfragefunktion
$x(p)$ geh\"orige inverse Funktion ist durch die ebenfalls streng
monoton fallende, nichtnegative {\bf Preis--Absatz--Funktion} 
(engl.: price vs sales function)  $p(x)$
(in GE/ME) gegeben. \"Uber $p(x)$ berechnet sich der {\bf Ertrag} 
(engl.: total revenue) (in GE) eines Monopolisten in 
Abh\"angigkeit einer Ausbringungsmenge nach 
Abschnitt~\ref{sec:oekfkt} zu
%
\be
\lb{eq:ertrag}
E(x) = xp(x) \ .
\ee
%
Unter der Annahme ertragsgesetzlich modellierter Kosten $K(x)$
(in GE) f\"ur die Produktion, etc., hat im Monopol eine
ertragsgesetzliche {\bf Gewinnfunktion} (engl.: profit function) 
(in GE) in Abh\"angigkeit einer Ausbringungsmenge die Form
%
\be
\lb{eq:gewinn}
G(x) = E(x) - K(x)
= \underbrace{x\overbrace{p(x)}^{\text{Preis--Absatz}}}_{
\text{Ertrag}}
- \underbrace{\left[a_{3}x^{3}+a_{2}x^{2}
+a_{1}x+a_{0}\right]}_{\text{Kosten}} \ .
\ee
%
Ihre ersten zwei Ableitungen nach $x$ sind folglich durch
die Ausdr\"ucke
%
\begin{eqnarray}
G^{\prime}(x) = E^{\prime}(x) - K^{\prime}(x)
& = & xp^{\prime}(x)+p(x)
-\left[3a_{3}x^{2}+2a_{2}x+a_{1}\right] \\
%
G^{\prime\prime}(x) = E^{\prime\prime}(x) - K^{\prime\prime}(x)
& = & xp^{\prime\prime}(x)
+2p^{\prime}(x)-\left[6a_{3}x+2a_{2}\right]
\end{eqnarray}
%
gegeben. Mit den in Abschnitt~\ref{sec:kurvdisk} zur
Kurvendiskussion angef\"uhrten Regeln berechnen sich
damit f\"ur $G(x)$ die folgenden charakteristischen Kennwerte:
%\pagebreak
%
\begin{itemize}

\item {\bf Gewinnschwelle} (engl.: break-even point)

$x_{S} > 0~\text{ME}$ als L\"osung von
%
\be
G(x) \stackrel{!}{=} 0~\text{GE} \quad\text{(notwendige Bedingung)}
\ee
%
und
%
\be
G^{\prime}(x) \stackrel{!}{>} 0~\text{GE}/\text{ME}
\quad\text{(hinreichende Bedingung)} \ ,
\ee
%

\item {\bf Gewinngrenze} (engl.: end of profitable zone)

$x_{G} > 0~\text{ME}$ als L\"osung von
%
\be
G(x) \stackrel{!}{=} 0~\text{GE} \quad\text{(notwendige Bedingung)}
\ee
%
und
%
\be
G^{\prime}(x) \stackrel{!}{<} 0~\text{GE}/\text{ME}
\quad\text{(hinreichende Bedingung)} \ ,
\ee
%

\item {\bf Gewinnmaximum} (engl.: maximum profit)

$x_{M} > 0~\text{ME}$ als L\"osung von
%
\be
G^{\prime}(x) \stackrel{!}{=} 0~\text{GE}/\text{ME}
\quad\text{(notwendige Bedingung)}
\ee
%
und
%
\be
G^{\prime\prime}(x) \stackrel{!}{<} 0~\text{GE}/\text{ME}^{2}
\quad\text{(hinreichende Bedingung)} \ .
\ee
%
\end{itemize}
%
Beachte insbesondere die folgende spezielle geometrische
Eigenschaft: im Gewinnmaximum besitzen Ertrags- und
Kostenfunktion grunds\"atzlich immer
zueinander {\em parallele Tangenten\/}. Diese Tatsache folgt
aus der notwendigen Bedingung
%
\be
0~\text{GE}/\text{ME} \stackrel{!}{=} G^{\prime}(x)
= E^{\prime}(x) - K^{\prime}(x)
\qquad\Leftrightarrow\qquad
E^{\prime}(x) \stackrel{!}{=} K^{\prime}(x) \ .
\ee
%

\medskip
\noindent
\underline{\bf GTR:} Die Berechnung von Nullstellen und
lokalen Maxima (bzw.\ Minima) kann f\"ur eine
eingespeicherte Funktion im Modus {\tt CALC} leicht mit den
vorprogrammierten interaktiven Funktionen {\tt zero} und
{\tt maximum} (bzw.\ {\tt minimum}) durchgef\"uhrt werden.

\medskip
\noindent
Von besonderem mathematischen Interesse ist in diesem Zusammenhang
die folgende Betrachtung. Bekannt sei eine Gewinnfunktion $G(x)$
der polynomialen Form
%
\be
G(x) = -a_{3}(x+a)(x-b)(x-c)
= a_{3}\left[-x^{3}+(b+c-a)x^{2}+(ab-bc+ca)x-abc\right] \ ,
\ee
%
mit $a,b,c > 0$, $b < c$ und $x \geq 0~\text{ME}$, der nach Gln. 
(\ref{eq:ertrag}) und (\ref{eq:gewinn}) eine {\em lineare\/} 
Preis--Absatz--Funktion $p(x)$ zugrunde liegt. Dann
\"uberf\"uhren die simultanen {\bf Skalentransformationen} (engl.: 
scale transformation)
%
\be
x \mapsto \lambda x \ , \qquad
G(x) \mapsto \frac{1}{\lambda^{3}}\,G(\lambda x) \ ,
\qquad
\lambda > 0
\ee
%
von unabh\"angiger und abh\"angiger Variable die
Gewinnfunktion $G(x)$ in eine zweite, zu $G(x)$
{\bf selbst\"ahnliche} (engl.: self-similar) Gewinnfunktion
%
\bea
\tilde{G}(x) & = & -a_{3}\left(x+\frac{a}{\lambda}\right)
\left(x-\frac{b}{\lambda}\right)\left(x-\frac{c}{\lambda}\right)
\nonumber \\
& = & a_{3}\left[-x^{3}+\frac{(b+c-a)}{\lambda}\,x^{2}
+\frac{(ab-bc+ca)}{\lambda^{2}}\,x
-\frac{abc}{\lambda^{3}}\right] \ .
\eea
%
Analog l\"asst sich mit polynomialen \"okonomischen 
Funktionen $p(x)$, $E(x)$ und $K(x)$ verfahren.

\medskip
\noindent
Diese Betrachtung abschlie\ss end wenden wir uns kurz dem Begriff
des {\bf Cournotschen Punktes} (engl.: Cournot's point) (benannt 
nach dem franz\"osischen Mathematiker und Wirtschaftstheoretiker
\href{http://turnbull.mcs.st-and.ac.uk/history/Biographies/Cournot.html}{Antoine--Augustin
Cournot, 1801--1877}) zu. Man versteht hierunter ganz einfach die
gewinnmaximale Mengen--\-Preis--\-Wertekom\-bi\-na\-tion
$(x_{M}, p(x_{M}))$ der Preis--Absatz--Funktion $p(x)$ im Monopol.
F\"ur diese Wertekombination gilt die
{\bf Amoroso--Robinson--Relation} (engl.: Amoroso--Robinson 
formula) (\href{http://en.wikipedia.org/wiki/Luigi_Amoroso}{Luigi 
Amoroso, 1886--1965}, italienischer Mathematiker und
Wirtschaftswissenschaftler;
\href{http://en.wikipedia.org/wiki/Joan_Robinson}{Joan Violet
Robinson, 1903--1983}, britische \"Okonomin)
%
\be
p(x_{M}) = \frac{K^{\prime}(x_{M})}{1+\varepsilon_{p}(x_{M})} \ ,
\ee
%
mit $K^{\prime}(x_{M})$ dem Wert der Grenzkosten bei $x_{M}$,
und $\varepsilon_{p}(x_{M})$ dem Wert der {\bf Elastizit\"at}
(engl.: elasticity)
(siehe den nachfolgenden Abschnitt~\ref{sec:elast})
der Preis--Absatz--Funktion bei $x_{M}$.
Ausgehend von $E(x)=xp(x)$, erfolgt ihre Herleitung durch
Umstellen aus dem gewinnmaximalen Grenzertag
%
\[
E^{\prime}(x_{M}) = p(x_{M}) + x_{M}p^{\prime}(x_{M})
= p(x_{M})\left[1
+ x_{M}\,\frac{p^{\prime}(x_{M})}{p(x_{M})}\right]
\overbrace{=}^{\text{Abschn.~\ref{sec:elast}}} p(x_{M})\left[1
+ \varepsilon_{p}(x_{M})\right] \ ,
\]
%
unter Verwendung der Extremaleigenschaft $E^{\prime}(x_{M})
=K^{\prime}(x_{M})$.

\medskip
\noindent
\underline{\bf Bem.:} Bei einer Marktsituation mit {\bf totaler
Konkurrenz} (engl.: perfect competition) wird davon ausgegangen, 
dass die Preis--Absatz--Funktion einen vom Markt fest vorgegebenen
{\em konstanten\/} Wert annimmt, also
$p(x)=p=\text{konstant}>0~\text{GE/ME}$ (und deshalb
$p^{\prime}(x)=0~\text{GE/ME}^{2}$) gilt.

%------------------------------------------------------------------
\subsection{Extremwerte von Verh\"altnisgr\"o\ss en}
%------------------------------------------------------------------
Wir wollen nun kurz die Bestimmung von Extremwerten
\"okonomischer Funktionen im Fall von Verh\"altnisgr\"o\ss en im
Sinne der in der Einleitung genannten Konstruktion
%
\[
\frac{\text{OUTPUT}}{\text{INPUT}}
\]
%
besprechen. Die technische Vorgehensweise bleibt dabei nat\"urlich
gleich; lediglich der ben\"otigte Rechenaufwand steigt
m\"oglicherweise an, kann aber gegebenenfalls von einem GTR
\"ubernommen werden.

\medskip
\noindent
Zwei Beispiele von {\bf Maximalwertbestimmungen} wollen wir
betrachten: 
%
\begin{itemize}
\item[(i)]
Wir beginnen mit dem {\bf Durchschnittsgewinn} (engl.: average 
profit)
%
\be
\frac{G(x)}{x}
\ee
%
in Abh\"angigkeit von der Ausbringungsmenge
$x \geq 0~\text{ME}$. Die Bedingungen zur Bestimmung
eines Maximalwertes lauten dann $[G(x)/x]^{\prime}
\stackrel{!}{=} 0~\text{GE/ME}^{2}$ und $[G(x)/x]^{\prime\prime}
\stackrel{!}{<} 0~\text{GE/ME}^{3}$. Unter Beachtung der
Quotientenregel des Ableitens (siehe Abschnitt~\ref{sec:ablt})
liefert die erste Bedingung
%
\be
\frac{G^{\prime}(x)x-G(x)}{x^{2}} = 0~\text{GE/ME}^{2} \ .
\ee
%
Da ein Quotient den Wert Null aber nur annehmen kann, wenn sein
Z\"ahler zu Null wird und gleichzeitig sein Nenner ungleich Null
bleibt, folgt daraus
%
\be
\lb{dgextr2}
G^{\prime}(x)x-G(x) = 0~\text{GE}
\quad\Rightarrow\quad
x\,\frac{G^{\prime}(x)}{G(x)} = 1 \ .
\ee
%
Zu Bestimmen sind Ausbringungsmengen, die die letztgenannte
Bedingung erf\"ullen, und f\"ur welche zus\"atzlich die
zweite Ableitung des Durchschnittsgewinns negativ wird.

\item[(ii)]
Als Ma\ss zahlen zum Vergleich zweier Unternehmen, beispielsweise
ihrer Leistungsf\"ahigkeit bezogen auf einen vorgegebenen Zeitraum,
eignen sich letztendlich nur dimensionslose,
d.h.\ {\em einheitenunabh\"angige\/} Verh\"altnisgr\"o\ss en.
Eine Gr\"o\ss e dieser Art ist die {\bf Wirtschaftlichkeit} 
(engl.: economic efficiency)
%
\be
W(x)=\frac{E(x)}{K(x)} \ ,
\ee
%
gebildet aus den Kosten- und Ertragsfunktionen (in $\text{GE}$)
eines Unternehmens in Abh\"angigkeit von der Ausbringungsmenge
$x \geq 0~\text{ME}$. Analog zu oben lauten hier die Bedingungen
zur Bestimmung eines Maximalwertes $[E(x)/K(x)]^{\prime}
\stackrel{!}{=} 0 \cdot 1/\text{ME}$ und $[E(x)/K(x)]^{\prime\prime}
\stackrel{!}{<} 0 \cdot 1/\text{ME}^{2}$. \"Uber die
Quotientenregel des Ableitens (siehe Abschnitt~\ref{sec:ablt})
ergibt sich aus der ersten Bedingung
%
\be
\frac{E^{\prime}(x)K(x)-E(x)K^{\prime}(x)}{K^{2}(x)}
= 0\cdot 1/\text{ME} \ ,
\ee
%
also f\"ur $K(x) > 0~\text{GE}$
%
\be
\lb{wirtextr1}
E^{\prime}(x)K(x)-E(x)K^{\prime}(x) = 0~\text{GE}^{2}/\text{ME}
\ .
\ee
%
Nach Umstellen und Durchmultiplizieren mit $x > 0~\text{ME}$,
l\"asst sich diese letzte Bedingung auch als
%
\be
\lb{wirtextr2}
x\,\frac{E^{\prime}(x)}{E(x)} = x\,\frac{K^{\prime}(x)}{K(x)}
\ee
%
schreiben. Hintergrund dieser auf den ersten Blick ein wenig
seltsam anmutenden Darstellungsweise [ebenso in
Gl.~(\ref{dgextr2})] sind \"Uberlegungen, auf welche wir
unmittelbar im n\"achsten Abschnitt eingehen
werden. Abschlie\ss end bemerken wir: die Kennzahl
Wirtschaftlichkeit wird maximal f\"ur Ausbringungsmengen, f\"ur
welche Gl.~(\ref{wirtextr2}) erf\"ullt und zus\"atzlich die zweite
Ableitung der Wirtschaftlichkeit negativ ist.
\end{itemize}
%

%%%%%%%%%%%%%%%%%%%%%%%%%%%%%%%%%%%%%%%%%%%%%%%%%%%%%%%%%%%%%%%%%%%
\section[Elastizit\"aten]{Elastizit\"aten}
\lb{sec:elast}
%%%%%%%%%%%%%%%%%%%%%%%%%%%%%%%%%%%%%%%%%%%%%%%%%%%%%%%%%%%%%%%%%%%
Wir wenden uns nun nochmals dem {\bf lokalen \"Anderungsverhalten} 
(engl.: local variability)
von differenzierbaren reellen Funktionen einer Variablen
$f:D\subseteq\mathbb{R} \rightarrow W\subseteq\mathbb{R}$ zu. Aus
Gr\"unden, die wir in K\"urze erl\"autern werden, wollen wir uns
hierbei ausschlie\ss lich auf Bereiche von $f$ mit {\em 
positiven\/} Werten des Argumentes $x$ und {\em positiven\/} 
Funktionswerten $y=f(x)$ beschr\"anken. Wie schon zuvor in 
Abschnitt~\ref{sec:ablt}, gehen wir wieder von einer kleinen 
\"Anderung eines vorgegebenen Wertes des Argumentes~$x$ aus, und 
untersuchen die daraus resultierende Wirkung auf $y=f(x)$:
%
\be
x \stackrel{\Delta x \in \mathbb{R}}{\longrightarrow} x+\Delta x
\qquad \Longrightarrow \qquad
y=f(x) \stackrel{\Delta y \in \mathbb{R}}{\longrightarrow}
y+\Delta y=f(x+\Delta x) \ .
\ee
%

%\pagebreak
\medskip
\noindent
Folgende Begriffe spielen in dieser Betrachtung eine wichtige
Rolle:
%
\begin{itemize}
\item Vorgegebene {\bf absolute \"Anderung} (engl.: absolute 
change) der Variablen $x$:
\qquad $\Delta x$\ ,
\item Resultierende {\bf absolute \"Anderung} der Funktion $f$:
\qquad $\Delta y=f(x+\Delta x)-f(x)$\ ,
\item Vorgegebene {\bf relative \"Anderung} (engl.: relative 
change) der Variablen $x$:
\qquad $\displaystyle \frac{\Delta x}{x}$\ ,
\item Resultierende {\bf relative \"Anderung} der
Funktion $f$:
\qquad $\displaystyle \frac{\Delta y}{y}
=\frac{f(x+\Delta x)-f(x)}{f(x)}$\ .
\end{itemize}
%
{\bf Relative \"Anderungen} nichtnegativer
quantitativer Gr\"o\ss en sind definiert durch die
Berechnungsvorschrift
%
\[
\displaystyle
\frac{\text{Neuer\ Wert}-\text{Alter\ Wert}}{\text{Alter\ 
Wert}} \ ,
\]
%
unter der Vorraussetzung, dass "`$\text{Alter\ Wert}>0$"' gilt. 
Aus dieser Berechnungsvorschrift ergibt sich folglich "`$-1$"' als 
Minimalwert f\"ur eine relative \"Anderung (was einer Reduzierung 
um "`minus 100\%"' entspricht).

\medskip
\noindent
Nun wollen wir die Gr\"o\ss enordnungen der beiden relativen
\"Anderungen $\displaystyle \frac{\Delta x}{x}$ und $\displaystyle 
\frac{\Delta y}{y}$ miteinander vergleichen. Hierzu betrachten wir 
deren Quotienten, d.h. "`resultierende relative \"Anderung von
$f$ zu vorgegebener relativer \"Anderung von $x$"':
%
\[
\frac{\displaystyle\frac{\Delta y}{y}}{\displaystyle\frac{\Delta 
x}{x}}
=\frac{\displaystyle\frac{f(x+\Delta 
x)-f(x)}{f(x)}}{\displaystyle\frac{\Delta x}{x}} \ .
\]
%
Da $f$ als differenzierbar angenommen wurde, k\"onnen wir den
Grenzwert dieses Quotienten f\"ur den Fall immer kleiner
werdender vorgegebener relativer \"Anderungen $\displaystyle 
\frac{\Delta x}{x} \to 0 \Rightarrow \Delta x \to 0$ bei $x > 0$ 
betrachten und definieren:

\medskip
\noindent
\underline{\bf Def.:}
F\"ur eine differenzierbare reelle Funktion $f$ einer Variablen
wird die \emph{einhei\-ten\-unabh\"angige Gr\"o\ss e}
%
\be
\fbox{$\displaystyle
\varepsilon_{f}(x)
:=\lim_{\Delta x\to 0}\frac{\displaystyle\frac{\Delta y}{y}}{
\displaystyle\frac{\Delta x}{x}}
= \lim_{\Delta x\to 0}\frac{\displaystyle\frac{f(x+\Delta 
x)-f(x)}{f(x)}}{\displaystyle\frac{\Delta x}{x}}
= x\,\frac{f^{\prime}(x)}{f(x)}
$}
\ee
%
die {\bf Elastizit\"at} (engl.: elasticity) der Funktion $f$ an 
der Stelle $x$ genannt.

\medskip
\noindent
Die Elastizit\"at gibt die resultierende relative \"Anderung
von $f$ bei vorgegebener infinitesimal kleiner relativer
\"Anderung von $x$, ausgehend von $x>0$, an und ist somit ein
Ma\ss\ f\"ur die {\bf relative lokale \"Anderungsrate} (engl.: 
relative local rate of change) der Funktion $f$ im Punkt 
$(x,f(x))$. Allgemein gilt insbesondere in der Wirtschaftstheorie 
f\"ur die Elastizit\"at $\varepsilon_{f}(x)$ die Interpretation: 
wir das Argument $x$ einer differenzierbaren positiven reellen 
Funktion $f$ um $1~\%$ erh\"oht, so \"andert sich die Funktion $f$ 
in Konsequenz dessen n\"aherungsweise um $\varepsilon_{f}(x)\cdot 
1~\%$.

\medskip
\noindent
%\underline{\bf Bem.:}
In der Fachliteratur dr\"uckt man die
Elastizit\"at von $f$ oft auch mithilfe von logarithmischen
Ableitungen wie folgt aus:
%
$$
\varepsilon_{f}(x)
:= \frac{{\rm d}\ln[f(x)]}{{\rm d}\ln(x)}
\qquad\text{f\"ur}\ x>0 \ \text{und} \ f(x)>0 \ ,
$$
%
denn es gilt mit der Kettenregel der Differenziation
%
$$
\frac{{\rm d}\ln[f(x)]}{{\rm d}\ln(x)}
= \frac{\displaystyle\frac{{\rm 
d}f(x)}{f(x)}}{\displaystyle\frac{{\rm d}x}{x}}
= x\,\frac{\displaystyle\frac{{\rm d}f(x)}{{\rm d}x}}{f(x)}
= x\,\frac{f^{\prime}(x)}{f(x)} \ .
$$
%
In dieser logarithmischen Schreibweise der Regel f\"ur das
Berechnen der Elastizit\"at einer differenzierbaren
reellen Funktion $f$ erschlie\ss t sich direkt, wieso wir zu Beginn
dieser Betrachtung die Einschr\"ankung auf Bereiche von $f$
mit positiven Argumenten und positiven Funktionswerten
vornahmen: der Definitionsbereich einer jeden
Logarithmusfunktion umfasst nur {\em positive\/} reelle
Zahlen.\footnote{Um den Anwendungsbereich der Messgr\"o\ss e
$\varepsilon_{f}$ zu erweitern, k\"onnte man in
verallgemeinernder Weise mit den Betr\"agen $|x|$ und $|f(x)|$
arbeiten. In diesem Fall m\"usste man die vier M\"oglichkeiten
(i) $x>0$, $f(x)>0$, (ii) $x<0$, $f(x)>0$, (iii) $x<0$,
$f(x)<0$ und (iv) $x>0$, $f(x)<0$ getrennt betrachten.}
Ein kurzer Blick auf die in Abschnitt \ref{sec:oekfkt}
bereit gestellte Liste \"okonomischer Funktionen zeigt:
in wirtschaftswissenschaftlichen Anwendungen der
Differenzialrechnung treten h\"aufig (aber nicht
ausschlie\ss lich) Funktionen mit nichtnegativen
Argumenten und nichtnegativen Funktionswerten auf.

%\pagebreak
\medskip
\noindent
F\"ur die in Abschnitt~\ref{sec:fkt} diskutierten
Familien elementarer reeller Funktionen einer Variablen
gilt:

\medskip
\noindent
{\bf Standardelastizit\"aten}
%Alternativ: Liste in Tabellenform

%
\begin{enumerate}
\item $f(x)=x^{n}$ f\"ur $n \in \mathbb{N}$
und $x \in \mathbb{R}_{> 0}
\ \Rightarrow\ \varepsilon_{f}(x)=n$
\hfill ({\bf Nat\"urliche Potenzfunktionen})
\item $f(x)=x^{\alpha}$ f\"ur $\alpha \in \mathbb{R}$
und $x \in \mathbb{R}_{> 0}
\ \Rightarrow\ \varepsilon_{f}(x)=\alpha$
\hfill ({\bf Allgemeine Potenzfunktionen})
\item $f(x)=a^{x}$ f\"ur $a \in \mathbb{R}_{> 0}\backslash\{1\}$
und $x \in \mathbb{R}_{> 0}
\ \Rightarrow\ \varepsilon_{f}(x)=\ln(a)x$
\hfill ({\bf Exponentialfunktionen})
\item $f(x)=e^{ax}$ f\"ur $a \in \mathbb{R}$
und $x \in \mathbb{R}_{> 0}
\ \Rightarrow\ \varepsilon_{f}(x)=ax$
\hfill ({\bf Nat\"urliche Exponentialfunktionen})
\item $f(x)=\log_{a}(x)$ f\"ur $a \in \mathbb{R}_{> 0}
\backslash\{1\}$ und $x \in \mathbb{R}_{> 0}$

\hfill\hfill$\displaystyle \quad\Rightarrow\quad
\varepsilon_{f}(x)=\frac{1}{\ln(a)\log_{a}(x)}$
\hfill ({\bf Logarithmusfunktionen})
\item $f(x)=\ln(x)$ f\"ur $\displaystyle
x \in \mathbb{R}_{> 0}
\quad\Rightarrow\quad \varepsilon_{f}(x)=\frac{1}{\ln(x)}$
\hfill ({\bf Nat\"urliche Logarithmusfunktion}).
\end{enumerate}
%
Beachten Sie die Tatsache, dass f\"ur allgemeine Potenzfunktionen
die Elastizit\"at $\varepsilon_{f}(x)$ eine $x$--unabh\"angige,
{\em konstante\/} Gr\"o\ss e darstellt. {\bf Allgemeine
Potenzfunktionen} werden deshalb auch {\bf skaleninvariant}
(engl.: scale-invariant) genannt. Unter Skaleninvarianz reduzieren 
sich einheitenunabh\"angige Verh\"altnisgr\"o\ss en, also 
Quotienten von Variablen gleicher Dimension, generell zu 
Konstanten. In diesem Zusammenhang ist zu bemerken, dass 
skaleninvariante (fraktale) Potenzgesetze der Form 
$f(x)=Kx^{\alpha}$, mit $K > 0$ und $\alpha \in 
\mathbb{R}_{<0}\backslash \{-2,-1\}$, h\"aufig als 
Wahrscheinlichkeitsverteilungsfunktionen beim mathematischen 
Modellieren von Unsicherheit in den Wirtschafts-
und Sozialwissenschaften zum Einsatz kommen, da sie bei geeigneter
Wahl des Wertespektrums der Variablen $x$ und des Wertes ihrer 
Potenz $\alpha$ zu unbeschr\"ankter Varianz f\"uhren; siehe z.B.
Gleick (1987) \ct[S.~86]{gle1987}, Taleb (2007) 
\ct[S.~326ff]{tal2007}, oder Ref.~\ct[Sec.~8.9]{hve2013}.

\medskip
\noindent
Praktische Anwendungen des Konzepts der Elastizit\"at st\"utzen
sich auf folgende einfache (lineare!) N\"aherungsformel zur
Berechnung der Konsequenzen kleiner prozentualer \"Anderungen
des Argumentes $x$ einer differenzierbaren reellen Funktion $f$:
Ausgehend vom Wert $x_{0}>0$, gilt f\"ur vorgegebene prozentuale
\"Anderungen von $x$ im Werteintervall
$0~\% < {\displaystyle\frac{\Delta x}{x_{0}}} \leq 5~\%$
f\"ur die resultierenden prozentualen \"Anderungen von $f$
n\"aherungsweise die Relation
%
\be
(\text{Prozentuale\ \"Anderung\ von}\ f)
\approx (\text{Elastizit\"at\ von}\ f\ \text{bei}\ x_{0}) \times
(\text{Prozentuale\ \"Anderung\ von}\ x) \ ,
\ee
%
oder als Formel
%
\be
\frac{f(x_{0}+\Delta x)-f(x_{0})}{f(x_{0})}
\approx \varepsilon_{f}(x_{0})
\frac{\Delta x}{x_{0}} \ .
\ee
%

\medskip
\noindent
Nun wenden wir uns noch einer in den Wirtschaftswissenschaften
praktizierten speziellen Sprachregelung zu.
Das relative lokale \"Anderungsverhalten einer Funktion $f$
hei\ss t f\"ur jene $x \in D(f)$
\begin{itemize}
	\item {\bf unelastisch} (engl.: inelastic), f\"ur welche 
	$|\varepsilon_{f}(x)|<1$,
	\item {\bf proportional elastisch} (engl.: unit elastic) , f\"ur 
	welche $|\varepsilon_{f}(x)|=1$,
	\item {\bf elastisch} (engl.: elastic), f\"ur welche 
	$|\varepsilon_{f}(x)|>1$.
\end{itemize}
%\medskip
%\noindent
Beispielsweise weist eine ertragsgesetzliche Kostenfunktion
$K(x)$ im Betriebsoptimum $x=x_{g_{2}}$  die Eigenschaft
$\varepsilon_{K}(x_{g_{2}})=1$ auf [vgl.\ Gl.~(\ref{betropt2})].
Ebenso gilt im Maximum eines Durchschnittsgewinns $G(x)/x$, dass
$\varepsilon_{G}(x)=1$ [vgl.\ Gl.~(\ref{dgextr2})].

%\pagebreak
\medskip
\noindent
Die folgenden Rechenregeln erweisen sich als n\"utzlich bei der 
Berechnung der Elastizit\"aten von zusammengesetzten Funktionen in 
Sinne von Abschnitt~\ref{subsec:kombfkt} hin:

%\pagebreak
\medskip
\noindent
{\bf Rechenregeln f\"ur Elastizit\"aten}

\noindent
Sind $f$ und $g$ differenzierbare reelle Funktionen einer 
Variablen, mit Elastizit\"aten $\varepsilon_{f}$ und 
$\varepsilon_{g}$, so gilt:
%
\begin{enumerate}
	\item {\bf Produkt} $f \cdot g$: \qquad\qquad
	$\varepsilon_{f \cdot g}(x)
	= \varepsilon_{f}(x) + \varepsilon_{g}(x)$,
	\item {\bf Quotient} ${\displaystyle\frac{f}{g}}$, $g \neq 0$: 
	\qquad\qquad 	$\varepsilon_{f/g}(x)
	= \varepsilon_{f}(x) - \varepsilon_{g}(x)$,
	\item {\bf Verkettung} $f \circ g$: \qquad\qquad
	$\varepsilon_{f \circ g}(x)
	= \varepsilon_{f}(g(x))\cdot\varepsilon_{g}(x)$,
	\item {\bf Inverse Funktion} $f^{-1}$: \qquad\qquad
	$\displaystyle \varepsilon_{f^{-1}}(x)
	= \left.\frac{1}{\varepsilon_{f}(y)}\right|_{y=f^{-1}(x)}$.
	%= 1/\varepsilon_{f}(f^{-1}(x))$
\end{enumerate}
%

\medskip
\noindent
Zum Schluss dieses Kapitels wollen wir uns noch mit einer 
m\"oglichen Erweiterung des Elastizit\"atsbegriffs befassen. F\"ur 
positive differenzierbare reellwertige Funktionen~$f$ einer 
positiven reellen Variablen~$x$ l\"asst sich eine zweite 
Elastizit\"at nach der Ableitungsvorschrift
%
\be
\lb{eq:secondelasticity}
\varepsilon_{f}\left[\varepsilon_{f}(x)\right]
:= x\,\frac{{\rm d}}{{\rm d}x}\left[\frac{x}{f(x)}\,\frac{{\rm 
d}f(x)}{{\rm d}x}\right]
\ee
%
definieren. Ohne Schwierigkeiten l\"ast sich diese Vorgehensweise 
auch noch auf h\"ohere Ableitungen von~$f$ verallgemeinern.

%%%%%%%%%%%%%%%%%%%%%%%%%%%%%%%%%%%%%%%%%%%%%%%%%%%%%%%%%%%%%%%%%%%
%%%%%%%%%%%%%%%%%%%%%%%%%%%%%%%%%%%%%%%%%%%%%%%%%%%%%%%%%%%%%%%%%%%
