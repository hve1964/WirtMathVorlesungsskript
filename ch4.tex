%%%%%%%%%%%%%%%%%%%%%%%%%%%%%%%%%%%%%%%%%%%%%%%%%%%%%%%%%%%%%%%%%%%
%  File name: ch4.tex
%  Title:
%  Version: 10.09.2015 (hve)
%%%%%%%%%%%%%%%%%%%%%%%%%%%%%%%%%%%%%%%%%%%%%%%%%%%%%%%%%%%%%%%%%%%
%%%%%%%%%%%%%%%%%%%%%%%%%%%%%%%%%%%%%%%%%%%%%%%%%%%%%%%%%%%%%%%%%%%
\chapter[Leontiefsches 
Input--Output--Matrizenmodell]{Station\"{a}res Leontiefsches
Input--Output--Matrizenmodell}
\lb{ch4}
%%%%%%%%%%%%%%%%%%%%%%%%%%%%%%%%%%%%%%%%%%%%%%%%%%%%%%%%%%%%%%%%%%%
\hfill\hbox{\fbox{\vbox{\hsize=10cm
Der Inhalt dieses Kapitels ist relevant f\"ur das
Verst\"andnis der in Kapitel 5.2 des Lehrbuchs von Schrey\"ogg
und Koch (2010)~\ct{schkoc2010} behandelten quantitativen Themen.
%Gleichfalls f\"ur Kapitel 10.5 des Lehrbuchs von Schmalen und
%Pechtl 2006~\ct{schpec2006}.
}}}

\vspace{10mm}
\noindent
Dieses Kapitel befasst sich mit einem Teilbereich der
Wirtschaftstheorie, in welchem einfache, in den vorangegangenen
Kapiteln besprochene Methoden der linearen
Algebra herangezogen werden, um Prozesse des G\"uteraustauschs
zwischen wirtschaftstreibenden Akteuren m\"oglichst realit\"atsnah
quantitativ zu modellieren.

%%%%%%%%%%%%%%%%%%%%%%%%%%%%%%%%%%%%%%%%%%%%%%%%%%%%%%%%%%%%%%%%%%%
\section[Allgemeines]%
{Allgemeines}
\lb{sec:leongrund}
%%%%%%%%%%%%%%%%%%%%%%%%%%%%%%%%%%%%%%%%%%%%%%%%%%%%%%%%%%%%%%%%%%%
Das zu betrachtende Modell (engl.: Leontief's stationary 
input--output matrix model) dient der quantitativen Beschreibung 
von G\"uterstr\"omen in
%einer Volkswirtschaft
einem wirtschaftlichen System mit
$\boldsymbol{n}$~{\bf interdependenten Produktionssektoren} 
w\"ahrend eines vorgegebenen Zeitraums. Es
wurde in seinen Grundlagen von dem russischen
Wirtschaftswissenschaftler
\href{http://en.wikipedia.org/wiki/Leontief}{Wassily Wassiljewitsch
Leontief (1905--1999)} entwickelt, vgl.~Leontief 
(1936)~\ct{leo1936}, der, unter anderem f\"ur diese Leistung, im 
Jahre 1973 mit dem
\href{http://www.nobelprize.org/nobel_prizes/economics/laureates/1973/}{Preis f\"ur Wirtschaftswissenschaften im Gedenken an Alfred 
Nobel} ausgezeichnet wurde.

\medskip
\noindent
Der Einfachheit halber wegen stelle jeder der $n$, der Annahme nach
interdependenten (d.h., %(zu Teilen)
aufeinander angewiesenen) Produktionssektoren genau {\em ein\/}
Gut her; die im betrachteten
Zeitraum brutto produzierten und auch ausgelieferten Mengen
dieser $n$~G\"uter sind die {\bf OUTPUT--Gr\"o\ss en} des Modells. 
Die {\bf INPUT--Gr\"o\ss en} des betrachteten Produktionsmodells 
gliedern sich in {\em zwei\/} Bereiche. Einerseits gibt es so 
genannte {\bf exogene} INPUT--Gr\"o\ss en; darunter verstehen wir 
hier die Mengen von $m$~verschiedenen {\bf Rohstoffen}, aus 
welchen, in unterschiedlichen Mischungsverh\"altnissen, die 
$n$~G\"uter gefertigt werden. Andererseits gibt es so genannte 
{\bf endogene} INPUT--Gr\"o\ss en; dies sind hier Mengen der 
{\bf G\"uter der Nachbarproduzenten}, die ein einzelner Produzent 
zus\"atzlich (neben den Rohstoffen), im Sinne von Zulieferungen, 
zur Fertigung seines eigenen Gutes  ben\"otigt. Die 
OUTPUT--Gr\"o\ss en des
Modells, also die im betrachteten Zeitraum produzierten und
ausgelieferten Mengen der $n$~verschiedenen G\"uter,
flie\ss en durch {\em zwei\/} Kan\"ale ab: (i) zu Teilen durch
einen {\bf exogenen} Kanal zu den {\bf Endverbrauchern} des offenen
Marktes, und (ii) zu Teilen durch einen {\bf endogenen} Kanal
zu den {\bf Nachbarproduzenten}. Als Impulsgeber der zu 
betrachtenden G\"{u}terstr\"{o}me gilt die {\bf Wertsch\"{o}pfung}.

\medskip
\noindent
Das Leontiefsche Modell basiert auf den folgenden drei 
grundlegenden

\medskip
\noindent
{\bf Annahmen}:
%
\begin{enumerate}
\item F\"ur alle G\"{u}ter sei der funktionale Zusammenhang
zwischen den INPUT-- und den OUTPUT--Mengen von
{\bf linearer Natur} [vgl.\ Gl.~(\ref{lin})].

\item Die Mengenverh\"altnisse "`INPUT zu OUTPUT"' seien
\"uber den betrachteten Zeitraum hinweg {\bf konstant}
gehalten, die G\"{u}terstr\"{o}me also als {\bf station\"{a}r} 
(engl.: stationary) anzusehen.

\item Es herrsche {\bf \"okonomisches Gleichgewicht} (engl.: 
economic equilibrium): die
im betrachteten Zeitraum produzierte Mengen der G\"uter
entsprechen ihren in diesem Zeitraum (aufgrund
entsprechender Nachfrage) abgesetzten Mengen.
\end{enumerate}
%
Zur mathematischen Formulierung von Leontiefs quantitativem
Modell bedienen wir uns folgender

\medskip
\noindent
{\bf Vektor- und matrizenwertige Gr\"o\ss en}:\footnote{Die
Abk\"urzung ME steht f\"ur "`Mengeneinheiten"'.}
%
\begin{enumerate}

\item $\vec{q}$ --- {\bf Bruttoproduktionsvektor}
$\in \mathbb{R}^{n \times 1}$, $q_{i} \geq 0$~ME
\hfill (dim: ME)

(engl.: total output vector)

\item $\vec{y}$ --- {\bf Endnachfragevektor}
$\in \mathbb{R}^{n \times 1}$, $y_{i} \geq 0$~ME
\hfill (dim: ME)

(engl.: final demand vector)

\item $\mathbf{P}$ --- {\bf Input--Output--Matrix}
{\bf (Direktbedarfsmatrix)}
$\in \mathbb{R}^{n \times n}$, $P_{ij} \geq 0$
\hfill (dim: 1)

(engl.: input--output matrix)

\item $(\mathbf{1}-\mathbf{P})$ --- {\bf Technologiematrix}
$\in \mathbb{R}^{n \times n}$, invertierbar
\hfill (dim: 1)

(engl.: technology matrix)

\item $(\mathbf{1}-\mathbf{P})^{-1}$ --- {\bf Gesamtbedarfsmatrix}
$\in \mathbb{R}^{n \times n}$
\hfill (dim: 1)

(engl.: total demand matrix)

\item $\vec{v}$ --- {\bf Rohstoffvektor}
$\in \mathbb{R}^{m \times 1}$, $v_{i} \geq 0$~ME
\hfill (dim: ME)

(engl.: resource vector)

\item $\mathbf{R}$ --- {\bf Rohstoffverbrauchsmatrix}
$\in \mathbb{R}^{m \times n}$, $R_{ij} \geq 0$,
\hfill (dim: 1)

(engl.: resource consumption matrix)

\end{enumerate}
%
wobei $\mathbf{1}$ die {\bf $\boldsymbol{(n \times 
n)}$-Einheitsmatrix}
bezeichnet [vgl.\ Gl.~(\ref{einmatr})]. Beachten Sie, dass die
Komponenten aller Vektoren sowie der Input--Output--Matrix und
der Rohstoffverbrauchsmatrix nur {\em nichtnegative Werte (!)\/}
annehmen k\"onnen.

%%%%%%%%%%%%%%%%%%%%%%%%%%%%%%%%%%%%%%%%%%%%%%%%%%%%%%%%%%%%%%%%%%%
\section[Input--Output--Matrix und Rohstoffverbrauchsmatrix]%
{Input--Output--Matrix und Rohstoffverbrauchsmatrix}
\lb{sec:inoutmat}
%%%%%%%%%%%%%%%%%%%%%%%%%%%%%%%%%%%%%%%%%%%%%%%%%%%%%%%%%%%%%%%%%%%
Wir wenden uns nun den Definitionen und Eigenschaften der zwei
zentralen matrizenwertigen Gr\"o\ss en des Leontiefschen Modells
zu.

%------------------------------------------------------------------
\subsection{Input--Output--Matrix}
%------------------------------------------------------------------
Wir beginnen mit einer {\bf Bilanzierung} der OUTPUT--Mengen eines
jeden der $n$~Produktionssektoren, d.h.\ bestimmte
{\bf G\"utermengen} (engl.: amounts of goods) und die 
handelsbedingten, der Annahme nach station\"{a}ren {\bf 
G\"uterstr\"ome} (engl.: flows of goods), die sie ausbilden,
stehen im Fokus des wirtschaftstheoretischen Interesses. Wir nehmen
an, dass im betrachteten Zeitraum Sektor~$1$ von seinem Gut an sich
selbst die Menge $n_{11}$ lieferte, an Sektor~$2$ die Menge
$n_{12}$, an Sektor~$3$ die Menge $n_{13}$, usw.; schlie\ss lich
an Sektor~$n$ die Menge $n_{1n}$. An die Endverbraucher lieferte
Sektor~$1$ im gleichen Zeitraum die Menge $y_{1}$. In Summe ergibt
sich somit f\"ur die im betrachteten Zeitraum von Sektor $1$ brutto
produzierte (und, nach Annahme~3, auch ausgelieferte) G\"utermenge
der Wert
$q_{1}:=n_{11} + \ldots + n_{1j} + \ldots + n_{1n} + y_{1}$.
Analog verfahren wir mit den von den anderen $n-1$ der
$n$~Produktionssektoren brutto ausgelieferten G\"utermengen
$q_{2}$, $q_{3}$, \ldots, $q_{n}$. Somit erhalten wir das
Zwischenresultat
%
\begin{eqnarray}
q_{1} & = & n_{11} + \ldots + n_{1j} + \ldots + n_{1n} + y_{1} > 0
\\
 & \vdots & \nonumber \\
q_{i} & = & n_{i1} + \ldots + n_{ij} + \ldots + n_{in} + y_{i} > 0
\\
 & \vdots & \nonumber \\
q_{n} & = & n_{n1} + \ldots + n_{nj} + \ldots + n_{nn} + y_{n} > 0
\ .
\end{eqnarray}
%
Diese einfache Bilanz l\"asst sich \"ubersichtlich
in einer {\bf Input--Output--Tabelle} (engl.: input--output table) 
zusammenfassen:
%
\begin{center}
    \begin{tabular}[h]{c|ccccc|c|c}
    \hline\hline
    [Werte in ME] & Sektor~$1$ & $\cdots$ & Sektor~$j$ & $\cdots$ & Sektor~$n$ & Endverbraucher & $\Sigma$: Bruttomengen  \\
    \hline
    Sektor~$1$ & $n_{11}$ & $\ldots$ & $n_{1j}$ & $\ldots$ & $n_{1n}$ & $y_{1}$ & $q_{1}$ \\
    $\vdots$ & $\vdots$ & $\ddots$ & $\vdots$ & $\ddots$ & $\vdots$ & $\vdots$ & $\vdots$ \\
    Sektor~$i$ & $n_{i1}$ & \ldots & $n_{ij}$ & \ldots & $n_{in}$ & $y_{i}$ & $q_{i}$ \\
    $\vdots$ & $\vdots$ & $\ddots$ & $\vdots$ & $\ddots$ & $\vdots$ & $\vdots$ & $\vdots$ \\
    Sektor~$n$ & $n_{n1}$ & \ldots & $n_{nj}$ & \ldots & $n_{nn}$ & $y_{n}$ & $q_{n}$ \\
    \hline\hline
    \end{tabular}
\end{center}
%
In der ersten Spalte dieser Tabelle werden die Namen aller
$n$~{\bf G\"uterstromsender} (also der Quellen der verschiedenen
G\"uterstr\"ome; auch: Anbieter von G\"utern) angegeben,
in der ersten Zeile die Namen aller
$n+1$~{\bf G\"uterstromempf\"anger} (also der Senken der
verschiedenen G\"uterstr\"ome; auch: Verbraucher von G\"utern).
In der letzten Spalte finden wir die von jedem
der $n$~G\"uterstromsender im betrachteten
Zeitraum brutto ausgesendeten G\"utermengen.

\medskip
\noindent
Nun bilden wir f\"ur jeden der $n$~Produktionssektoren die
{\em nichtnegativen Mengenverh\"altniszahlen\/} (engl.: ratios)
%
\be
P_{ij} := \frac{\text{INPUT\ in\ ME\ von\ Sektor\ $i$\ an
\ Sektor $j$\ (im\ betrachteten\ Zeitraum)}}{
\text{OUTPUT\ in\ ME\ von\ Sektor\ $j$\ (im\ betrachteten
\ Zeitraum)}} \ ,
\ee
%
bzw.\ in kompakter Notation\footnote{Beachten Sie,
dass als Normierungsgr\"o\ss en in den $P_{ij}$-Quotienten die
Bruttoproduktionsmengen $q_{j}$ der G\"uterstromempf\"anger~$j$
auftreten, und {\em nicht\/} die $q_{i}$
der G\"uterstromsender~$i$. In letzterem Fall entspr\"achen die
$P_{ij}$ prozentualen Anteilen der $q_{i}$.}
%
\be
\fbox{$\displaystyle
P_{ij} := \frac{n_{ij}}{q_{j}} \ ,
$}
\ee
%
mit $i,j = 1, \ldots, n$. Wir betrachten diese
$n \cdot n = n^{2}$ Zahlenwerte als die Elemente
einer quadratischen Matrix $\mathbf{P}$ vom Format $(n \times n)$.
Allgemein hat diese quadratische Matrix das Aussehen
%
\be
\fbox{$\displaystyle
\mathbf{P} =
\left(\begin{array}{ccccc}
\frac{n_{11}}{n_{11}+\ldots+n_{1j}+\ldots+n_{1n}+y_{1}} &
\ldots &
\frac{n_{1j}}{n_{j1}+\ldots+n_{jj}+\ldots+n_{jn}+y_{j}} &
\ldots &
\frac{n_{1n}}{n_{n1}+\ldots+n_{nj}+\ldots+n_{nn}+y_{n}} \\
\vdots & \ddots & \vdots & \ddots & \vdots \\
\frac{n_{i1}}{n_{11}+\ldots+n_{1j}+\ldots+n_{1n}+y_{1}} &
\ldots &
\frac{n_{ij}}{n_{j1}+\ldots+n_{jj}+\ldots+n_{jn}+y_{j}} &
\ldots &
\frac{n_{in}}{n_{n1}+\ldots+n_{nj}+\ldots+n_{nn}+y_{n}} \\
\vdots & \ddots & \vdots & \ddots & \vdots \\
\frac{n_{n1}}{n_{11}+\ldots+n_{1j}+\ldots+n_{1n}+y_{1}} &
\ldots &
\frac{n_{nj}}{n_{j1}+\ldots+n_{jj}+\ldots+n_{jn}+y_{j}} &
\ldots &
\frac{n_{nn}}{n_{n1}+\ldots+n_{nj}+\ldots+n_{nn}+y_{n}}
\end{array}\right) \ ,
$}
\ee
%
und wird {\bf Input--Output--Matrix} (oder Direktbedarfsmatrix; 
engl.: input--output matrix) des zu analysierenden station\"{a}ren 
wirtschaftlichen Systems genannt.
%$P_{ij} \geq 0$ (Voraussetzung: $q_{i} > 0$).

\medskip
\noindent
Im relativ einfachen, \"uberschaubaren Fall mit nur $n=3$
Produktionssektoren nimmt die Input--Output--Matrix die
explizite Gestalt
%
\[
\displaystyle
\mathbf{P} =
\left(\begin{array}{ccc}
\frac{n_{11}}{n_{11}+n_{12}+n_{13}+y_{1}} &
\frac{n_{12}}{n_{21}+n_{22}+n_{23}+y_{2}} &
\frac{n_{13}}{n_{31}+n_{32}+n_{33}+y_{3}} \\
\frac{n_{21}}{n_{11}+n_{12}+n_{13}+y_{1}} &
\frac{n_{22}}{n_{21}+n_{22}+n_{23}+y_{2}} &
\frac{n_{23}}{n_{31}+n_{32}+n_{33}+y_{3}} \\
\frac{n_{31}}{n_{11}+n_{12}+n_{13}+y_{1}} &
\frac{n_{32}}{n_{21}+n_{22}+n_{23}+y_{2}} &
\frac{n_{33}}{n_{31}+n_{32}+n_{33}+y_{3}}
\end{array}\right)
\]
%
an. Wir betonen, dass f\"ur ein gegebenes wirtschaftliches System
eine Input--Output--Matrix $\mathbf{P}$ erst {\em nach Ablauf
eines betrachteten Referenzzeitraums\/} erstellt werden kann.

\medskip
\noindent
Der Nutzen des station\"{a}ren Leontiefschen Matrizenmodells liegt 
in der {\bf Extrapolation} (engl.: extrapolation) der G\"ultigkeit 
der Input--Output--Matrix $\mathbf{P}_{\text{Referenzperiode}}$ 
hinein in einen nachfolgenden, hinreichend begrenzten Zeitraum, 
d.h. in der {\em Annahme\/}
%
\be
\fbox{$\displaystyle
\mathbf{P}_{\text{kommende\ Periode}}
\approx
\mathbf{P}_{\text{Referenzperiode}} \ ,
$}
\ee
%
bzw. in Matrizenkomponenten
%
\be
\left.P_{ij}\right|_{\text{kommende\ Periode}}
= \left.\frac{n_{ij}}{q_{j}}\right|_{\text{kommende\ Periode}}
\approx
\left.\frac{n_{ij}}{q_{j}}\right|_{\text{Referenzperiode}}
= \left.P_{ij}\right|_{\text{Referenzperiode}}
\ .
\ee
%
Hierdurch werden Berechnungen von ben\"otigten
INPUT--Mengen f\"ur eine kommende Produktionsperiode aus bekannten
vorgegebenen OUTPUT--Mengen erm\"oglicht. Langj\"ahrige
Erfahrungen in der Praxis belegen, dass diese Vorgehensweise zu
durchaus brauchbaren Ergebnissen f\"uhrt. Dieser Art von
Berechnungen unterliegen lineare station\"{a}re 
G\"uterstromrelationen, deren Besprechung wir uns in K\"urze 
zuwenden wollen.

%------------------------------------------------------------------
\subsection{Rohstoffverbrauchsmatrix}
%------------------------------------------------------------------
Eine zweite zentrale matrizenwertige Gr\"o\ss e des Leontiefschen
Modells ist durch die {\bf Rohstoffverbrauchsmatrix} (engl.: 
resource consumption matrix) gegeben. Sie
kann als "`Kochrezept"' bzgl.\ der Mischungsverh\"altnisse
zwischen den INPUT--Gr\"o\ss en "`$m$~Rohstoffe"' und den
OUTPUT--Gr\"o\ss en "`$n$~G\"uter"' aufgefasst werden. Ihre
Elemente sind durch die Angaben
%
\be
\fbox{$\displaystyle
R_{ij} := \text{Ben\"otigte\ Menge\ in\ ME\ von\ Rohstoff\ $i$
\ pro\ eine\ Einheit\ in\ ME\ von\ Produkt\ $j$} \ ,
$}
\ee
%
mit $i=1,\ldots,m$ und $j=1,\ldots,n$, gegeben. Zeilenweise liegen
in der Matrix $\mathbf{R}$ also Informationen bzgl.\ der
$m$~Rohstoffe, spaltenweise Informationen bzgl.\ der $n$~G\"uter
vor. Beachten Sie, dass die Rohstoffverbrauchsmatrix~$\mathbf{R}$
im allgemeinen {\em keine (!)\/} quadratische (und somit keine
invertierbare) Matrix ist; generell hat sie das
Format~$(m \times n)$.

%%%%%%%%%%%%%%%%%%%%%%%%%%%%%%%%%%%%%%%%%%%%%%%%%%%%%%%%%%%%%%%%%%%
\section[Station\"{a}re lineare G\"uterstromrelationen]%
{Station\"{a}re lineare G\"uterstromrelationen}
\lb{sec:qstroeme}
%%%%%%%%%%%%%%%%%%%%%%%%%%%%%%%%%%%%%%%%%%%%%%%%%%%%%%%%%%%%%%%%%%%
%------------------------------------------------------------------
\subsection{G\"uterstr\"ome endogener INPUT zu gesamtem OUTPUT}
%------------------------------------------------------------------
Wir wenden uns nun der quantitativen Beschreibung der mit den
im betrachteten Zeitraum brutto produzierten
{\bf G\"utermengen}~$\vec{q}$ verbundenen station\"{a}ren {\bf 
G\"uterstr\"omen} zu. Nach Leontiefs Annahme~1 besteht zwischen 
den endogenen INPUT--Mengen~$\vec{q}-\vec{y}$ und den gesamten
OUTPUT--Mengen~$\vec{q}$ ein {\em linearer\/} funktionaler
Zusammenhang. Dieser wird durch die Matrizenrelation
%
\be
\lb{strom1}
\fbox{$\displaystyle
\vec{q}-\vec{y} = \mathbf{P}\vec{q}
\quad \Leftrightarrow \quad
q_{i}-y_{i} = \sum_{j=1}^{n}P_{ij}q_{j} \ ,
$}
\ee
%
mit $i = 1, \ldots, n$, zum Ausdruck gebracht, in welcher die
Input--Output--Matrix~$\mathbf{P}$ die Rolle der Zuordnung
(Abbildung!) zwischen beiden Mengen \"ubernimmt. Nach Annahme~2
bleiben die Elemente von $\mathbf{P}$ im betrachteten
Zeitraum konstant, die betrachteten G\"{u}terstr\"{o}me also 
station\"{a}r.

\medskip
\noindent
Die Relation~(\ref{strom1}) l\"asst sich auch sehr gut
aus einer alternativen, naturwissenschaftlich orientierten
Perspektive heraus motivieren. Die im
betrachteten Zeitraum brutto produzierten und, nach Annahme~3,
ausgelieferten Mengen~$\vec{q}$ der $n$~G\"uter
des Modells unterliegen einem {\bf Erhaltungssatz} (engl.: 
conservation law): "`im
betrachteten Zeitraum geht nichts von ihnen verloren."'
Quantitativ formuliert schreiben wir diese Feststellung
als die Aussage
%
\[
\underbrace{\vec{q}}_{\text{Bruttoproduktionsmengen}}
=\underbrace{\vec{y}}_{\text{Endverbrauchermengen (exogen)}}
+\underbrace{\mathbf{P}\vec{q}}_{\text{Zulieferungsmengen an}\ n
\ \text{Produktionssektoren (endogen)}}
\]
%
auf.

\medskip
\noindent
F\"ur Anwendungszwecke kann die zentrale station\"{a}re
G\"uterstromrelation~(\ref{strom1}) beliebig umgestellt werden.
Dabei greifen wir auf die Matrizenidentit\"at
$\vec{q}=\mathbf{1}\vec{q}$ zur\"uck, in welcher $\mathbf{1}$ die
$\boldsymbol{(n \times n)}${\bf -Einheitsmatrix} bezeichnet
[vgl.\ Gl.~(\ref{einmatr})].


%\pagebreak
%\medskip
\noindent
{\bf Beispiele:}

\medskip
\noindent
(i) vorgegeben: $\mathbf{P}$, $\vec{q}$

\medskip
\noindent
Dann ist
%
\be
\lb{strom12}
\vec{y} = (\mathbf{1}-\mathbf{P})\vec{q}
\quad \Leftrightarrow \quad
y_{i} = \sum_{j=1}^{n}(\delta_{ij}-P_{ij})q_{j} \ ,
\ee
%
mit $i = 1, \ldots, n$; $(\mathbf{1}-\mathbf{P})$ steht f\"ur
die invertierbare {\bf Technologiematrix} (engl.: technology 
matrix) des Systems.

\medskip
\noindent
(ii) vorgegeben: $\mathbf{P}$, $\vec{y}$

\medskip
\noindent
Dann ist
%
\be
\lb{strom13}
\vec{q} = (\mathbf{1}-\mathbf{P})^{-1}\vec{y}
\quad \Leftrightarrow \quad
q_{i} = \sum_{j=1}^{n}(\delta_{ij}-P_{ij})^{-1}y_{j} \ ,
\ee
%
mit $i = 1, \ldots, n$; $(\mathbf{1}-\mathbf{P})^{-1}$ steht f\"ur
die {\bf Gesamtbedarfsmatrix} (engl.: total demand matrix) des 
Systems, die die Inverse der Technologiematrix bildet.

%------------------------------------------------------------------
\subsection{G\"uterstr\"ome exogener INPUT zu gesamtem OUTPUT}
%------------------------------------------------------------------
Auch zwischen den exogenen INPUT--Mengen~$\vec{v}$ und den gesamten
OUTPUT--Mengen~$\vec{q}$ wird nach Annahme~1 ein {\em linearer\/}
funktionaler Zusammenhang vorausgesetzt. Dieser wird durch die
Matrizenrelation

%
\be
\lb{strom2}
\fbox{$\displaystyle
\vec{v} = \mathbf{R}\vec{q}
\quad\Leftrightarrow\quad
v_{i} = \sum_{j=1}^{n}R_{ij}q_{j} \ ,
$}
\ee
%
mit $i = 1, \ldots, m$, zum Ausdruck gebracht. Nach Annahme~2
bleiben die Elemente der
{\bf Rohstoffverbrauchsmatrix}~$\mathbf{R}$ im betrachteten
Zeitraum konstant, die betrachteten Rohstoffmengenstr\"{o}me also 
station\"{a}r.

\medskip
\noindent
Durch Kombination von Gl.~(\ref{strom2}) mit Gl.~(\ref{strom13})
erhalten wir die M\"oglichkeit, die zur Deckung von vorgegebenen
Nachfragemengen~$\vec{y}$ f\"ur die $n$~G\"uter (im betrachteten
Zeitraum) ben\"otigten Rohstoffmengen~$\vec{v}$ zu berechnen. Es
gilt
%
\be
\lb{strom22}
\vec{v} = \mathbf{R}\vec{q}
= \mathbf{R}(\mathbf{1}-\mathbf{P})^{-1}\vec{y}
\quad\Leftrightarrow\quad
v_{i} = \sum_{j=1}^{n}\sum_{k=1}^{n}R_{ij}
(\delta_{jk}-P_{jk})^{-1}y_{k} \ ,
\ee
%
mit $i = 1, \ldots, m$.

\medskip
\noindent
\underline{\bf GTR:} Resultate im Sinne der
Relationen~(\ref{strom12}), (\ref{strom13}) und (\ref{strom22})
lassen sich in F\"allen mit $n \leq 5$, f\"ur vorgegebene
Matrizen $\mathbf{P}$ und $\mathbf{R}$ und vorgegebene Vektoren
$\vec{q}$ oder $\vec{y}$, leicht unmittelbar mit dem GTR berechnen.

%%%%%%%%%%%%%%%%%%%%%%%%%%%%%%%%%%%%%%%%%%%%%%%%%%%%%%%%%%%%%%%%%%%
\section[Ausblick]%
{Ausblick}
\lb{sec:geldstroeme}
%%%%%%%%%%%%%%%%%%%%%%%%%%%%%%%%%%%%%%%%%%%%%%%%%%%%%%%%%%%%%%%%%%%
Das Leontiefsche Modell l\"asst sich in einfacher Weise auf
weiterf\"uhrende wirtschaftstheoretische \"Uberlegungen 
erweitern. F\"ur ein aus $n$ interdependenten Produktionssektoren
bestehendes, (idealisiert!) abgeschlossenes
wirtschaftliches System $G$, das nicht notwendigerweise 
station\"{a}r ist, k\"onnen die INPUT--Gr\"o\ss en
"`Rohstoffmengen"'~$\vec{v}$ sowie die OUTPUT--Gr\"o\ss en
"`Bruttoproduktionsmengen"'~$\vec{q}$ und
"`Nettoabsatzmengen"'~$\vec{y}$ jeweils mit Geldbetr\"agen
bewertet werden. Neben G\"utermengen und deren Str\"ome bzgl.~$G$
w\"ahrend eines vorgegebenen Zeitraums, k\"onnen dann zus\"atzlich
an erste gekoppelte {\bf Geldmengen} und die damit verbundenen
{\bf Geldstr\"ome} zeitlich und r\"aumlich analysiert werden.
Allerdings gen\"ugen Geldmengen, im Gegensatz zu G\"utermengen,
im allgemeinen {\em keinem\/} Erhaltungssatz; dieser Umstand kann
die mathematische Analyse von Geldstr\"omen erschweren. Denn im
Sinne von {\bf Wertsch\"opfung} {\em kann Geld in $G$ w\"ahrend
eines betrachteten Zeitraums auch erzeugt oder vernichtet
werden\/}; es muss nicht auf Zufluss nach bzw.\ Abfluss von $G$
beschr\"ankt bleiben. Zentral f\"ur derartige \"Uberlegungen
ist eine {\bf Bilanzgleichung} (engl.: balance equation) f\"ur die 
\"uber einen
gegebenen Zeitraum hinweg in $G$ enthaltene Geldmenge.
Diese Bilanzgleichung besitzt die in der {\bf Physik} sehr 
vertraute Struktur\footnote{Die Abk\"urzungen GE und ZE stehen 
f\"ur "`Geldeinheiten"' bzw.\ "`Zeiteinheiten"'.} (vgl. Herrmann 
(2003) \ct[S.~7ff]{her2003})
%
\[
\left(\begin{array}{c}
\text{{\bf Zeitliche \"Anderungsrate}} \\
\text{{\bf der Geldmenge} in}\ G\ \text{(in GE/ZE)}
\end{array}\right)
= \left(\begin{array}{c}
\text{{\bf Geldstrom}} \\
\text{nach}\ G\ \text{(in GE/ZE)}
\end{array}\right)
+ \left(\begin{array}{c}
\text{{\bf Gelderzeugungsrate}} \\
\text{in}\ G\ \text{(in GE/ZE)}
\end{array}\right) \ .
\]
%
Beachten Sie, dass, bezogen auf $G$, Geldstr\"ome und
Gelderzeugungsraten prinzipiell sowohl positive als auch
negative Vorzeichen haben k\"onnen. Zur konkreten quantitativen
Behandlung dieser Thematik ben\"otigt man als mathematische
Werkzeuge die Differenzial- und Integralrechnung, auf
welche wir in den Kapiteln~\ref{ch7} und~\ref{ch8}
einf\"uhrend eingehen werden. Wir tangieren hier unmittelbar
den relativ jungen, interdiziplin\"ar agierenden
Wissenschaftszweig der {\bf Econophysics} (vgl. z.B. Bouchaud
und Potters (2003) \ct{boupot2003}), dessen Zielsetzung
wir an dieser Stelle jedoch nicht weiter vertiefen wollen.


\medskip
\noindent
Durch die hier angedeutete Erweiterung des Leontiefschen
Modells k\"onnen  neben wirtschaftsbezogenen
Verh\"altnisgr\"o\ss en der Art
%
\[
\frac{\text{OUTPUT (in ME)}}{\text{INPUT (in ME)}} \ ,
\]
%
wie sie in der Einleitung betrachtet wurden, zus\"atzlich
{\bf Wirtschaftlichkeit} (engl.: economic efficiency) genannte 
dimensionslose Verh\"altnisgr\"o\ss en der Form
%
\[
\frac{\text{ERTRAG (in GE)}}{\text{KOSTEN (in GE)}}
\]
%
f\"ur verschiedene wirtschaftliche Systeme bzw.\ deren
Produktionssektoren gebildet und miteinander verglichen werden.
In Kapitel~\ref{ch7} werden wir auf diesen Aspekt kurz
zur\"uckkommen.

%%%%%%%%%%%%%%%%%%%%%%%%%%%%%%%%%%%%%%%%%%%%%%%%%%%%%%%%%%%%%%%%%%%
%%%%%%%%%%%%%%%%%%%%%%%%%%%%%%%%%%%%%%%%%%%%%%%%%%%%%%%%%%%%%%%%%%%
