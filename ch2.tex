%%%%%%%%%%%%%%%%%%%%%%%%%%%%%%%%%%%%%%%%%%%%%%%%%%%%%%%%%%%%%%%%%%%
%  File name: ch2.tex
%  Title:
%  Version: 12.09.2015 (hve)
%%%%%%%%%%%%%%%%%%%%%%%%%%%%%%%%%%%%%%%%%%%%%%%%%%%%%%%%%%%%%%%%%%%
%%%%%%%%%%%%%%%%%%%%%%%%%%%%%%%%%%%%%%%%%%%%%%%%%%%%%%%%%%%%%%%%%%%
\chapter[Matrizen]{Matrizen}
\lb{ch2}
%%%%%%%%%%%%%%%%%%%%%%%%%%%%%%%%%%%%%%%%%%%%%%%%%%%%%%%%%%%%%%%%%%%
In diesem Kapitel f\"uhren wir eine zweite, im Vergleich zu
Vektoren allgemeinere Klasse von mathematischen Objekten ein,
ebenfalls mit geeigneten Rechenoperationen und Rechenregeln.

%%%%%%%%%%%%%%%%%%%%%%%%%%%%%%%%%%%%%%%%%%%%%%%%%%%%%%%%%%%%%%%%%%%
\section[Matrizen als lineare Abbildungen]%
{Matrizen als lineare Abbildungen}
\lb{sec:matlinab}
%%%%%%%%%%%%%%%%%%%%%%%%%%%%%%%%%%%%%%%%%%%%%%%%%%%%%%%%%%%%%%%%%%%
Gegeben sei jetzt eine Ansammlung von $m \times n$ beliebigen
reellen Zahlen $a_{11}$, $a_{12}$ \ldots, $a_{ij}$, \ldots,
$a_{mn}$, die wir nach einer einfachen Vorschrift systematisch
zusammenfassen wollen.

\medskip
\noindent
\underline{\bf Def.:} Eine reelle
$\boldsymbol{(m\times n)}${\bf-Matrix}
(engl.: matrix) sei formal durch ein Zahlenschema der Struktur
%
\be
\mathbf{A}
:= \left(\begin{array}{cccccc}
   	a_{11} & a_{12} & \ldots & a_{1j} & \ldots & a_{1n} \\
   	a_{21} & a_{22} & \ldots & a_{2j} & \ldots & a_{2n} \\
    \vdots & \vdots & \ddots & \vdots & \ddots & \vdots \\
    a_{i1} & a_{i2} & \ldots & a_{ij} & \ldots & a_{in} \\
    \vdots & \vdots & \ddots & \vdots & \ddots & \vdots \\
    a_{m1} & a_{m2} & \ldots & a_{mj} & \ldots & a_{mn}
	\end{array}\right) \ ,
\ee
%
mit $a_{ij} \in {\mathbb R}$,	$i=1, \dots, m; j=1, \dots, n$,
definiert.

\noindent
Schreibweise: $\mathbf{A} \in \mathbb{R}^{m \times n}$.

\medskip
\noindent
Besondere Eigenschaften dieses Zahlenschemas sind:
%
\begin{itemize}
	\item $m$ ist die Anzahl der {\bf Zeilen} (engl.: rows) von 
	$\mathbf{A}$, $n$ die Anzahl der {\bf Spalten} (engl.: columns) 
	von $\mathbf{A}$.
	
	\item $a_{ij}$ sind die {\bf Elemente} (engl.: elements) von 
	$\mathbf{A}$;
	$a_{ij}$ steht im Kreuzungspunkt von $i$-ter Zeile und $j$-ter
	Spalte von $\mathbf{A}$.
	
	\item Elemente der $i$-ten Zeile bilden den {\bf Zeilenvektor}
	$\left(a_{i1}, a_{i2}, \ldots, a_{ij}, \ldots, a_{in}\right)$,
	Elemente der $j$-ten Spalte den {\bf Spaltenvektor}
	$\left(\begin{array}{c} a_{1j} \\ a_{2j} \\ \vdots \\ a_{ij} \\
	\vdots \\ a_{mj}
	\end{array}\right)$.
\end{itemize}
%
Spaltenvektoren sind also formal $(n\times 1)$-Matrizen,
Zeilenvektoren formal $(1\times n)$-Matrizen. Matrizen
mit {\em gleicher\/} Anzahl von Zeilen und Spalten, also $m=n$,
werden {\bf quadratische Matrizen} (engl.: quadratic matrices) 
genannt. Unter den quadratischen Matrizen hat die
$\boldsymbol{(n \times n)}${\bf-Einheitsmatrix} (engl.: unit 
matrix)
%
\be
\lb{einmatr}
\mathbf{1} := \left(\begin{array}{cccccc}
   	1 & 0 & \ldots & 0 & \ldots & 0 \\
   	0 & 1 & \ldots & 0 & \ldots & 0 \\
    \vdots & \vdots & \ddots & \vdots & \ddots & \vdots \\
    0 & 0 & \ldots & 1 & \ldots & 0 \\
    \vdots & \vdots & \ddots & \vdots & \ddots & \vdots \\
    0 & 0 & \ldots & 0 & \ldots & 1
	\end{array}\right)
\ee
%
ausgezeichneten Status.

\medskip
\noindent
Als n\"achstes kl\"aren wir, in welchem Sinne wir
$(m\times n)$-Matrizen als mathematische Objekte
begreifen wollen.

\medskip
\noindent
\underline{\bf Def.:} Eine Matrix $\mathbf{A} \in
\mathbb{R}^{m \times n}$ definiert \"uber die Rechenoperation
%
\bea
\lb{matabb}
\mathbf{A}\vec{x}
& := & \left(\begin{array}{cccccc}
   	a_{11} & a_{12} & \ldots & a_{1j} & \ldots & a_{1n} \\
   	a_{21} & a_{22} & \ldots & a_{2j} & \ldots & a_{2n} \\
    \vdots & \vdots & \ddots & \vdots & \ddots & \vdots \\
    a_{i1} & a_{i2} & \ldots & a_{ij} & \ldots & a_{in} \\
    \vdots & \vdots & \ddots & \vdots & \ddots & \vdots \\
    a_{m1} & a_{m2} & \ldots & a_{mj} & \ldots & a_{mn}
	\end{array}\right)
	\left(\begin{array}{c}
x_{1} \\ x_{2} \\ \vdots \\ x_{j} \\ \vdots \\ x_{n}
\end{array}\right) \nonumber \\
& := & \left(\begin{array}{c}
a_{11}x_{1}+a_{12}x_{2}+\ldots+a_{1j}x_{j}+\ldots+a_{1n}x_{n} \\
a_{21}x_{1}+a_{22}x_{2}+\ldots+a_{2j}x_{j}+\ldots+a_{2n}x_{n} \\
\vdots \\
a_{i1}x_{1}+a_{i2}x_{2}+\ldots+a_{ij}x_{j}+\ldots+a_{in}x_{n} \\
\vdots \\
a_{m1}x_{1}+a_{m2}x_{2}+\ldots+a_{mj}x_{j}+\ldots+a_{mn}x_{n}
\end{array}\right)
%= \left(
%\begin{array}{c}
%\sum_{j=1}^{n}a_{1j}x_{j} \\
%\sum_{j=1}^{n}a_{2j}x_{j} \\
%\vdots \\
%\sum_{j=1}^{n}a_{mj}x_{j}
%\end{array}
%\right)
=: \left(
\begin{array}{c}
y_{1} \\
y_{2} \\
\vdots \\
y_{i} \\
\vdots \\
y_{m}
\end{array}
\right)
= \vec{y} \ .
\eea
%
eine {\bf Abbildung} (engl.: mapping) $\mathbf{A}: \mathbb{R}^{n 
\times 1} \rightarrow \mathbb{R}^{m \times 1}$, d.h. eine 
Abbildung der Menge der $n$-komponentigen Spaltenvektoren (hier: 
$\vec{x}$) in die Menge der $m$-komponentigen Spaltenvektoren 
(hier: $\vec{y}$).

\medskip
\noindent
In loser Analogie repr\"asentieren $\vec{x}$ das "`Objekt"',
$\mathbf{A}$ die "`Kamera"' und $\vec{y}$ das "`Bild"'.

\medskip
\noindent
Da diese durch $\mathbf{A} \in \mathbb{R}^{m \times n}$
gegebene Art von Abbildungsvorschrift, f\"ur Vektoren
$\vec{x}_{1},\vec{x}_{2} \in \mathbb{R}^{n \times 1}$ und
reelle Zahlen $\lambda \in \mathbb{R}$, die zwei besonderen
Eigenschaften
%
\be
\lb{lin}
\fbox{$\displaystyle\begin{array}{c}
\mathbf{A}(\vec{x}_{1}+\vec{x}_{2})
= (\mathbf{A}\vec{x}_{1})+(\mathbf{A}\vec{x}_{2}) \\[5mm]
\mathbf{A}(\lambda\vec{x}_{1})
= \lambda(\mathbf{A}\vec{x}_{1})
\end{array}
$}
\ee
%
aufweist, stellen Matrizenabbildungen so genannte {\bf
lineare Abbildungen} (engl.: linear mappings) dar.\footnote{An 
dieser Stelle soll ausdr\"ucklich darauf hingewiesen werden, dass 
in konkreten praktischen Anwendungen quantitative Relationen die
Bedingungen (\ref{lin}) h\"aufig {\em nicht\/} erf\"ullen,
man es also mit {\em nichtlinearen Abbildungen\/} zu tun hat.
Lineare Methoden k\"onnen jedoch oft hilfreiche erste
N\"aherungen zur Verf\"ugung stellen.}

\medskip
\noindent
Jetzt gehen wir kurz auf die wichtigsten Rechenoperationen
f\"ur $(m \times n)$-Matrizen sowie deren Rechenregeln ein.

%%%%%%%%%%%%%%%%%%%%%%%%%%%%%%%%%%%%%%%%%%%%%%%%%%%%%%%%%%%%%%%%%%%
\section[Grundbegriffe und Grundrechenarten]%
{Grundbegriffe und Grundrechenarten}
\lb{sec:matrech}
%%%%%%%%%%%%%%%%%%%%%%%%%%%%%%%%%%%%%%%%%%%%%%%%%%%%%%%%%%%%%%%%%%%

\noindent
\underline{\bf Def.:}
{\bf Transponieren} (engl.: transpose) von Matrizen

\noindent
F\"ur $\mathbf{A} \in \mathbb{R}^{m \times n}$:
%
\be
\fbox{$\displaystyle
\mathbf{A}^{T}\!: \quad
a_{ij}^{T} := a_{ji} \ ,
$}
\ee
%
mit $i=1,\ldots,m$ und $j=1,\ldots,n$. Beachte, dass
$\mathbf{A}^{T} \in \mathbb{R}^{n \times m}$.

\medskip
\noindent
Beim Transponieren einer $(m \times n)$-Matrix werden ganz
einfach deren Zeilen mit deren Spalten (und umgekehrt)
vertauscht; in der entsprechenden Reihenfolge werden dabei die
Elemente der ersten Zeile die Elemente der ersten Spalte, usw. Hieraus folgt insbesondere die Eigenschaft
%
\be
(\mathbf{A}^{T})^{T} = \mathbf{A} \ .
\ee
%

\medskip
\noindent
F\"ur quadratische Matrizen (mit $m=n$) k\"onnen zwei
Spezialf\"alle auftreten:
%
\begin{itemize}
	\item Falls $\mathbf{A}^{T}=\mathbf{A}$, wird von einer
	{\bf symmetrischen Matrix} (engl.: symmetric matrix) gesprochen,
	\item Falls $\mathbf{A}^{T}=-\mathbf{A}$, von einer
	{\bf antisymmetrischen Matrix} (engl.: antisymmetric matrix).
\end{itemize}
%

\medskip
\noindent
\underline{\bf Def.:}
{\bf Addition} von Matrizen (engl.: addition of matrices)

\noindent
F\"ur $\mathbf{A}, \mathbf{B} \in \mathbb{R}^{m \times n}$
sei
%
\be
\fbox{$
\displaystyle
\mathbf{A} + \mathbf{B} =: \mathbf{C}\!: \quad
a_{ij} + b_{ij} =: c_{ij} \ .
$}
\ee
%
mit $i=1,\ldots,m$ und $j=1,\ldots,n$.

\medskip
\noindent
Beachte, dass nur Matrizen von {\em gleichem Format\/}
miteinander addiert werden k\"onnen.

\pagebreak
\medskip
\noindent
\underline{\bf Def.:}
{\bf Skalieren} von Matrizen (engl.: scaling of matrices)

\noindent
F\"ur $\mathbf{A} \in \mathbb{R}^{m \times n}$
und $\lambda \in \mathbb{R}\backslash \{0\}$ sei
%
\be
\fbox{$\displaystyle
\lambda\mathbf{A} =: \mathbf{C}\!: \quad
\lambda a_{ij} =: c_{ij} \ ,
$}
\ee
%
mit $i=1,\ldots,m$ und $j=1,\ldots,n$.

\medskip
\noindent
Beim Skalieren einer Matrix werden alle ihre Elemente
mit der gleichen reellen Zahl $\lambda$ multipliziert.

\medskip
\noindent
{\bf Rechenregeln f\"ur Addition und Skalierung
von Matrizen}

\noindent
F\"{u}r Matrizen $\mathbf{A}, \mathbf{B}, \mathbf{C}
\in \mathbb{R}^{m \times n}$ gelten:

\begin{enumerate}
	\item $\mathbf{A}+\mathbf{B} = \mathbf{B}+\mathbf{A}$
	\hfill ({\bf Kommutativit\"at})
	\item $\mathbf{A}+(\mathbf{B}+\mathbf{C})
	= (\mathbf{A}+\mathbf{B})+\mathbf{C}$
	\hfill ({\bf Assoziativit\"at})
	\item Zu $\mathbf{A}$ und $\mathbf{B}$ gibt es genau ein
	$\mathbf{Z}$, sodass $\mathbf{A}+\mathbf{Z}=\mathbf{B}$.
	\hfill ({\bf Umkehrbarkeit})
	\item $(\lambda\mu)\mathbf{A}=\lambda(\mu\mathbf{A})$
	mit $\lambda,\mu	\in \mathbb{R} \backslash \{0\}$ 
	\hfill ({\bf Assoziativit\"at})
	\item $1\mathbf{A}=\mathbf{A}$
	\hfill ({\bf Skalenfestlegung})
	\item $\lambda(\mathbf{A}+\mathbf{B})
	= \lambda\mathbf{A}+\lambda\mathbf{B}$;
	
	$(\lambda+\mu)\mathbf{A} = \lambda\mathbf{A}+\mu\mathbf{A}$
	mit $\lambda, \mu	\in \mathbb{R} \backslash \{0\}$
	\hfill ({\bf Distributivit\"at})
	
	\item $(\mathbf{A}+\mathbf{B})^{T} = \mathbf{A}^{T} + \mathbf{B}^{T}$ \hfill ({\bf Transpositionsregel 1})
	
	\item $(\lambda\mathbf{A})^{T} = \lambda\mathbf{A}^{T}$ mit $\lambda \in \mathbb{R} \backslash \{0\}$. \hfill ({\bf Transpositionsregel 2})
\end{enumerate}

\medskip
\noindent
Nun besprechen wir eine f\"ur Matrizen besonders n\"utzliche
Rechenoperation.

%%%%%%%%%%%%%%%%%%%%%%%%%%%%%%%%%%%%%%%%%%%%%%%%%%%%%%%%%%%%%%%%%%%
\section[Matrizenmultiplikation]%
{Matrizenmultiplikation}
\lb{sec:matmult}
%%%%%%%%%%%%%%%%%%%%%%%%%%%%%%%%%%%%%%%%%%%%%%%%%%%%%%%%%%%%%%%%%%%

\noindent
\underline{\bf Def.:}
F\"ur eine $(m \times n)$-Matrix $\mathbf{A}$ und eine
$(n \times r)$-Matrix $\mathbf{B}$
definiert das {\bf Matrizenprodukt} (engl.: matrix multiplication)
%
\be
\fbox{$
\displaystyle\begin{array}{c}
\mathbf{A}\mathbf{B} =: \mathbf{C} \\[5mm]
a_{i1}b_{1j}+\ldots+a_{ik}b_{kj}+\ldots
+a_{in}b_{nj}
=: \sum_{k=1}^{n}a_{ik}b_{kj} =: c_{ij} \ ,
\end{array}
$}
\ee
%
mit $i=1,\ldots,m$ und $j=1,\ldots,r$, eine
$(m \times r)$-Matrix $\mathbf{C}$, d.h. $\mathbf{C} \in
\mathbb{R}^{m \times r}$.

\medskip
\noindent
Das Element am Kreuzungspunkt der $i$-ten Zeile und $j$-ten
Spalte von $\mathbf{C}$ ist bestimmt durch die Vorschrift
%
\be
c_{ij} = \text{Skalarprodukt zwischen $i$-tem Zeilenvektor von
$\mathbf{A}$ und $j$-tem Spaltenvektor von $\mathbf{B}$} \ .
\ee
%
Beachten Sie bitte: die Definition der oben eingef\"uhrten
Matrizenmultiplikation h\"angt essenziell von der Tatsache ab, dass
die im Produkt {\em links stehende Matrix $\mathbf{A}$ genau so
viele (!) Spalten haben muss wie die rechts stehende Matrix
$\mathbf{B}$ Zeilen.\/} Anderfalls l\"asst sich die
Matrizenmultiplikation nicht sinnvoll definieren.

\medskip
\noindent
\underline{\bf GTR:} F\"ur eingespeicherte Matrizen ${\tt [A]}$
und ${\tt [B]}$, von passenden Formaten, berechnet sich das
Matrizenprodukt im Modus {\tt MATRIX} $\rightarrow$ {\tt NAMES}
\"uber ${\tt [A]*[B]}$.

%\vspace{5mm}
%\pagebreak
\medskip
\noindent
{\bf Rechenregeln f\"ur die Matrizenmultiplikation}

\noindent
F\"{u}r $\mathbf{A}, \mathbf{B}, \mathbf{C}$ reelle Matrizen
von entsprechendem Format gelten:

\begin{enumerate}
	\item $\mathbf{A}\mathbf{B} = \mathbf{0}$ ist m\"oglich
	mit $\mathbf{A}\neq\mathbf{0},\mathbf{B}\neq\mathbf{0}$.
	\hfill ({\bf Nullteiler})
	\item $\mathbf{A}(\mathbf{B}\mathbf{C})
	= (\mathbf{A}\mathbf{B})\mathbf{C}$
	\hfill ({\bf Assoziativit\"at})
	\item $\mathbf{A}
	\underbrace{\mathbf{1}}_{\in \mathbb{R}^{n \times n}}
	=\underbrace{\mathbf{1}}_{\in \mathbb{R}^{m \times m}}
	\mathbf{A}=\mathbf{A}$
	\hfill ({\bf Neutrale Elemente})
	\item $(\mathbf{A}+\mathbf{B})\mathbf{C}
	= \mathbf{A}\mathbf{C}+\mathbf{B}\mathbf{C}$
	
	$\mathbf{C}(\mathbf{A}+\mathbf{B})
	= \mathbf{C}\mathbf{A}+\mathbf{C}\mathbf{B}$
	\hfill ({\bf Distributivit\"at})
	\item $\mathbf{A}(\lambda\mathbf{B})
	=(\lambda\mathbf{A})\mathbf{B}
	=\lambda(\mathbf{A}\mathbf{B})$
	mit $\lambda \in \mathbb{R}$ 
	\hfill ({\bf Homogenit\"at})
	\item $(\mathbf{A}\mathbf{B})^{T}
	=\mathbf{B}^{T}\mathbf{A}^{T}$
	\hfill ({\bf Transponierungsregel}).
\end{enumerate}

%%%%%%%%%%%%%%%%%%%%%%%%%%%%%%%%%%%%%%%%%%%%%%%%%%%%%%%%%%%%%%%%%%%
%%%%%%%%%%%%%%%%%%%%%%%%%%%%%%%%%%%%%%%%%%%%%%%%%%%%%%%%%%%%%%%%%%%
