%%%%%%%%%%%%%%%%%%%%%%%%%%%%%%%%%%%%%%%%%%%%%%%%%%%%%%%%%%%%%%%%%%%
%  File name: learnoutEMQM.tex
%  Version: 08.03.2009 (hve)
%%%%%%%%%%%%%%%%%%%%%%%%%%%%%%%%%%%%%%%%%%%%%%%%%%%%%%%%%%%%%%%%%%%
\addcontentsline{toc}{chapter}{Qualifikationsziele des Moduls
(Auszug)}
%%%%%%%%%%%%%%%%%%%%%%%%%%%%%%%%%%%%%%%%%%%%%%%%%%%%%%%%%%%%%%%%%%%
\chapter*{Qualifikationsziele des Moduls (Auszug)}
%%%%%%%%%%%%%%%%%%%%%%%%%%%%%%%%%%%%%%%%%%%%%%%%%%%%%%%%%%%%%%%%%%%
Studierende, die dieses Modul erfolgreich absolviert haben, sind
in der Lage, die Rollenbilder des Managers im Kontext der
Unternehmung und anderer Organisationen sowie in der Gesellschaft
zu beschreiben und ausgew\"ahlte Aufgabenstellungen des
Managements mit Hilfe geeigneter und insbesondere auch
quantitativer Methoden zu l\"osen. Insbesondere sind sie in der
Lage,
%
\begin{itemize}
\item \ldots,
\item Aufgaben der Linearen Algebra und der Analysis zu l\"osen
und auf konkrete Fragestellungen der Managementlehre anzuwenden,
\item das Gelernte auf aktuelle Fragestellungen und in
ausgew\"ahlten Fallbeispielen anzuwenden und, auch im Hinblick
auf die eigene Verortung im Studium, kritisch zu hinterfragen.
\end{itemize}
%

%%%%%%%%%%%%%%%%%%%%%%%%%%%%%%%%%%%%%%%%%%%%%%%%%%%%%%%%%%%%%%%%%%%
%%%%%%%%%%%%%%%%%%%%%%%%%%%%%%%%%%%%%%%%%%%%%%%%%%%%%%%%%%%%%%%%%%%