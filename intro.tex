%%%%%%%%%%%%%%%%%%%%%%%%%%%%%%%%%%%%%%%%%%%%%%%%%%%%%%%%%%%%%%%%%%%
%  File name: intro.tex
%  Version: 19.08.2013 (hve)
%%%%%%%%%%%%%%%%%%%%%%%%%%%%%%%%%%%%%%%%%%%%%%%%%%%%%%%%%%%%%%%%%%%
\addcontentsline{toc}{chapter}{Einleitung}
%%%%%%%%%%%%%%%%%%%%%%%%%%%%%%%%%%%%%%%%%%%%%%%%%%%%%%%%%%%%%%%%%%%
\chapter*{Einleitung}
%%%%%%%%%%%%%%%%%%%%%%%%%%%%%%%%%%%%%%%%%%%%%%%%%%%%%%%%%%%%%%%%%%%
Diese Vorlesungsnotizen begleiten den quantitativen Teil des
Moduls "`0.1.1 EMQM: Einf\"uhrung in das Management und seine 
quantitativen Methoden"' aus dem 1.\ Studiensemester. Es werden
Ihnen darin ausgew\"ahlte, in der Praxis bew\"ahrte mathematische
Werkzeuge zur Behandlung wirtschaftswissenschaftlicher
Fragestellungen quantitativer Natur vorgestellt. Die zu
besprechenden Werkzeuge bilden das Fundament f\"ur einen
systematischen Umgang mit einfachen, im Rahmen eines
Bachelorstudiums auftretenden quantitativen Problemen. Es geht
dabei vorallem um das Verdeutlichen grundlegender Prinzipien in
der Anwendung quantitativer Methoden in den
Wirtschaftswissenschaften. Dar\"uber hinaus bilden sich gen\"ugend
thematische Ankn\"upfungspunkte f\"ur die Diskussion
weiterf\"uhrender quantitativer Methoden, die im Rahmen eines
vertiefenden wirtschaftswissenschaftlichen Masterstudiums
behandelt werden k\"onnen.

\medskip
\noindent
Zum Verst\"andnis der zu besprechenden Themen ben\"otigen Sie an
mathematischen Vorkenntnissen lediglich Oberstufenmathematik auf
Grundkursniveau, d.h. Sie sollten beispielsweise mit der
Bruchrechnung, der Potenzrechnung, den binomischen Formeln, dem
Berechnen des Schnittpunktes zweier Geraden, dem L\"osen
quadratischer Gleichungen, oder der Diskussion der Eigenschaften
von (ableitbaren) reellen Funktionen einer Variablen und der
Beschreibung ihres lokalen \"Anderungsverhaltens  vertraut sein.

\medskip
\noindent
Bei vielen der im Nachfolgenden angesprochenen Themen wird
darauf geachtet Ihnen darzulegen, welche Arbeitsschritte
bequemerweise von einem modernen grafikf\"ahigen Taschenrechner
({\bf GTR}) \"ubernommen werden k\"onnen. G\"angige aktuelle
Modelle, die h\"aufig in Schulen zum Einsatz kommen, sind
beispielsweise
%
\begin{itemize}
\item Texas Instruments {\em TI--84 plus\/},
\item Casio {\em CFX--9850GB PLUS\/}.
\end{itemize}
%

\medskip
\noindent
Hauptthema dieser Vorlesungsnotizen ist das Erlernen und Anwenden
von einfachen und effektiven mathematischen Methoden in einem 
wirtschaftswissenschaftlich orientierten Umfeld. Hierbei stehen
quantitative Aspekte von {\bf Prozessen} der Art
%
\[
\text{INPUT} \rightarrow \text{OUTPUT}
\]
%
im Vordergrund, f\"ur die spezielle {\bf funktionale Relationen}
zwischen einer bestimmten Menge von zahlenartigen INPUT--Gr\"o\ss en
und einer zweiten Menge von zahlenartigen OUTPUT--Gr\"o\ss en
betrachtet werden. Besonderes Interesse kommt hierbei (den
Zahlenwerten von) vergleichenden {\bf Verh\"altnisgr\"o\ss en}
der Struktur
%
\[
\frac{\text{OUTPUT}}{\text{INPUT}}
\]
%
zu. Im Allgemeinen wird man versuchen, die Werte der Gr\"o\ss en
INPUT und OUTPUT derart zu gestalten, dass daraus ein
f\"ur den Anwender m\"oglichst g\"unstiger Wert dieser
Verh\"altnisgr\"o\ss en resultiert. Viele der im
Nachfolgenden betrachteten quantitativen Fragestellungen
enthalten deshalb als zentralen Aspekt \"Uberlegungen zur
{\bf Optimierung} der Werte von Gr\"o\ss en. Die Optimierung kann
sich wahlweise als {\bf Minimierung} beziehungsweise
{\bf Maximierung} manifestieren. Das Thema Optimierung wird in
den nachfolgenden Kapiteln
%dieses Vorlesungsskriptes
einen Roten Faden bilden.

\medskip
\noindent
Teil I dieser Vorlesungsnotizen ist ausgew\"ahlten mathematischen
Methoden aus der {\bf Linearen Algebra} gewidmet und umfasst die
Kapitel~\ref{ch1} bis \ref{ch5}. Anwendung finden diese vorallem
bei der mengenm\"a\ss igen Betrachtung von G\"uterstr\"omen in
einfachen Wirtschaftsmodellen sowie der Linearen Optimierung.
In Teil II werden elementare \"Uberlegungen der
{\bf Finanzmathematik} kurz angerissen; diese sind in
Kapitel~\ref{ch6} zu finden.
Grundprinzipien der zur {\bf Analysis} geh\"orenden
Differenzial-- und Integralrechnung, und deren exemplarischer
Anwendung zur L\"osung wirtschaftswissenschaftlicher Probleme
quantitativer Natur, treten im Teil III auf und werden dort in den
Kapiteln~\ref{ch7} und \ref{ch8} diskutiert. Das Besprechen von
expliziten Rechenbeispielen zu allen angeschnittenen Themengebieten
bleibt den Vorlesungen und \"Ubungen dieser Veranstaltung
vorbehalten.

\medskip
\noindent
Diesen Vorlesungsnotizen unterliegende Fachb\"ucher, sowie
interessante begleitende und weiterf\"uhrende Texte, sind in der
Literaturliste am Ende \"ubersichtlich zusammengefasst.
Die Vorlesung orientiert sich vornehmlich an den Lehrb\"uchern
von Schrey\"ogg und Koch (2010)~\ct{schkoc2010},
Bauer {\em et al.\/} (2008)~\ct{bauetal2008},
Bosch (2003)~\ct{bos2003}, Schmalen und Pechtl
(2006)~\ct{schpec2006}, Auer und Seitz (2006)~\ct{auesei2006},
sowie H\"ulsmann {\em et al.\/} (2005)~\ct{hueetal2005}. Die
Relevanz mathematischer Methoden in der modernen Wirtschaft wird
verst\"andlich und eindrucksvoll in dem von Greuel {\em et al.\/}
(2008)~\ct{greetal2008} herausgegebenen Buch dargelegt.
\"Uber die Wirtschaftswissenschaften hinausgehende Standardwerke
der Angewandten Mathematik sind die B\"ucher von
Bronstein {\em et al.\/} (2005)~\ct{broetal2005} und
Arens {\em et al.\/} (2008)~\ct{areetal2008}. Wer sich zudem
inspiriert f\"uhlt, mehr \"uber Eleganz, \"Asthetik und Effizienz
mathematischer Methoden, in unterschiedlichsten Forschungs- und
Wissensbereichen, zu erfahren, der/m sei ein Blick in die Werke
von Penrose (2004)~\ct{pen2004}, Singh (2002)~\ct{sin2002}, Gleick
(1987)~\ct{gle1987} und Smith (2007)~\ct{smi2007} empfohlen. Alle
hier genannten B\"ucher sind Ihnen \"uber die Fachbibliothek der
Karlshochschule International University zug\"anglich.

\medskip
\noindent
Die *.pdf--Version dieser Vorlesungsnotizen enth\"alt aktive 
Querverweise zu biografischen Informationen der Internetportale 
The MacTutor History of Mathematics archive 
(\href{http://www-history.mcs.st-and.ac.uk}{{\tt 
www-history.mcs.st-and.ac.uk}}) und 
\href{http://en.wikipedia.org/wiki/Main_Page}{{\tt 
en.wikipedia.org}} \"uber Wissenschaftler, die pr\"agend an der 
historischen Entwicklung der hier besprochenen mathematischen 
Analyseverfahren beteiligt waren.

%%%%%%%%%%%%%%%%%%%%%%%%%%%%%%%%%%%%%%%%%%%%%%%%%%%%%%%%%%%%%%%%%%%
%%%%%%%%%%%%%%%%%%%%%%%%%%%%%%%%%%%%%%%%%%%%%%%%%%%%%%%%%%%%%%%%%%%