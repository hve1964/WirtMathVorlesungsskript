%%%%%%%%%%%%%%%%%%%%%%%%%%%%%%%%%%%%%%%%%%%%%%%%%%%%%%%%%%%%%%%%%%%
%  File name: ch5.tex
%  Title:
%  Version: 10.09.2015 (hve)
%%%%%%%%%%%%%%%%%%%%%%%%%%%%%%%%%%%%%%%%%%%%%%%%%%%%%%%%%%%%%%%%%%%
%%%%%%%%%%%%%%%%%%%%%%%%%%%%%%%%%%%%%%%%%%%%%%%%%%%%%%%%%%%%%%%%%%%
\chapter[Lineare Optimierung]{Lineare Optimierung}
\lb{ch5}
%%%%%%%%%%%%%%%%%%%%%%%%%%%%%%%%%%%%%%%%%%%%%%%%%%%%%%%%%%%%%%%%%%%
\hfill\hbox{\fbox{\vbox{\hsize=10cm
Der Inhalt dieses Kapitels ist relevant f\"ur das
Verst\"andnis der in Kapitel 5.3 und 5.4 des Lehrbuchs von
Schrey\"ogg und Koch (2010)~\ct{schkoc2010} behandelten
quantitativen Themen.
%Gleichfalls f\"ur Kapitel 5.3
%des Lehrbuchs von Schmalen und Pechtl 2006~\ct{schpec2006}.
}}}

\vspace{10mm}
\noindent
Vor dem Hintergrund des {\bf \"Okonomischen Prinzips} besprechen
wir in diesem Kapitel das L\"osen einer speziellen Klasse von
quantitativen Problemen, wie sie in nachfolgend beschriebener
Qualit\"at in der Praxis regelm\"a\ss ig auftreten.
Grunds\"atzlich unterscheidet man zwei m\"ogliche
Auspr\"agungen des \"Okonomischen Prinzips: (i) aus begrenzten
Ressourcen maximalen Nutzen zu ziehen, und (ii) ein vorgegebenes
Ziel mit minimalem Aufwand zu erreichen. Bez\"uglich der in
der Einleitung eingef\"uhrten, unseren \"Uberlegungen
unterliegenden Verh\"altnisgr\"o\ss e
$(\text{OUTPUT})/(\text{INPUT})$ geht es darum, den Wert dieses
Quotienten gr\"o\ss tm\"oglich zu gestalten. Dieses Ziel kann
(i)~bei festgehaltenem Nenner durch Vergr\"o\ss ern des Z\"ahlers,
oder (ii)~bei festgehaltenem Z\"ahler durch Verkleinern des Nenners
realisiert werden. Wir werden uns in diesem Kapitel vornehmlich mit
der ausf\"uhrlichen Behandlung von dem Fall~(i) zugeh\"origen
Ausgangssituationen besch\"aftigen.

%%%%%%%%%%%%%%%%%%%%%%%%%%%%%%%%%%%%%%%%%%%%%%%%%%%%%%%%%%%%%%%%%%%
\section[Einf\"uhrung]{Einf\"uhrung}
\lb{sec:lopeinf}
%%%%%%%%%%%%%%%%%%%%%%%%%%%%%%%%%%%%%%%%%%%%%%%%%%%%%%%%%%%%%%%%%%%
Zu Maximieren sei eine (nichtnegative) quantitative Gr\"o\ss e~$z$,
welche in {\em linearer Weise\/} funktional von einer festen
Anzahl~$n$ (nichtnegativer) variabler Gr\"o\ss en $x_{1}, \ldots,
x_{n}$ abh\"ange. Diesen $n$~variablen Gr\"o\ss en $x_{1}, \ldots,
x_{n}$ wiederum seien eine feste Anzahl~$m$ von mengenm\"a\ss igen
Einschr\"ankungen auferlegt, die ebenfalls in {\em linearer
Weise\/} von $x_{1}, \ldots, x_{n}$ abh\"angen. Die
$m$~Einschr\"ankungen, oder Restriktionen, sollen den Charakter
von vorgegebenen Kapazit\"atsobergrenzen haben.

\medskip
\noindent
\underline{\bf Def.:}
Gegeben seien eine Matrix~$\mathbf{A} \in
\mathbb{R}^{m \times n}$, ein Vektor~$\vec{b} \in
\mathbb{R}^{m \times 1}$, zwei Vektoren $\vec{c},\vec{x} \in
\mathbb{R}^{n \times 1}$, sowie eine Konstante
$d \in \mathbb{R}$. Eine Aufgabe der Form
%
\be
\fbox{$\displaystyle
\text{max}\left\{z=\vec{c}^{T}\cdot\vec{x}
+d\left|\mathbf{A}\vec{x}
\leq \vec{b}, \vec{x} \geq \vec{0}\right.\right\} \ ,
$}
\ee
%
in Komponentenschreibweise also
%
\begin{eqnarray}
\lb{lops1}
\text{max}\ z(x_{1},\ldots,x_{n})
= c_{1}x_{1} + \ldots + c_{n}x_{n} + d & & \\
a_{11}x_{1}+\ldots+a_{1n}x_{n} & \leq & b_{1} \\
%
 & \vdots & \nonumber \\
%
a_{m1}x_{1}+\ldots+a_{mn}x_{n} & \leq & b_{m} \\
%
x_{1} & \geq & 0 \\
%
 & \vdots & \nonumber \\
%
\lb{lopsm1}
x_{n} & \geq & 0 \ ,
\end{eqnarray}
%
wird {\bf Standard--Maximum--Aufgabe} der
{\bf linearen Optimierung} (engl.: standard maximum problem of 
linear programming) mit $n$~Variablen genannt.

\medskip
\noindent
Allgemein spricht man bei dieser Art von quantitativer
Fragestellung auch von einem {\bf Linearen Optimierungsproblem
(LOP)}. Die in der angef\"uhrten Definition auftretenden
Gr\"o\ss en und Relationen tragen die Bezeichnungen

\begin{itemize}

\item $z(x_{1},\ldots,x_{n})$ ---
{\bf Lineare Zielfunktion} (engl.: linear objective function) von 
$n$ Variablen,

\item $\mathbf{A}\vec{x} \leq \vec{b}$ ---
$m$ {\bf Restriktionen} (engl.: restrictions),

\item $\vec{x} \geq \vec{0}$ ---
$n$ {\bf Nichtnegativit\"atsbedingungen} (engl.: non-negativity 
constraints).

\end{itemize}

\noindent
\underline{\bf Bem.:} In %, zum gerade Beschriebenen,
analoger Weise  l\"asst sich auch eine
{\bf Standard--Minimum--Aufgabe} der
linearen Optimierung formulieren, welche durch
%
\[
\text{min}\left\{z=\vec{c}^{T}\cdot\vec{x}
+d\left|\mathbf{A}\vec{x}
\geq \vec{b}, \vec{x} \geq \vec{0}\right.\right\}
\]
%
mathematisch beschrieben wird. Die Komponenten
des Vektors $\vec{b}$ sind hierbei
als vorgegebene Kapazit\"atsuntergrenzen aufzufassen.

\medskip
\noindent
F\"ur eine gegebene lineare Zielfunktion $z(x_{1},\ldots,x_{n})$
wird die Menge aller Punkte $\vec{x} = (x_{1},\ldots,x_{n})^{T}$
mit der Eigenschaft
%
\be
\fbox{$\displaystyle
z(x_{1},\ldots,x_{n}) = C = \text{konstant} \in \mathbb{R}
$}
\ee
%
eine {\bf Isoquante} (engl.: isoquant) dieser linearen 
Zielfunktion genannt.
Isoquanten linearer Zielfunktionen stellen bei $n=2$ Variablen
Geraden, bei $n=3$ Variablen Ebenen, bei $n=4$ Variablen 
euklidische R\"aume, und bei $n\geq 5$ Variablen euklidische 
Hyperr\"aume dar.

\medskip
\noindent
In den einfachsten F\"allen von LOP h\"angt die lineare
Zielfunktion~$z$ nur von $n=2$ Variablen $x_{1}$ und
$x_{2}$ ab. Eine anschauliche und effektive Methode des L\"osens
von LOP aus dieser Klasse stellen wir im n\"achsten Abschnitt
vor.

%%%%%%%%%%%%%%%%%%%%%%%%%%%%%%%%%%%%%%%%%%%%%%%%%%%%%%%%%%%%%%%%%%%
\section[Grafisches L\"osen von Problemen mit zwei Variablen]%
{Grafisches L\"osen von Problemen mit zwei Variablen}
\lb{sec:lopgraf}
%%%%%%%%%%%%%%%%%%%%%%%%%%%%%%%%%%%%%%%%%%%%%%%%%%%%%%%%%%%%%%%%%%%
Die systematische Vorgehensweise des grafischen L\"osens von
MAX--LOP mit $n=2$~Variablen umfasst folgende Arbeitsschritte:
%
\begin{enumerate}

\item Aufstellen der {\bf linearen Zielfunktion}
%
$$
z(x_{1},x_{2})=c_{1}x_{1}+c_{2}x_{2}+d
$$
%
in Abh\"angigkeit von den Variablen $x_{1}$ und $x_{2}$.

\item Bestimmen des durch die $m$ Restriktionen
festgelegten {\bf zul"assigen Bereichs} (engl.: feasible region) 
$D$ von $z$ in der $x_{1},x_{2}$--Ebene.
Konkret entspricht $D$ dem Definitionsbereich von $z$
(siehe Kapitel~\ref{ch6}).

\item Einzeichnen der durch den Ursprung ($0=x_{1}=x_{2}$)
der $x_{1},x_{2}$--Ebene verlaufenden Projektion der
{\bf Isoquanten} der linearen Zielfunktion~$z$,
die (falls $c_{2} \neq 0$) durch die Gleichung
%
\[
x_{2} = -(c_{1}/c_{2})x_{1}
\]
%
beschrieben wird.

\item Einzeichnen der durch den konstanten Gradienten
%
$$
(\boldsymbol{\nabla} z)^{T}
=\left(\begin{array}{c}\frac{\ptl z}{\ptl x_{1}} \\
\frac{\ptl z}{\ptl x_{2}}\end{array}\right)
=\left(\begin{array}{c} c_{1} \\ c_{2}\end{array}\right)
$$
%
gegebenen {\bf Optimierungsrichtung} (engl.: direction of 
optimisation) f\"ur $z$ im Ursprung der $x_{1},x_{2}$--Ebene.

\item {\bf Paralleles Verschieben} (engl.: parallel displacement) 
in der $x_{1},x_{2}$--Ebene
der Projektion der Ursprungsisoquanten der linearen
Zielfunktion~$z$ entlang der
Optimierungsrichtung~$(\boldsymbol{\nabla} z)^{T}$ \"uber den
zul\"assigen Bereich~$D$ hinweg, bis dieser gerade noch
ber\"uhrt wird.

\item Bestimmen der {\bf optimalen L\"osung} (engl.: optimal 
solution) $(x_{1{\rm O}},
x_{2{\rm O}})$ als Schnittpunkt bzw. -menge zwischen
verschobener Projektion der Ursprungsisoquanten und dem
oberen Rand von $D$.

\item Berechnen des {\bf optimalen Wertes} (engl.: optimal value) 
der linearen Zielfunktion~$z_{{\rm O}}=z(x_{1{\rm O}},x_{2{\rm O}})$ aus der optimalen L\"osung.

\item Feststellen von m\"oglichen {\bf Restkapazit\"aten} (engl.: 
remaining resources) durch Einsetzen der optimalen L\"osung 
$(x_{1{\rm O}}, x_{2{\rm O}})$ in die $m$~Restriktionen.

\end{enumerate}
%
Generell findet man, dass die zul\"assigen Bereiche~$D$ von
linearen Zielfunktionen~$z$ mit $n=2$ Variablen~$x_{1}$ und
$x_{2}$, sofern sie {\em nichtleer und beschr\"ankt\/} sind,
{\bf Vielecken} (Drei-, Vier-, F\"unf-, Sechs-,
\ldots ecke) in der $x_{1},x_{2}$--Ebene entsprechen. Die
gesuchten Extremwerte der linearen Zielfunktionen~$z$ finden
sich in diesen F\"allen immer in den Ecken bzw.\ auf den
R\"andern der zul\"assigen Bereiche~$D$. Ist der zul\"assige
Bereich~$D$ eines LOP leer, so ist dieses unl\"osbar.
Bleibt der zul\"assige Bereich~$D$ eines LOP unbeschr\"ankt,
ist letzteres m\"oglicherweise auch unl\"osbar; es kommt
hierbei jedoch auf die konktete Fragestellung an.

\medskip
\noindent
\underline{\bf Bem.:} Zum grafischen L\"osen einer
Standard--Minimum--Aufgabe der linearen
Optimierung parallelverschiebt man die Projektion der
Ursprungsisoquanten von~$z$ in die $x_{1},x_{2}$--Ebene
entlang der Optimierungsrichtung~$(\boldsymbol{\nabla} z)^{T}$
soweit, bis der zul\"assige Bereich~$D$ gerade erreicht (und
ber\"uhrt) wird. Die optimale L\"osung ergibt sich sodann als
Schnittpunkt bzw. -menge zwischen verschobener
Ursprungsisoquante und dem {\em unteren Rand\/} von $D$.

%%%%%%%%%%%%%%%%%%%%%%%%%%%%%%%%%%%%%%%%%%%%%%%%%%%%%%%%%%%%%%%%%%%
\section[Dantzigscher Simplexalgorithmus]%
{Dantzigscher Simplexalgorithmus}
\lb{sec:lopsimplex}
%%%%%%%%%%%%%%%%%%%%%%%%%%%%%%%%%%%%%%%%%%%%%%%%%%%%%%%%%%%%%%%%%%%
Ein entscheidender Nachteil der grafischen L\"osungsmethode von
LOP ist ihre Limitierung auf F\"alle mit nur $n=2$~Variablen.
In der Praxis hat man es jedoch h\"aufig mit linearen
Optimierungsproblemen zu tun, die von mehr als zwei variablen
Gr\"o\ss en abh\"angen. Zur systematischen Behandlung derartiger
F\"alle hat der US-amerikanische Mathematiker 
\href{http://www-groups.dcs.st-and.ac.uk/~history/Biographies/Dantzig_George.html}{George
Bernard Dantzig (1914--2005)} in den 40er-Jahren des zwanzigsten
Jahrhunderts einen leistungsf\"ahigen Algorithmus entwickelt, der
sich recht einfach auf Computern programmieren l\"asst; vgl. 
Dantzig (1949,1955~\ct{dan1949,dan1955}.

\medskip
\noindent
Unter einem {\bf Simplex} (engl.: simplex) versteht man in der 
Mathematik ein konvexes Polyeder (engl.: convex polyhedron), d.h.\ 
einen vieleckigen K\"orper mit
entsprechend vielen Kanten und Ecken in mehr als zwei Dimensionen.
Die zul\"assigen Bereiche von h\"oherdimensionalen LOP bilden
im allgemeinen solche Simplexe. Da auch in diesen F\"allen die
optimale L\"osung eines LOP, verbunden mit einem Extremwert der
linearen Zielfunktion, in einer Ecke oder auf einer Kante
des Simplex liegen wird, ist Dantzigs so genannter
{\bf Simplexalgorithmus} darauf ausgelegt, systematisch die
Ecken und Kanten eines zul\"assigen Bereichs mit m\"oglichst
wenigen Schritten abzufahren, um die gesuchte Extremall\"osung
aufzusp\"uren und als optimal zu identifizieren.

\medskip
\noindent
Ausgangspunkt ist eine Standard--Maximum--Aufgabe in der Form
(\ref{lops1})--(\ref{lopsm1}). Die die $m$ zu ber\"ucksichtigenden
Restriktionen darstellenden Ungleichungen werden mithilfe von
$m$ nichtnegativen so genannten {\bf Schlupfvariablen} (engl.: 
slack variables) $s_{1}, \ldots, s_{m}$ in $m$ lineare Gleichungen 
umgewandelt.
Aufgabe der Schlupfvariablen ist es, m\"ogliche Differenzen
zwischen den Werten der linken und der rechten Seiten der $m$
Ungleichungen in sich zu vereinen. Zusammen mit der
Definitionsgleichung f\"ur die lineare Zielfunktion~$z$
ergibt sich somit ein System von $1+m$ linearen Gleichungen
f\"ur $1+n+m$ Variablen $z, x_{1},
\ldots, x_{n}, s_{1}, \ldots, s_{m}$, gegeben durch

\medskip
\noindent
{\bf Lineares Maximierungsproblem in kanonischer Form}
\nopagebreak
%
\begin{eqnarray}
\lb{lopk1}
z - c_{1}x_{1} - c_{2}x_{2} - \ldots - c_{n}x_{n} & = & d \\
%
a_{11}x_{1} + a_{12}x_{2} + \ldots + a_{1n}x_{n} + s_{1} & = & b_{1} \\
%
a_{21}x_{1} + a_{22}x_{2} + \ldots + a_{2n}x_{n} + s_{2} & = & b_{2} \\
%
 & \vdots & \nonumber \\
%
\lb{lopkm1}
a_{m1}x_{1} + a_{m2}x_{2} + \ldots + a_{mn}x_{n} + s_{m} & = & b_{m} \ .
\end{eqnarray}
%
Dieses LGS vom Format $(1+m) \times (1+n+m)$ ist somit
{\em unterbestimmt\/}, und deshalb h\"ochstens
{\em mehrdeutig\/} l\"osbar. Der $(1+n+m)$-dimensionale
allgemeine L\"osungsvektor
%
\be
\vec{x}_{L} = \left(z_{L}, x_{1, L}, \ldots, x_{n, L},
s_{1, L}, \ldots, s_{m, L}\right)^{T}
\ee
%
enth\"alt also $n$ frei w\"ahlbare Variablen.
Es ist wichtig, sich dieser Tatsache bewusst zu sein.
Sie impliziert, dass wir, so das betrachtete LGS \"uberhaupt
l\"osbar ist, die {\em Auswahl\/} unter mehreren L\"osungen
haben. Wir k\"onnen uns also unter allen L\"osungen dieses LGS
die f\"ur unsere Zwecke Optimale aussuchen. Dantzigs
Simplexalgorithmus stellt ein Werkzeug bereit, diese optimale
L\"osung systematisch herzuleiten.

\medskip
\noindent
Wir beginnen mit dem \"Ubertragen der Koeffizienten und der rechten Seiten (RS) des oben angef\"uhrten unterbestimmten LGS (\ref{lopk1})--(\ref{lopkm1}) in ein geeignetes Zahlentableau.

\medskip
\noindent
{\bf Anfangssimplextableau}
%
\be
\lb{tableau1}
\begin{array}{c|rrcr|cccc|r}
z & x_{1} & x_{2} & \ldots & x_{n} & s_{1} & s_{2} & \ldots
& s_{m} & \text{RS} \\
\hline
1 & -c_{1} & -c_{2} & \ldots & -c_{n} & 0 & 0 & \ldots & 0 & d \\
\hline
0 & a_{11} & a_{12} & \ldots & a_{1n} & 1 & 0 & \ldots & 0 & b_{1} \\
0 & a_{21} & a_{22} & \ldots & a_{2n} & 0 & 1 & \ldots & 0 & b_{2} \\
\vdots & \vdots & \vdots & \ddots & \vdots & \vdots & \vdots &
\ddots & \vdots & \vdots \\
0 & a_{m1} & a_{m2} & \ldots & a_{mn} & 0 & 0 & \ldots & 1 & b_{m}
\end{array}
\ee
%
In diesem Zahlentableau (engl.: initial simplex tableau) 
unterscheiden wir so genannte {\bf Basisvariablen} (engl.: basis 
variable) von so genannten {\bf Nichtbasisvariablen} (engl.: 
non-basis variables).
Basisvariablen sind jene, in deren Spalten $(1+m)$-komponentige
kanonische Einheitsvektoren [vgl.\ Gl.~(\ref{kanbasis})] zu
finden sind; hiervon gibt es $1+m$ St\"uck. Nichtbasisvariablen
sind jene, in deren Spalten {\em keine\/} kanonischen
Einheitsvektoren enthalten sind; hiervon gibt es $n$ St\"uck.
Die komplette Basis ist also $(1+m)$-dimensional.
Anf\"anglich sind $z$ und die $m$ Schlupfvariablen
$s_{1}, \ldots, s_{m}$ Basisvariablen, die $n$ eigentlichen
Variablen $x_{1}, \ldots, x_{n}$ Nichtbasisvariablen
[vgl.\ das Anfangstableau~(\ref{tableau1})]. Die zugeh\"orige
so genannte (erste) {\bf Basisl\"osung} (engl.: basis solution) hat generell das Aussehen
%
\[
\vec{x}_{B_{1}} = \left(z_{B_{1}},
x_{1,B_{1}}, \ldots, x_{n, B_{1}},
s_{1, B_{1}}, \ldots, s_{m, B_{1}}\right)^{T}
= \left(d, 0, \ldots, 0,
b_{1}, \ldots, b_{m}\right)^{T} \ ,
\]
%
denn jeder der $n$ Nichtbasisvariablen kann der Einfachheit
halber der Wert Null zugeordnet werden, da ihre Werte frei
gew\"ahlt werden d\"urfen. Basisl\"osungen sind immer {\em
spezielle L\"osungen\/} des unterbestimmten
LGS~(\ref{lopk1})--(\ref{lopkm1}), %, hier also
--- des linearen Maximierungsproblems in kanonischer Form.

\medskip
\noindent
Ziel des Simplexalgorithmus ist es, m\"oglichst viele der
$n$ eigentlichen Variablen $x_{1}, \ldots, x_{n}$ mit in
die $(1+m)$-dimensionale Basis aufzunehmen und gegen
jeweils eine der Schlupfvariablen $s_{1}, \ldots, s_{m}$
auszutauschen, um dadurch sukzessiv g\"unstigere spezielle
L\"osungsvektoren f\"ur das Optimierungsproblem zu
konstruieren. Letztendlich verbirgt sich hinter dem
Simplexalgorithmus eine spezielle Form des in
Kapitel~\ref{ch3} behandelten Gau\ss schen Algorithmus,
mit fest vorgeschriebenen Schritten der \"aquivalenten
Umformung des Ausgangs-LGS (\ref{lopk1})--(\ref{lopkm1})
bzw.\ des Anfangstableaus~(\ref{tableau1}). Diese Schritte
(engl.: algebraic simplex operations) lassen sich wie folgt 
zusammenfassen:

\medskip
\noindent
{\bf Simplexschritte}
\nopagebreak
\begin{itemize}
\item[S1:] Gilt im aktuellen Tableau $-c_{j} \geq 0$
f\"ur alle $j \in \{1,\ldots,n\}$? Falls ja, ist die
Basisl\"osung {\bf optimal}. {\it ENDE}. Andernfalls gehe zu S2.

\item[S2:] W\"ahle {\bf Pivotspaltenindex} (engl.: pivot column 
index) $j^{*} \in
\{1,\ldots,n\}$ mit $-c_{j^{*}}:= \text{min}\{-c_{j}|j
\in \{1,\ldots,n\}\} < 0$.

\item[S3:] Gibt es ein $i^{*} \in \{1,\ldots,m\}$ mit
$a_{i^{*}j^{*}} > 0$? Falls nein, ist die Zielfunktion
nach oben unbeschr\"ankt. {\it ENDE}. Andernfalls gehe zu S4.

\item[S4:] W\"ahle {\bf Pivotzeilenindex} (engl.: pivot row index) 
$i^{*}$ mit
$a_{i^{*}j^{*}} > 0$ und $b_{i^{*}}/a_{i^{*}j^{*}}
:=\text{min}\{b_{i}/a_{i^{*}j^{*}}|a_{i^{*}j^{*}} > 0,
i \in \{1,\ldots,m\}\}$. F\"uhre einen {\bf Pivotschritt} (engl.: 
pivot operation) mit dem {\bf Pivotelement} (engl.: pivot element) 
$a_{i^{*}j^{*}}$ durch. Gehe zu S1.
\end{itemize}

\medskip
\noindent
Wenn das Endtableau des Simplexalgorithmus (engl.: final simplex
tableau) gefunden ist, wird den
Nichtbasisvariablen wiederum der Wert Null zugeordnet. Die Werte
der finalen Basisvariablen, also der gesuchten optimalen
L\"osung eines gegebenen LOP, werden dann aus dem Endtableau
\"uber r\"uckwertige Substitution zeilenweise von unten nach oben
ausgerechnet. Schlupfvariablen mit positiven Werten im Endtableau
geben unmittelbar Auskunft \"uber bestehende Restkapazit\"aten.

%%%%%%%%%%%%%%%%%%%%%%%%%%%%%%%%%%%%%%%%%%%%%%%%%%%%%%%%%%%%%%%%%%%
%%%%%%%%%%%%%%%%%%%%%%%%%%%%%%%%%%%%%%%%%%%%%%%%%%%%%%%%%%%%%%%%%%%
